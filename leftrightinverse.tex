
\aaa{computation of rank}
\exe Compute the rank of the following matrix
$$
\m{-3}111,
1{-3}11,
11{-3}1,
111{-3}.
$$

\aaa




\aaa{Left null space}

\begin{defi}
We define the left null space of $A$ to be $Null(A^T)$
\end{defi}

\a\aa
\begin{rem}
Strictly speaking, the left null space for a $m × n$ matrix $A$ should be 
$$
\{x: xA = 0\}
$$
However, $x$ in the left null space is a $1 × m$ row vector, so not an element in $ℝ^m$ (since $ℝ^m$ are set for column vectors). So \x{for adapting elementary level learners}, the textbook just transposes the whole expression
$$
xA = 0 ⟺   A^Tx^T  = 0
$$
to define left null space as $Null(A^T)$. You might find it unnatural.
\end{rem}

\a\aa
\begin{defi}
We define the row space of $A$ to be $Col(A^T)$
\end{defi}
\a\aa
\begin{rem}
Strictly speaking, the row space for a $m × n$ matrix $A$ should be 
$$
\{y: xA = y\}
$$
However, $y$ in the row space is a $1 × n$ row vector, so not an element in $ℝ^n$ (since $ℝ^n$ are set for column vectors). So \x{for adapting elementary level learners}, the textbook just transposes the whole expression
$$
xA = y ⟺   A^Tx^T  = y^T
$$
to define row space as $Col(A^T)$. You might find it unnatural.
\end{rem}
\a\aa
We address the importance of the four subspaces here
$$
Col(A^T), ␣ Null(A^T), ␣ Col(A), ␣ Null(A).
$$
But in the future, you will see

\begin{prop}
If $Col(B^T) = Null(A)$, then $Col(A^T)=Null(B)$.
\end{prop}
In other words, the \x{row space} and \x{null space} are \x{determined each other}. This means they have the same amount of information.

\[rem]{We will prove this proposition afterwards.}
\a\aa
\exe Suppose the null space of a matrix $A$ is spanned by the following vectors
$$
\m 1,2,1., ␣ \m 1,0,0.
$$
Please determine the row space of $A$. 

\textbf{Solution}: It is same of asking, if a row space of matrix $B$ is given by the vector 
$$
\m 1,2,1., ␣ \m 1,0,0.
$$
then determine the null space of $B$, which is just
$$
“Col“(A^T) = “Null“(B) = “Null“\m121,100. = “span“\left\{\m0,{-1},2.\right\}
$$
\a{Row space}
Let me copy the same proposition again, but replace $A$ by $A^T$:
\begin{prop}
If $Col(B^T) = Null(A^T)$, then $Col(A)=Null(B)$.
\end{prop}
It says that the left null space of $A$ if determined by the column space of $A$.

\a{Conclusion for four fundamental subspaces}

Therefore we have the following relation for the four fundamental subspaces.

\vfill

Let $A$ be $ m × n$ matrix of rank $r$.

$$
\underbrace{“Col“(A)}_{\substack{
“Column space“ ⊆ ℝ^m\\
“dim “ = r
}}
\underset{“determine each other“}{\leftrightarrow}
\underbrace{“Null“(A^T)}_{\substack{
“Left Null space“ ⊆ ℝ^m\\
“dim “ = m - r
}}
$$

$$
\underbrace{“Null“(A)}_{\substack{
“Null space“ ⊆ ℝ^n\\
“dim “ = n-r
}}
\underset{“determine each other“}{\leftrightarrow}
\underbrace{“Col“(A^T)}_{\substack{
“Row space“ ⊆ ℝ^n\\
“dim “ = r
}}
$$
Therefore, \x{$“Col“(A)$ and $“Null“(A)$ have included all informations} of four fundamental subspaces.
\aaa



\aaa{Column spaces}

From this part, we require you to know the definition of one set contains in another:
\vfill
$W ⊆ V$ ⟺   ($x ∈ W ⟹  x ∈ V$)
\vfill
The definition of two sets are equal is given by
\vfill
$W = V$ ⟺  ($W ⊆ V$ and $V ⊆ W$).
\a\aa
Recall the definition of Null space and column space. 
\begin{defi}
Let $A$ be an $m × n$ matrix. So $A$ has $m$ rows and $n$ columns.
\begin{itemize}
\item $“Null“(A) = \{x ∈ ℝ^n| Ax = 0\} ⊆ ℝ^n$
\item $“Col“(A) = \{y ∈ ℝ^m| y = Ax \} ⊆ ℝ^m$
\end{itemize}
\end{defi}

\begin{prop}Suppose $A$ is $m × n$ matrix
\begin{itemize}
\item $“Null“(A)=\{\vec 0\}$ ⟺  Columns of $A$ \li
\item $“Col“(A)=ℝ^m$ ⟺  Columns of $A$ \sws
\end{itemize}
\end{prop}
\a\aa

\begin{prop}
Let $A$ be $m × n$ and $B$ be $n × q$ matrices, then both $“Col“(A)$ and $“Col“(AB)$ are subsets of $ℝ^m$. In particular,
$$
“Col“(AB) ⊆ 
“Col“(A) 
$$
\end{prop}
$$
y ∈ “Col“(AB)  ⟹   y = ABx “ for some “x
$$
$$
⟹   y = At “ for “ t  = Bx 
$$
$$
 ⟹   y ∈ “Col“(A).
$$

\a\aa
One may understand $
“Col“(AB) ⊆ 
“Col“(A) 
$
by follwing:
\vfill
\[rem]{
Columns of $AB$ are obtained from linear combination of columns of the left factor $A$. So $
“Col“(AB) ⊆ 
“Col“(A) 
$
}

\exe Can you give some example that $“Col“(AB)≠“Col“(A)$?
$$
\underbrace{\m 100,010,001.}_A\underbrace{\m1,0,0. }_B = \underbrace{\m1,0,0. }_{AB}
$$

\x{Conclusion}: Multiplying a \x{\color{blue}right-factor}, column space \x{not getting bigger}.


\a\aa

\begin{prop}
Let $A$ be $m × n$ and $B$ be $n × q$ matrices, then both $“Null“(B)$ and $“Null“(AB)$ are subsets of $ℝ^n$. In particular,
$$
  “Null“(AB)  ⊇ 
“Null“(B)
$$
\end{prop}
$$
x ∈ “Null“(B)  ⟹   Bx=0 ⟹  ABx = 0 ⟹  x ∈ “Null“(AB)
$$

\a\aa
The null space are linear relations. In the following example
$$
B = \m123,123,123.
$$
The second column is double of the first, so 
$$
\m {-1},2,0. ∈ “Null“(B).
$$
Note that doing whatever on rows, would not change the relation. For example, we double the second row and delete the last row
$$
\underbrace{\m 100,020.}_A\underbrace{ \m123,123,123.}_B =\underbrace{ \m123,246.}_{AB}
$$
Therefore $“Null“(B) ⊆ “Null“(AB)$
\a\aa


\x{Conclusion}: Multiplying a \x{\color{red}left-factor}, null space \x{not getting smaller}.

%%%%%%%%%%%%%%%%%%%%%%%%%%%%%%%%%%%%%%%%%%%%%%%%%%%%%%%%%%%
%%%%%%%%%%%%%%%%%%%%%%%%%%%%%%%%%%%%%%%%%%%%%%%%%%%%%%%%%%%
%%%%%%%%%%%%%%%%%%%%%%%%%%%%%%%%%%%%%%%%%%%%%%%%%%%%%%%%%%%
%%%%%%%%%%%%%%%%%%%%%%%%%%%%%%%%%%%%%%%%%%%%%%%%%%%%%%%%%%%
%%%%%%%%%%%%%%%%%%%%%%%%%%%%%%%%%%%%%%%%%%%%%%%%%%%%%%%%%%%

\a{Factors that not changing column space}
Multiplying factors might makes the null space bigger or column space smaller. What kind of factor would \x{not} change them?
\a\aa
\x{Warm up Question}: What kind of matrix have \x{\color{red} LEFT INVERSE}? (Choose two)


\vfill
A. Columns \li
\vfill
B. Columns \sws
\vfill
C. Rows \li
\vfill
D. Rows \sws.
\vfill
$$\m100,010.\m10,01,00. = \m 10,01.$$
\a\aa
\x{Warm up Question}: What kind of matrix have \x{\color{blue} RIGHT INVERSE}? (Choose two)

\vfill
A. Columns \li
\vfill
B. Columns \sws
\vfill
C. Rows \li
\vfill
D. Rows \sws.
\vfill
$$\m100,010.\m10,01,00. = \m 10,01.$$
\a\aa
\begin{prop}
If the right factor $B$ have \x{\color{blue}right inverse}, then 
$$
“Col“(AB) = “Col(A)“
$$
\end{prop}

$$
“Col“(AB) ⊆  “Col(A)“
$$
Let $C$ be the right inverse of $B$, then $ABC = A$
$$
“Col“(A) = “Col“(ABC) ⊆ “Col“(AB)  ⊆ “Col“(A).
$$

\a{Factors that not changing null space}

\begin{prop}
If left factor $A$ have \x{\color{red}left inverse}, then 
$$
“Null“(AB) = “Null(B)“
$$
\end{prop}

\a\aa
\exe Please find a matrix $C$ such that 
$$
“Col“(C) = “span“\{\m1,2,0,1.\m1,1,1,1.\}
$$
$$
“Null“(C)=“span“\{\m1,2,3.\}
$$

Our method is to write $C=AB$, where $A$ has \x{\color{red} left inverse} and $B$ has \x{\color{blue} right inverse}.
\a\aa
If we put
$$
A = \m 11,21,01,11. ⟹   “Col“(A)= “span“\{\m1,2,0,1.\m1,1,1,1.\}.
$$

$$
B = \m2{-1}0,30{-1}. ⟹   “Null“(B)=“span“\{\m1,2,3.\}
$$
\vfill

Clear $A$ \x{\color{red} has left inverse} and $B$ \x{\color{blue} has right inverse}. If we put $C=AB$, we gonna get $“Col“(C)=“Col“(A)$ and $“Null“(C)=“Null“(B)$.
\vfill
$$
C = AB = \m 11,21,01,11.\m2{-1}0,30{-1}.= \m5{-1}{-1},7{-2}{-1},30{-1},5{-1}{-1}.
$$


\a\aa
Recall that \x{invertible row operations } including row adding, row multiplying and row switching, all of them are the same as left multiplying an \x{invertible left factor}
$$
\underbrace{\m01,10.}_{\substack{“Row switching“\\“invertible“}}A
␣ 
\underbrace{\m1k,01.}_{\substack{“Row adding“\\“invertible“}}A
␣ 
\underbrace{\m 90,01.}_{\substack{“Row multiplying“\\“invertible“}}A
$$
Call the elementary matrices as $E$. Since $E$ is invertible, in particular it have \x{\color{blue} right inverse}. So
$$
“Null“(EA) = “Null“(A)
$$
\a\aa

\begin{prop} \x{Invertible} Row operations does not change the row space and null space of $A$ \end{prop}

\[rem]{
However, deleting a row might change the row space or null space. Since deleting is not an invertible row action
}
\a\aa
\exe After several row operations, we reduce a matrix into the following
$$
\m 1\square9,2\square0,3\square0. 
\overset{“After invertible row operations“}\longrightarrow
\m 120,001,000.
$$
Please fill in the missing number in $\square$!
\a\aa
A similar result holds for columns as well.
\begin{prop} \x{Invertible} Column operations does not change the column space and left null space of $A$ \end{prop}

\aaa



\aaa{Left inverse and right inverse}

\a{Left Inverse}
\x{\color{red}Thin} full rank matrix ⟺   rank = number of columns ⟺  columns \li ⟺  having \x{\color{red}left inverse}



$$
\underbrace{\m100,010.}_{“\color{blue}fat matrix“}\underbrace{\m 10,01,00.}_{“\color{red}thin matrix“}=\m10,01.
$$

But left inverse not unique

$$
\m10a,01b.\m 10,01,00. = \m10,01.
$$

\a{Right Inverse}

\x{\color{blue}Fat} full rank matrxi ⟺   rank = number of rows ⟺   columns \sws ⟺  having \x{\color{blue}right inverse}

$$
\underbrace{\m100,010.}_{“\color{blue}fat matrix“}\underbrace{\m 10,01,00.}_{“\color{red}thin matrix“}=\m10,01.
$$

But right inverse may not unique

$$
\m100,010.\m 10,01,ab. = \m10,01.
$$
\a{Invertible}

\x{\color{purple}Square} full rank matrix ⟺  rank = number of cols = number of rows ⟺  columns \li and \sws (\bas) ⟺  have \x{\color{purple}both inverse}



\vfill\x{Question:} Is the inverse of square matrix unique?
\vfill\x{Question:} Is the {\color{red}left inverse} and {\color{blue}right inverse} of a square matrix the same?
%If $AB = I_n$ and $CA= I_n$? Is that true $B=C$? 


%The inverse of square matrix is unique. We solve this slides.




\aaa



\aaa{Inverse Pair}
This slides study left and right inverse.
$$
\underbrace{\m1000,0100,0010.}_A\underbrace{\m100,010,001,000.}_B = \m 100,010,001.
$$
\a\aa
A is $m × n$ matrix, B is $n × m$ matrix, 
$$
AB = I_m.
$$

\begin{itemize}
\item $B$ is right inverse of $A$
\item $A$ is left inverse of $B$
\end{itemize}

$$ “Col“(A) ⊇   “Col“(AB) = “Col“(I_m) = ℝ^m $$  ⟹   columns of $A$ must \sws
$$ “Null“(B) ⊆    “Null“(AB) = “Null“(I_m) = \{\vec 0\} $$ ⟹  columns of $B$ must \li

\a\aa

%We would like to understand the pair $A$ and $B$ by their column and null spaces. Now already
%$$
%“Col“(A) = ℝ^m ␣ “Null“(B) = \{\vec 0\}
%$$
%What about $“Null“(A)$ and $“Col“(B)$? For this, we study
%
%$$
%“Col“(B)=“Col“(BAB) ⊆  “Col“(BA) ⊆  “Col“(B)
%$$
%Therefore
%$$
%“Col“(B)=“Col“(BA)
%$$
%Similarly
%$$
%“Null“(A) = “Null“(ABA) ⊆ “Null“(BA) ⊆  “Null“(A).
%$$
%Therefore
%$$
%“Null“(A)=“Null“(BA)
%$$
%Therefore $BA$ has all information of $“Col“(B)$ and $“Null“(A)$.
%
%\a\aa
\begin{prop}
Suppose $A$ is $m × n$ matrix and $B$ is $n × m$ matrix such that $Null(B)=\{\vec 0\}$ and $Col(A)=ℝ^m$. Then
$$
AB “ is \inve “ ⟺   “Null“(A) ∩ “Col“(B) =\{\vec 0\}.
$$
\end{prop}

(This is a homework)

Hint: ⟹  is easy. For ⟸   part, show that $“Null“(AB)=0$, and that $AB$ is a square matrix.
\a\aa

\[rem]{We have important observation! If $AB$ is \inve 
$$ (AB)^{-1}A “ is a \x{\color{red} left inverse} of “B.  $$
$$ B(AB)^{-1} “ is a \x{\color{blue} right inverse} of “A.  $$
}
\a\aa
\begin{cor}
If $AB=I_m$, then $“Null“(A) ∩ “Col“(B) =\{\vec 0\}$
\end{cor}


\a\aa
\begin{thm}
Let $B$ be a full rank \x{\color{red}thin} $n × m$ matrix ($n ≥ m$, $“rank“(B) = m$). Then $“Col“(B) ⊆ ℝ^n$. Let $W ⊆ ℝ^n$ be a subspace such that
$$
W ∩ “Col“(B)= \{\vec 0\}␣
“dim“(W) = n-m.
$$
Then there exists a unique left inverse $A$, such that 
$$
AB = I_m, ␣ 
“Null“(A) = W.
$$
\end{thm}
In other wrods, the left inverse is not unique, but it is uniquely determined by specifying a complement as a null space.
\vfill
In this case, a basis in Null(A) together with columns of B gives a basis for the whole space $ℝ^n$. You may think of finding left inverse as extending \li vectors into \bas.


\a\aa
\begin{thm}
Let $A$ be a full rank \x{\color{blue}fat} $m × n$ matrix ($n ≥ m$, $“rank“(A) = m$). Then $“Null“(A) ⊆ ℝ^n$. Let $W ⊆ ℝ^n$ be a subspace such that
$$
W ∩ “Null“(A)= \{\vec 0\}␣
“dim“(W) = n-m.
$$
Then there exists a unique {\color{blue}right inverse} $B$, such that 
$$
AB = I_m, ␣ 
“Col“(B) = W.
$$
\end{thm}
In other wrods, the {\color{blue}right inverse} is not unique, but it is uniquely determined by specifying a complement as a column space.
\vfill
In this case, a basis in Null(A) together with columns of B gives a basis for the whole space $ℝ^n$. You may think of finding right inverse as extending \li row-vectors into \bas.
\a\aa
\exe Let $A$ be the following matrix
$$
\m123,111.
$$
Find the right inverse $B$ of $A$ such that 
$$
“Col“(B) “ is “ \spab \m1,1,1.,\m1,2,3.
$$

\vfill

Try $B = \m 11,12,13.$ and calculate
$$
AB = \m123,111.\m11,12,13.=\m6{14},36.
$$
\a\aa
The right inverse of $A$ is then given by 
$$
B(AB)^{-1} 
= \m 11,12,13.\m6{14},36.^{-1}
$$
$$
= \m 1{-1},10,11.\m6{2},30.^{-1}
$$
$$
= \m 4{-1},10,{-2}1.\m0{2},30.^{-1}
$$
$$
= \m {-\frac12}{\frac43},0{\frac13},{\frac12}{-\frac23}.
$$
\a{Inverse pair and space complement}

Therefore, studying a pair $A,B$ with $AB = I_m$ is the same as studying two subspaes $W_A$ and $W_B$ with 
\[itemize]{
\item $W_A ∩ W_B = \{\vec 0\}$
\item $“dim“(W_A)=n-m$
\item $“dim“(W_B)=m$
\item $“Null“(A) = W_A$
\item $“Col“(B) = W_B$
}
We call $W_A$ and $W_B$ complement to each other.

\vfill

Studying a pair of subspaces $W_A,W_B ⊆ ℝ^n$ with $W_A ∩ W_B = \{\vec 0\}$ and $“dim“(W_A)+“dim“(W_B)=n$ is the same as studying \x{projection matrices}(wil defined later)
\a\aa
Othorgonal Complement

\[tikzpicture]{[scale=0.3]

%	\draw[ultra thick,red] (-6,0)--(6,0) node[right]{$U_1$};
	\draw[ultra thick,red] (0,-4)--(0,0);
	\draw[fill=green,opacity=0.5] (-7,-1)--(1,-1)--(7,1) node[right]{$W_A$} --(-1,1)--(-7,-1);
        \draw[ultra thick,red] (0,0)--(0,5) node[above]{$W_B$};
%	\draw[ultra thick,red] (-6,-2)--(6,2) node [right]{$U_2$};
	\draw[fill=red] (0,0) circle[radius=0.1] node[right]{$\vec 0$};
	}

Complement which not necessarily orthogonal

\[tikzpicture]{[scale=0.3]
\draw[ultra thick](-3,-6)--(0.9,-0.8);
\draw[opacity=0.8,fill=cyan] (-6,-2)--(6,-2)--(10,0)--(-2,0)--(-6,-2);

\draw[fill=cyan] (.9+.75*5,-.8+5) circle[radius=0.1,fill=cyan] node[left]{};
\draw[fill=pink] (6+.75*5-0.2,-2+5) circle[radius=0.1] node[right]{};

\draw[radius=0.05,fill=red] (0.9,-.8) circle;
\draw[ultra thick](.9,-.8)--(6,6);

}

\a\aa
Now, if $AB = I_m$, we are curious about $BA$. 
\vfill
Since $B$ has \x{\color{red}left inverse}
$$
W_A = “Null“(A) = “Null(BA)“ 
$$
\vfill
Since $A$ has \x{\color{blue}right inverse}
$$
W_B = “Col“(B) = “Col(BA)“
$$

Therefore, $BA$ has all information of  $W_A$ and $W_B$!

\aaa


