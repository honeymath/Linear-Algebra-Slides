
\aaa{The exponential function}
Recall what we called exponential

$$
e^x:= \lim \left(1+\frac xn\right)^n.
$$
\a\aa
Recall the Binomial Expansion
$$
(1+x)^n = 1 + \frac n{1!} x +\frac{n(n-1)}{2!} x^2 +\cdots+\frac{n(n-1)(n-2)}{3!} x^3+\cdots
$$
\a\aa
We may think $e^x = (1+\frac x{{∞}})^{∞}$, the binomial expansion is then 
$$
\left(1+\frac x{{∞}}\right)^{∞} = 1 + \frac {∞}{1!} \frac x{∞} +\frac{{∞}({∞}-1)}{2!} \frac {x^2}{{∞}^2} +\cdots+\frac{{∞}({∞}-1)({∞}-2)}{3!} \frac{x^3}{{∞}^3}+\cdots
$$

Note that
$$
\frac{∞}{∞} = \frac{∞-1}{∞} = \frac{∞-2}{∞} = \cdots = 1.
$$
So finally we have the \x{Binomial Expansion}

$$
e^x = 1+x+\frac{x^2}{2!}+\frac{x^3}{3!}+\cdots
$$
\aaa


\aaa{Motivation of Exponential}
Consider the differential equation
$$
y' = 2y.
$$
How to understand it?

\vfill
The differential equation \( y' = 2y \) represents a scenario where the rate of change of \( y \) (speed) is proportional to its current value (distance).


\a\aa
Consider \( y(t) \) as a function of \( t \). When we examine the value of \( y \) at a very small time, say \( \frac{1}{n} \), we notice that \( y\left(\frac{1}{n}\right) \) is almost the same as \( y(0) \), since the time interval is extremely small.

\vfill
Given our differential equation \( y' = 2y \), the speed of change at \( t = 0 \) is \( 2y(0) \). %We can approximate the change in \( y \) over this small interval \( \frac{1}{n} \) as the speed times the time interval. Hence, the approximation is:

$$
y\left(\frac{1}{n}\right) \approx y(0) + \frac{1}{n} \cdot 2y(0)
$$

This leads to the expression:

$$
y\left(\frac{1}{n}\right) = y(0)\left(1 + \frac{2}{n}\right)
$$

Note that we have factored out \( y(0) \) to highlight the proportional change in \( y \) over the interval \( \frac{1}{n} \).
\a\aa

%Now, let's extend this idea to a larger time frame. We can iteratively apply the same reasoning for each small interval of time \( \frac{1}{n} \). 
Since the rate of change \( y' = 2y \) remains consistent over each interval, we can approximate the value of \( y \) at each step.
\vfill
For the next small interval, starting from \( \frac{1}{n} \), we use the updated value of \( y \) and apply the same formula:
\vfill
$$
y\left(\frac{2}{n}\right) = y\left(\frac{1}{n}\right)\left(1 + \frac{2}{n}\right)
$$

Repeating this process \( n \) times to reach time \( t = 1 \), we have:

$$
y(1) = y(0)\left(1 + \frac{2}{n}\right)^n
$$

\a\aa
As \( n \) becomes very large, this expression approaches the form of an exponential function:

$$
y(1)=\lim_{n \rightarrow \infty} y(0)\left(1 + \frac{2}{n}\right)^n = y(0)e^{2}
$$

%This demonstrates that the solution to the differential equation \( y' = 2y \) is indeed an exponential function. The value of \( y(t) \) grows exponentially over time, with the growth rate continuously proportional to its current value.
\a\aa
% Extending to General Time s Using Small Time Steps:
For finidng $y(s)$, consider extending our approach to a general time \( s \) using small time steps \( \frac{1}{n} \).
\vfill
% Small Time Step Analysis:
%For each step, we approximate \( y \) using our earlier logic, where the change in \( y \) is proportional to its current value. 
%This is akin to the approximation \( y\left(\frac{1}{n}\right) = y(0)\left(1 + \frac{2}{n}\right) \).

% Reaching General Time s in ns Steps:
To reach a general time \( s \), we perform \( ns \) such steps. After \( ns \) steps, the approximation is:
$$
y(s) \approx y(0)\left(1 + \frac{2}{n}\right)^{ns}
$$

% Taking the Limit as n -> infinity:
As \( n \) becomes very large, this expression approaches the form of an exponential function:
$$
\lim_{n \rightarrow \infty} y(0)\left(1 + \frac{2}{n}\right)^{ns} = y(0)e^{2s}
$$

% Conclusion: Solving the Differential Equation:
\a\aa
This result  
$$ y(s) = y(0)e^{2s} $$
 the solution to the differential equation \( y' = 2y \) .

\aaa
 
\aaa{Matrix Differential Equation}

% Introduction to Vector Differential Equation:
Consider the vector differential equation \( \mathbf{y}' = A\mathbf{y} \), where \( \mathbf{y} \) is a vector and \( A \) is a matrix. This equation describes a system where the rate of change of each component of the vector \( \mathbf{y} \) is determined by a linear combination of all components of \( \mathbf{y} \), with the coefficients provided by the matrix \( A \).

% Defining Matrix Exponential:
To solve this equation, we introduce the concept of a matrix exponential, denoted as \( e^{At} \). The matrix exponential is defined as:
$$
e^{At} = \lim_{n ⟶  ∞}(1+\frac An)^n 
$$
It has binomial expansion
$$
e^{At}=I + At + \frac{1}{2!}A^2t^2 + \frac{1}{3!}A^3t^3 + \cdots
$$
where \( I \) is the identity matrix.

% Solving the Vector Differential Equation:
Now, let's solve the differential equation \( \mathbf{y}' = A\mathbf{y} \) using a method analogous to the scalar case:

\a\aa
% Step 1: Small Time Step Analysis:
Consider a small time step \( \frac{1}{n} \). For each step, we approximate the change in \( \mathbf{y} \) using the matrix \( A \) and the small time step, leading to:
$$
\mathbf{y}\left(\frac{1}{n}\right) \approx \mathbf{y}(0) + \frac{1}{n} A\mathbf{y}(0)
$$

% Step 2: Iterative Process for General Time s:
To reach a general time \( s \), perform \( ns \) such steps. The approximation after \( ns \) steps is:
$$
\mathbf{y}(s) \approx \left(I + \frac{1}{n} A\right)^{ns}\mathbf{y}(0)
$$

% Step 3: Taking the Limit as n -> infinity:
As \( n \) becomes very large, this expression approaches the form of the matrix exponential function:
$$
\lim_{n \rightarrow \infty} \left(I + \frac{1}{n} A\right)^{ns}\mathbf{y}(0) = e^{As}\mathbf{y}(0)
$$

% Conclusion:
This shows that the solution to the vector differential equation \( \mathbf{y}' = A\mathbf{y} \) is given by the matrix exponential \( e^{As}\mathbf{y}(0) \), illustrating how the state of the system \( \mathbf{y}(s) \) evolves over time.

\a\aa
% Introduction to the Example System:
Consider the matrix \( A = \begin{pmatrix} 0 & -1 \\ 2 & 3 \end{pmatrix} \) and the vector differential equation \( \mathbf{y}' = A\mathbf{y} \), where \( \mathbf{y} = \begin{pmatrix} y_1 \\ y_2 \end{pmatrix} \) is a vector. This equation represents a system of linear differential equations.

% Writing Down the System Explicitly:
The system can be written explicitly as:
$$
\begin{pmatrix} y_1' \\ y_2' \end{pmatrix} = \begin{pmatrix} 0 & -1 \\ 2 & 3 \end{pmatrix} \begin{pmatrix} y_1 \\ y_2 \end{pmatrix}
$$

% Breaking Down Componentwise:
This breaks down to the following componentwise equations:
\begin{align*}
y_1' &= -y_2 \\
y_2' &= 2y_1 + 3y_2
\end{align*}

\a\aa
As we discussed, the solution of this system is given by
$$
\mathbf{y}(s)= e^{As}\mathbf{y}(0)
$$
Now let us directly calculate $e^{As}$.

Firstly, $A$ has characteristic polynomial $“det“(tI - A) = (t-1)(t-2)$. Therefore, $A$ satisfies
$$
(A-I)(A-2I) = 0.
$$
\a\aa
Using \x{interpolation polynomials}, we may decompose every polynomial $g(x)$ into the form
$$
g(x) = Q(x)(x-1)(x-2) + g(1)\frac{x-2}{1-2} + g(2) \frac{x-1}{2-1}.
$$
Use interpolations to construct \x{eigenspace projections}
$$
P_1 = \frac{A-2I}{1-2}=\m21,{-2}{-1}. ␣ 
P_2 = \frac{A-I}{2-1} = \m {-1}{-1},22.
$$
Plug in $A$ into the equation, we have a spectural decomposition, for any polynomial $g$ that
$$
g(A) = g(1)P_1 + g(2)P_2.
$$
\a\aa
How to calculate $e^As$? We actually use polynomials first
$$
\left(I+\frac An\right)^{ns} = 
\left(1+\frac 1n\right)^{ns}P_1
+ 
\left(1+\frac 2n\right)^{ns}P_2
$$
Now letting $s ⟶  ∞$, we obtain
$$
e^{As} = e^s \m21,{-2}{-1}. +e^{2s }\m {-1}{-1},22.
$$
Going back to the original equation, we have
$$
\m
{y_1(s)},
{y_2(s)}.
=
e^{As}
\m
{y_1(0)},
{y_2(0)}.
=
\m
{(2e^s-e^{2s})y_1(0) + (e^s-e^{2s})y_2(0)},
{(-2e^s+2e^{2s})y_1(0)+(-e^s+2e^{2s})y_2(0)}.
$$
We solved it!
\a{Geometric understanding via eigenvectors}


% Introduction to Eigenvectors:
We will now explore how eigenvectors and eigenvalues of a matrix can be used to solve the differential equation \( \mathbf{y}' = A\mathbf{y} \). But first, let's review what eigenvectors and eigenvalues are.

\a\aa
% Definition of Eigenvector and Eigenvalue:
An eigenvector of a matrix \( A \) is a non-zero vector \( \mathbf{v} \) that, when multiplied by \( A \), results in a scalar multiple of itself. This scalar is known as the eigenvalue. Mathematically, it is expressed as:
$$
A\mathbf{v} = \lambda\mathbf{v}
$$
where:
\begin{itemize}
\item \( \mathbf{v} \) is the eigenvector,
\item \( \lambda \) is the eigenvalue associated with \( \mathbf{v} \),
\item \( A \) is the matrix in question.
\end{itemize}
% Idea Behind Eigenvectors:
%The idea of eigenvectors is fundamental in understanding the behavior of linear transformations represented by matrices. 
Eigenvectors point in directions that are unaffected by the transformation, except for being scaled by their corresponding eigenvalues.

% Application to Differential Equations:
%In the context of differential equations, eigenvectors and eigenvalues provide a way to decouple the system and find solutions that grow (or decay) exponentially over time. Each eigenvector provides a direction in which the system evolves, and the corresponding eigenvalue determines the rate of this evolution.

% Conclusion:
%Understanding eigenvectors and eigenvalues is crucial in solving systems of linear differential equations, as they reveal the underlying structure and behavior of the system represented by the matrix \( A \).
\a\aa
Recall that we have used interpolations to construct \x{eigenspace projections}
$$
P_1 = \frac{A-2I}{1-2}=\m21,{-2}{-1}. ␣ 
P_2 = \frac{A-I}{2-1} = \m {-1}{-1},22.
$$
We have $P_1+P_2=I$, this gives a decomposition to any vector, in particular
$$
\m
{y_1(0)},
{y_2(0)}.
=
\m21,{-2}{-1}.\m
{y_1(0)},
{y_2(0)}.
+
\m {-1}{-1},22.
\m
{y_1(0)},
{y_2(0)}.
$$
Our idea is to name this two components as
$$
\textbf{w}(s)=\m
{w_1(s)},
{w_2(s)}.
=
\m21,{-2}{-1}.\m
{y_1(s)},
{y_2(s)}.
$$
and
$$
\textbf{u}(s)=\m
{u_1(s)},
{u_2(s)}.
=
\m {-1}{-1},22.\m
{y_1(s)},
{y_2(s)}.
$$
\a\aa
With this decomposition, 
$$
\mathbf y(s)=\mathbf w(s)+\mathbf u(s).
$$
and
$$
A\mathbf y(s)=A\mathbf w(s)+A\mathbf u(s) = \mathbf w(s)+2\mathbf u(s).
$$
Therefore, we are in fact solving two equations
$$ \mathbf w'(s) = A\mathbf w(s) = \mathbf w(s)  ⟹   \mathbf w(s) = e^s\mathbf w(0) $$
$$ \mathbf u'(s) = A\mathbf u(s) = 2\mathbf u(s)  ⟹   \mathbf u(s) = e^{2s}\mathbf u(0) $$

This illustrates the idea that
$$
\mathbf y(s) = \mathbf w(s)+\mathbf u(s) = e^s\mathbf w(0) + e^{2s}\mathbf u(0).
$$

\aaa

\aaa{Exponential of Differential Operator and Talor Expansion}
To try to understand exponential function better, we give an intuitive understanding of 
$$
\exp\left(\frac{\dd}{\dd x}\right)[f](x)=f(x+1)
$$
Since $\exp(x)=1+x+\frac{x^2}{2!}+\frac{x^3}{3!}+\cdots$. You can also write it as 
$$
f(x)+f'(x)+\frac{f''(x)}{2!}+\frac{f'''(x)}{3!}+\cdots = f(x+1)
$$
This is the same as \x{Talor} expansion. You know mathematically, but \x{why}?
\a\aa
Remember exponential functions are defined by
$$
\exp(x)=\lim_{n\rightarrow \infty}\left(1+\frac xn\right)^n
$$
Therefore, in nature, 
$$
\exp\left(\frac{\dd}{\dd x}\right)=\lim_{n\rightarrow \infty}\left(I+\frac1n\frac{\dd}{\dd x}\right)^n
$$
\vfill
What does the linear operator $I+\frac1n\frac{\dd}{\dd x}$ do for functions? 
\vfill
It maps $f(x)$ to $f(x)+\frac1n f'(x)$. We claim this is approximately $f\left(x+\frac1n\right)$. \x{To understand why, see next slides.}
\a\aa
To understand why 
$$
f(x)+\frac1n f'(x)\approx f\left(x+\frac1n\right)
$$
We firstly assume $f(x)$ is a linear function $f(x)=kx+b$, then
$$
f(x)+\frac1n f'(x)=kx+b+\frac kn=k(x+\frac1n)+b
$$
Geometrically, shifting a line $y=kx+b$ up by $\frac kn$ is shifting it left by $\frac1n$.

\[zzz]{
\draw[->] (-4,0)--(4,0) node [right]{$x$};
\draw[->] (0,-4)--(0,4) node {$y$};
\draw[thick,red] (-2,-4)--(2,4);
\draw[thick,red] (-3,-4)--(1,4);
\draw[ultra thick,blue,->] (0,0)--(0,2);
\draw[ultra thick,blue,->] (0,0)--(-1,0);
}
\a\aa
Since at the range $(x,x+\frac 1n)$, any function will become more straight. It can be approximate by line segments, therefore, the effect of $f(x)+\frac1n f'(x)$ when $n\rightarrow\infty$, is shifting the whole function to the left by $\frac 1n$. It changes $f(x)$ to $f\left(x+\frac1n\right)$.
\begin{tikzpicture}
      \draw[->] (-3,0) -- (4.2,0) node[right] {$x$};
      \draw[->] (0,-3) -- (0,4.2) node[above] {$y$};
      \draw[scale=0.5,domain=-3:3,smooth,variable=\x,blue,thick] plot ({\x},{\x*\x});
      \draw[scale=0.5,domain=-3:3,smooth,variable=\x,red] plot ({\x-0.2},{\x*\x});
    \end{tikzpicture}\a\aa
When $n$ is large. Apply this change $n$ times, 
$$
\underbrace{f(x)\rightarrow f(x+\frac1n)\rightarrow f(x+\frac 2n)\rightarrow \cdots\rightarrow f(x+1)}_{n\text{ times}}
$$
\vfill
the total will change $f(x)$ to $f(x+1)$. Therefore 
$$
\exp\left(\frac{\dd}{\dd x}\right)[f](x)=f(x+1)
$$
\x{This is a geometric understanding of Talor Expansion.}
\a\aa
Now use the binomial expansion we have that
$$
e^{\frac{\dd}{\dd x}}=\left(1+\frac1{∞}\frac{\dd}{\dd x}\right)^{∞} = 1+\frac{\dd}{\dd x}+\frac1{2!}\frac{\dd^2}{\dd x^2} + \frac1{3!}\frac{\dd^3}{\dd x^3} + \cdots
$$
Apply this operator to $f$, we know that 
$$
f(x+1) = f(x) + f'(x)+\frac{f''(x)}{2!} + \frac{f'''(x)}{3!}+\cdots
$$
This is Taylor expansion! Taylor expansion is nothing more than binomial expansion.

\exe Use the same method, deduce a formula for $f(x+s)$ for general $s$.
\aaa
