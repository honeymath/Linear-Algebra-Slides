

\aaa{Projection Matrices}
The importance of projection matrix comes from the following observation

\begin{defi}
An $n × n$ square matrix $P$ is a \x{projection matrix} if 
$$
P^2 = P.
$$
\end{defi}

\begin{prop}
If $AB = I_m$, then $BA$ is a projection matrix.
\end{prop}
$$
(BA)² =BABA = B(AB)A=BI_mA=BA
$$
\a\aa
It is very easy to verify if a vector is in $“Col“(P)$ or $“Null“(P)$ since 
\begin{prop}
Let $P$ be an $n × n$ projection matrix. For any vector $x ∈ ℝ^n$, we have
$$
y ∈ “Col“(P) ⟺   Py = y, ␣
x ∈ “Null“(P) ⟺   Px = 0.
$$
\end{prop}
If $y ∈ “Col“(P)$, then $y = Px$ for some $x$. Then $Py = PP x = Px = y$. Conversely, if $Py=y$, then $y ∈ “Col“(P)$ by definition.

Please finish the proof of $x ∈ “Null“(P) ⟺   Px = 0$ by yourself!
\a\aa
\exe The following matrix is a projection matrix
$$
P=\m000,{-1}10,{-1}01.
$$
Calculate $\underbrace{\m000,{-1}10,{-1}01.}_P\m0,2,0.$Is that true that $\m0,2,0. ∈ “Col“(P)$?
\vfill
(Hint: Since $P$ is a projection, $“Col“(P) = \{y | Py=y\}$ )

\a\aa
\begin{prop}
If $P$ is a $n × n$ projection matrix, then $I_n-P$ is also a projection matrix.
\end{prop}

(Homework, you only need to check $(I_n-P)² = I_n-P $)



\a\aa
\begin{prop}
If $P$ is a $n × n$ projection matrix, then 
$$
“Col“(P)=“Null“(I_n-P) ␣ 
“Col“(I_n-P)=“Null“(P)
$$
\end{prop}
(Homework)

\begin{rem}
Therefore $P$ and $I-P$ \x{interchanges} their columns space and null spaces! What a beautiful phenomenon!
\end{rem}
\aaa


\aaa{Decomposition of Projection}
Recall that just from $AB = I_m$, we create a projection matrix
$$
P = BA.
$$
\vfill
Are all projection matrices arises this way?


\a\aa
Giving a projection matrix $P$, recall that we might use cross-filling method
$$
\underbrace{\m000,{-1}10,{-1}01.}_P = \m0{\y0}0,{\y-1}{\y1}{\y0},0{\y0}0.+\m00{\y0},00{\y0},{\y-1}{\y0}{\y1}.
$$
This decompsoes
$$
\underbrace{\m000,{-1}10,{-1}01.}_P = \underbrace{\m00,10,01.}_B\underbrace{\m{-1}10,{-1}01.}_A
$$
Last time, we have learned that corss-filling process automatially give \li rows and columns, therefore, $A$ must have \x{\color{blue}right inverse} and $B$ must have \x{\color{red}left inverse}.
\vfill
But is that true that $AB=I_m$? if true, then all projection matrices arises from $P=BA$ with ivnerse pairs $AB=I_m$.
\a\aa
Cross-Filling decomposes $n × n$ rank $m$ projection matrix $P$ into

$$
P = BA
$$
where $B$ of size $n × m$ and $A$ of size $m × n$, and the cross-filling garantees that

\begin{itemize}
\item Columns of $B$ \li
\item Rows of $A$ \li.
\end{itemize}

So $B$ has {\color{red}left inverse}, $A$ has {\color{blue}right inverse}

\vfill
\x{Our question}: Is that true $AB=I_m$?

\a\aa
Yes! It is true, for simple reasons!!!

$$
P² = P ⟺  BABA = BA
$$

Since $A$ has right inverse
$$
BAB=B
$$
Since $B$ has left inverse
$$
AB = I_m.
$$

\a\aa

\begin{prop}
If $P$ is a projection matrix, then 
$$
“rank“(P) = “tr“(P).
$$
\end{prop}
Let $m=“rank“(P)$. Let $P=BA$ be obtained from cross-filling process. Then we must have $AB = I_m$.
\vfill
$“rank“(P)=m=“tr“(I_m)=“tr“(AB)=“tr“(BA)=“tr“(P)$.

\a\aa
\exe The following matrix is a projeciton matrix $P=P^2$, directly find out the rank of the matrix!
$$
\m00000,
{-1}1000,
{-1}0100,
{-1}0010,
{-1}0001.
$$
\exe The following matrix is a projection matrix, what is its rank?

$$
\m
{-4}{-9}54{-3},
49{-5}{-4}3,
{-8}{-18}{10}8{-6},
8{18}{-10}{-8}6,
{-8}{-18}{10}8{-6}.
$$
\aaa



\aaa{Cross-Filling for projection matrices}
This implies important observations.
Suppose one can decompose projection matrix by cross-filling

$$
P = c_1r_1^T + c_2r_2^T + \cdots +c_mr_m^T.
$$

The cross-filling process garantees that both $\m{c_1}\cdots{c_m}.$ and $\m{r_1}\cdots{r_m}.$ \li. Then we write

$$
P = \underbrace{\m{c_1}\cdots{c_m}.}_B\underbrace{\m{r_1^T},\vdots,{r_m^T}.}_A
$$
By what we learned before, using left and right cancelations, we must have $AB = I_m$!
\a\aa
But $I_m=AB$ gives this result
$$
\m 
10\cdots0,
01\cdots0,
\vdots\vdots\ddots\vdots,
00\cdots0.
=
\underbrace{\m{r_1^T},\vdots,{r_m^T}.}_A\underbrace{\m{c_1}\cdots{c_m}.}_B
=
\m
{r_1^Tc_1}{r_1^Tc_2}\cdots{r_1^Tc_n},
{r_2^Tc_1}{r_2^Tc_2}\cdots{r_2^Tc_n},
\vdots\vdots\ddots\vdots,
{r_2^Tc_1}{r_2^Tc_2}\cdots{r_2^Tc_n}.
$$
This means
$$
r_i^Tc_j = \begin{cases}
1& i=j\\
0& i≠j\\
\end{cases}
$$

\a\aa
\[thm]{Let $P=P^2$ be a $n × n$ projection martrix. Let $c_1,c_2,...,c_m$ and $r_1,r_2,...,r_m$ be two lists of linearly independent $n × 1$ matrices such that
$$
P = ∑_{i=1}^m c_ir_i^T
$$
Let $P_i:= c_ir_i^T$, so $P = P_1+P_2+\cdots+P_m$, and we have
\[itemize]{
\item $“rank“(P)=m$
\item $P_i$ are rank $1$ projection matrices in the sense that $P_i^2=P_i$.
\item $P_iP_j=0$ if $i ≠ j$, 
}
}
\x{Conclusion}: The cross-filling decomposes projection matrix into good rank-1 projections. We will address the importance of this in the future.

\aaa



\aaa{Rank 0 matrix}
Next, we finish our discussion of square matrices. Recall the question. If $A$ is a square matrix, does its right inverse equal to its left inverse?
\a\aa
\begin{prop}
A rank $0$ matrix is zero matrix.
\end{prop}

Because zero dimensional space can only have zero vector. A matrix with zero columns is a zero matrix.

\a{Full rank projection matrix}
\begin{prop}
%A $n × n$ projection matrix $P$ with $“tr“(P)=n$
An $n × n$ invertible perjection matrix $P$ must be $P=I_n$
\end{prop}

Because $I_n-P$ is a projection matrix of trace $0$, so is of rank $0$, therefore $I_n-P=0$.

\a\aa

\exe Please fill in the blanks so that the following matrix can be a projection matrix
$$
P = \m{0.5}\square,{0.5}\square.
$$


\a{Inverse for square matrices}

\begin{prop}
Supose $A$ is $n × n$ matrix with right inverse $B$ in the sense $AB=I_n$, then we have $BA=I_n$ as well.
\end{prop}

Because $BA$ must be a projection matrix with $“tr“(AB)=“tr“(BA)=n$, so $BA=I_n$.


\aaa



