

\def\1{\y1}
\def\2{\y2}
\def\3{\y3}
\def\4{\y4}
\def\5{\y5}
\def\6{\y6}
\def\7{\y7}
\def\8{\y8}
\def\9{\y9}
\def\0{\y0}
\def\-{\y-}
\aaa{Determinat and linearly independence}
For determinant, 

if one column is  a linear combination of others, then the determinant is 0 
$$
\m
|||,
{v_1}{v_2}{3v_1+2v_2} 
,|||
. = 0
$$



\a\aa
For determinant, we may add/subtract a column by linear combination of other columns, the determinant would not change.

For any $λ_1,...,λ_n$
$$
\det
\m
||\cdots|\cdots|
,{v_1}{v_2}\cdots{v_i}\cdots{v_n},
||\cdots|\cdots|.
$$
$$
=
\det
\m
||\cdots|\cdots|
,{v_1-λ_1v_i}{v_2-λ_2v_i}\cdots {v_i}\cdots{v_n-λ_nv_i},
||\cdots|\cdots|.
$$
\a\aa
Explaination: expand by columns

$$
\det\m||,{v_1+3v_2}{v_2},||.
=
\det\m||,{v_1}{v_2},||.
\underbrace{+\det\m||,{3v_2}{v_2},||.}_0
$$

\vfill

$$
\det\m|||,
{v_1+3v_2+5v_3}{v_2}{v_3},
|||.
=
\det\m|||,
{v_1}{v_2}{v_3},
|||.+
\det\m|||,
{3v_2+5v_3}{v_2}{v_3},
|||.
$$
\a\aa

Suppose cross-filling decomposes
$$
\m111,123,149. = \m{0.5}\1{1.5},\1{\h2}\3,2\46. +\m {0.5}0{-0.5},000,{-1}03.
$$


$$
\det\m111,123,149. = \det \m {0.5}\1{-0.5},0{\h2}0,{-1}\43.
=  \det \m {0.5}0{-0.5},0{\h2}0,{-1}03.
$$
\a\aa
Why 
$$
\det \m {0.5}\1{-0.5},0{\h2}0,{-1}\43.
=  \det \m {0.5}0{-0.5},0{\h2}0,{-1}03.?$$
\vfill
Think about it 
$$
\det \m {0.5}\1{-0.5},0{\h2}0,{-1}\43.
=  \det \m {0.5}0{-0.5},0{\h2}0,{-1}03. + \det\m {0.5}\1{-0.5},0{0}0,{-1}\43.
$$
\aaa



\aaa{determinant by cross-filling}
\[thm]{Let $A$ be an $n × n$ matrix and one have a one-step cross-filling decomposition
$$
A = A_1 + R
$$
where $A_1$ is a rank 1 matrix with cross-center $a_1$, $R$ has a zero cross. Let $\widetilde R$ be the $R$ relacing the cross center by $1$. Then
$$
\det A = a_1\det \widetilde R.
$$
}

\aaa

\aaa{Example}

\exe
Calculating determinant by cross-filling

$$
\det \underbrace{\m212,131,213.}_A
$$

$$
A=\m\262,{\h1}\3\1,\262.
+\m0{-\5}1,0\00,\0{\h-5}\1.
+\m\0\0{\h-1},00\0,00\0. 
$$



$$
\det A=1(-5)(-1)\m00\1,\100,0\10.
$$


\aaa



\aaa{Determinant of Switching Matrix}
The calculation of determinant by cross-filling gives us the question of determining the value of the following determinant
\[defi]{
A switching matrix is a matrix with each row and column a unique non-zero entry valued $1$}

$$\m\1000,000\1,00\10,0\100.$$
\a{zigzag method}
The formula for determinant of switching matrix can be summarized as follows
\[itemize]{
\item Find all horizontal and vertical segment inking $1$ and diagonal entrise(no matter what that is).
\vfill
\item Count the number $m$ of \x{connected loops }
\vfill
\item Determinant is $(-1)^{m+n}$.
}
\a\aa
Calculation example
$$
\xymatrix{
0\ar[rrrr]{}{}&&0&&1\ar[dddd]{}{}&&0\\\\
0&&0\ar[rrrr]{}{}&&0&&1\ar[dddd]{}{}\\\\
0&&1\ar[uu]{}{}&&0\ar[ll]{}{}&&0\\\\
1\ar[uuuuuu]{}{}&&0&&0&&\ar[llllll]{}{}0\\
}
$$
There are only one \x{loop} linking all one.
\vfill
$$
\det\m0010,0001,0100,1000. = (-1)^{4+1}=-1.
$$
\a\aa
To see why . This is because that each swithcing inside path will break a loop into two.

$$
\xymatrix{
0\ar[rrrr]{}{}&&0&&1\ar[dddd]{}{}&&0\\\\
1\ar[uu]{}{}&&0\ar[ll]{}{}&&0&&0\\\\
0&&1\ar[uu]{}{}&&0\ar[ll]{}{}&&0\\\\
&&&&&\ar[d]{}{}&\ar[l]{}{}\\
0&&0&&0&\ar[r]{}{}&1\ar[u]{}{}\\\\
}
$$
Note that in this picture, $1$ itself is a loop.





\aaa





\aaa{Determinant of Transpose}
\[thm]{
We have $“det“(A)=“det“(A^T)$
}
Because the cross filling is symmetric on rows and columns. 
$$
A = P_1+...+P_n
$$
is a cross-filling for $A$, then $A^T=P_1^T+...+P_n^T$ is cross-filling for $A^T$ with center value the same. 

Left $\det S= \det S^T$ for switching matrix.
$S$ with path denoted, the transpose is still a path. So number of path not changed.
\aaa




\aaa{Determinant of Block Triangular Matrices}
\[thm]{
We have
$$ “det“\m A0,CD. = “det“A“det“D.  $$
and similarly
$$ “det“\m AB,0D. = “det“A“det“D.  $$
}
\a\aa
Proof, this is because 

When performing cross filling for $A$, the matrix $B$ or $D$ has automatically be deleted, and the paths for the switching matrix has been constrained inside each diagonal block. Let us demonstrade this by examples.

\a\aa
For finding the determinant of the matrix
$$\m11897,21121,00111,00221,00231.$$
we notice the block makes it to upper triangular block matrix.
$$\m\1\1897,\2\1121,00\1\1\1,00\2\2\1,00\2\3\1.$$
\a\aa
while the cross filling for the first block has been performed, it automatically cleared non-diagonal blocks.
$$
\m\1{\h1}\8\9\7,1\1897,0\0000,0\0000,0\0000.
+\m\00000,{\h1}\0{-\7}{-\7}{-\6},\00111,\00221,\00231.
+\underbrace{\m00000,00000,00111,00221,00231.}_{“remainder block“}
$$
Then one perform the corss-filling for remainder block.
$$
\m
0000\0,
0000\0,
\0\0\1\1{\h1},
0011\1,
0011\1.
+\m
00\000,
00\000,
00\000,
\0\0{\h1}\1\0,
00\110.
+
\m
000\00,
000\00,
000\00,
000\00,
\0\0\0{\h1}\0.
$$
$$“product of cross center “=
\overbrace{
\t{\h1}.
× 
\t{\h1}.
}^{“first block“}
× 
\overbrace{
\t{\h1}.
× 
\t{\h1}.
× 
\t{\h1}.
}^{“second blcok“}
$$
\a\aa
Now we trying to figure out the sign. Using zigzag method, Explain why
$$
\det\m
\0{\h1}000,
{\h1}\0000,
00\00{\h1},
00{\h1}\00,
000{\h1}\0.
$$
\vfill
is 
\vfill
$$
\det \m
\0{\h1},{\h1}\0.
× 
\det
\m\00{\h1},
{\h1}\00,
0{\h1}\0.
$$
\a\aa
Inductively
\[thm]{ The determinant of block upper or lower triangular matrix is the product of determinant of blocks.
$$
“det“\m
{A_{11}}{A_{12}}\cdots{A_{1n}},
0{A_{22}}\cdots{A_{2n}},
\vdots\vdots\ddots\vdots,
00\cdots{A_{nn}}.
=
“det“(A_{11})
“det“(A_{22})
\cdots
“det“(A_{nn})
$$
}
\a\aa
The formula is also true for block lower triangular matrix and block diagonal matrix
$$
“det“\m
{A_{11}}0\cdots0,
{A_{12}}{A_{22}}\cdots0,
\vdots\vdots\ddots\vdots,
{A_{1n}}{A_{2n}}\cdots{A_{nn}}.
=
“det“(A_{11})
“det“(A_{22})
\cdots
“det“(A_{nn})
$$


$$
“det“\m
{A_{11}}0\cdots0,
{0}{A_{22}}\cdots0,
\vdots\vdots\ddots\vdots,
{0}{0}\cdots{A_{nn}}.
=
“det“(A_{11})
“det“(A_{22})
\cdots
“det“(A_{nn})
$$
\a\aa
In particular, the determinant of usual upper triangular, diagonal, and lower triangular matrices is the product of diagonal entries.

\aaa



\aaa{Determinant of product of matrices}
Look at the following graph, explain the matrix product


\[columns]{
\co4
\[zzz]{
\org
\grid{-5}{-5}55
\draw[thick,  ->] (0,0) -- (1,0);
\draw[thick,  ->] (0,0) -- (0,1);
\pgftransformcm{1}{2}{-1}{1}{\pgfpoint{0}{0}}
\grid[red!60!white!40]{-2}{-2}22
\draw[thick,red,  ->] (0,0) -- (1,0) node[right]{$\vec v_1$};
\draw[thick,red,  ->] (0,0) -- (0,1)node[left]{$\vec v_2$};
\draw[thick,purple,  ->] (0,0) -- (1,-1)node[left]{$\vec w_1$};
\draw[thick,purple,  ->] (0,0) -- (1,2)node[left]{$\vec w_2$};
%\draw[green,fill=green,opacity = 0.5] (0,0)--(1,2)--(2,1)--(1,-1)--(0,0);
	}
\co6
$$ \m{\vec v_1}{\vec v_2}. = \m{\vec e_1}{\vec e_2}.\m1{-1},21.  $$
$$ \m{\vec w_1}{\vec w_2}. = \m{\vec v_1}{\vec v_2}.\m11,{-1}2.$$
$$ \m{\vec w_1}{\vec w_2}. = \m{\vec e_1}{\vec e_2}.\m1{-1},21.\m11,{-1}2.  $$
}
\a\aa
How to calculate the deteminant without calculating matrix product?
\[columns]{
\co5
\[zzz]{
\org
\grid{-5}{-5}55
\draw[thick,  ->] (0,0) -- (1,0);
\draw[thick,  ->] (0,0) -- (0,1);
\pgftransformcm{1}{2}{-1}{1}{\pgfpoint{0}{0}}
\grid[red!60!white!40]{-2}{-2}22
\draw[thick,red,  ->] (0,0) -- (1,0) node[right]{$\vec v_1$};
\draw[thick,red,  ->] (0,0) -- (0,1)node[left]{$\vec v_2$};
\draw[green,fill=green,opacity = 0.5] (0,0)--(1,2)--(2,1)--(1,-1)--(0,0);
	}
\co5

$$
\[zzz]{
%\draw (0,0)--(0,1)--(1,1)--(1,0)--(0,0);
\pgftransformcm{1}{2}{-1}{1}{\pgfpoint{0}{0}}
\draw[fill=red,opacity=0.3] (0,0)--(0,1)--(1,1)--(1,0)--(0,0);
}
:\[zzz]{
\draw (0,0)--(0,1)--(1,1)--(1,0)--(0,0);
%\pgftransformcm{1}{2}{-1}{1}{\pgfpoint{0}{0}}
%\draw (0,0)--(0,1)--(1,1)--(1,0)--(0,0);
}
=\det\m1{-1},21.
$$

$$
\[zzz]{
\pgftransformcm{1}{2}{-1}{1}{\pgfpoint{0}{0}}
\draw[green,fill=green,opacity = 0.5] (0,0)--(1,2)--(2,1)--(1,-1)--(0,0);
	}
:
\[zzz]{
%\draw (0,0)--(0,1)--(1,1)--(1,0)--(0,0);
\pgftransformcm{1}{2}{-1}{1}{\pgfpoint{0}{0}}
\draw[fill=red,opacity=0.3] (0,0)--(0,1)--(1,1) node[above]{}--(1,0)--(0,0);
}
=\det\m1{1},{-1}2.
$$
}
\a\aa
\[thm]{
$$
“det“(AB)
=
“det“(A)
“det“(B)
$$
}
\vfill

We have a geometrical understanding of the theorem, but we do not yet have a mathematical proof.
\a\aa
$$
\m{B},{AB}. = \m{I},A.B
$$
Each column of $\m{B},{AB}.$ is a linear combination of columns of $\m{I},A.$ by coefficients listed in $B$
\a\aa
Here illustrates mathematical proof.
$$
“det“\m I{},A{AB}.
=
“det“\m I{-B},A0.
=
“det“\m I{-B+I},AA.
=
“det“\m B{-B+I},0A.
$$

\a\aa
\[prop]{A matrix $A$ is invertible if and only if $“det“(A)≠0$
}
If $A$ has inverse, then
$“det“(A)“det“(A^{-1}) = “det“(I_n)=1$,
so $“det“(A)≠0$

If 
 $“det“(A)≠0$, then we can construct $A^{-1}$ by $A^*/“det“(A)$

\a\aa
We have
$$ “det“(A^k) = “det“(A)^k $$
$$ “det“(λA) =λ^n “det“(A) $$
\aaa
