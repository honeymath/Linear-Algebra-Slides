

\def\1{\y1}
\def\2{\y2}
\def\3{\y3}
\def\4{\y4}
\def\5{\y5}
\def\6{\y6}
\def\7{\y7}
\def\8{\y8}
\def\9{\y9}
\def\0{\y0}
\def\-{\y-}

\aaa{Determinant}
\textbf{Warm up}: Suppose the area of each  small rectangle (for example, the pink colored one) is 1. How to calculate the {\color{cyan}blue area}? 

\[zz]{
	\wangge {-3}{-3}33
\draw[->](-4,0)--(4,0) node[right] {x};
\draw[->](0,-4)--(0,4) node[above] {y};
\draw[fill = pink,opacity=0.5] (0,0)--(1,0)--(1,1)--(0,1)--(0,0);
\draw[fill= cyan,opacity = 0.5] (0,0)--(-1,-2)--(-3,0)--(-2,2)--(0,0);
	}
\a\aa
The blue area is determined by two vectors, which describe the sides of the parallelogram. We put the function \x{determinant to describe the area of this part}
\[columns]{
	\co 5
\[zz]{
	\wangge {-3}{-3}33
\draw[->](-4,0)--(4,0) node[right] {x};
\draw[->](0,-4)--(0,4) node[above] {y};
\draw[fill = pink,opacity=0.5] (0,0)--(1,0)--(1,1)--(0,1)--(0,0);
\draw[fill= cyan,opacity = 0.5] (0,0)--(-1,-2)--(-3,0)--(-2,2)--(0,0);
\draw[->,ultra thick] (0,0)--(-1,-2) node [below]{$\m{-1},{-2}.$};
\draw[->,ultra thick] (0,0)--(-2,2) node [above]{$\m{-2},{2}.$};
	}
	\co 5
$$
\det\m{-2}{-1},2{-2}. = \text{\color{cyan}Blue area}
$$
}
\a{Positive area v.s. negative area}
What is the area of the rectangle enclosed by the vector $\vv_1$ and $\vv_2$?
\[columns]{\co 5
\[zz]{
	\wangge {-3}{-3}33
\draw[->](-4,0)--(4,0) node[right] {x};
\draw[->](0,-4)--(0,4) node[above] {y};
%\draw[fill = pink,opacity=0.5] (0,0)--(1,0)--(1,1)--(0,1)--(0,0);
\draw[fill= cyan,opacity = 0.5] (0,0)--(3,0)--(3,2)--(0,2)--(0,0);
\draw[->,ultra thick] (0,0)--(3,0) node [right]{$\vv_1$};
\draw[->,ultra thick] (0,0)--(0,2) node [above]{$\vv_2$};
	}

\co 5

Since the area is base times height, the area is $3\times 2 = 6$.
	}

\a\aa
However, if I put the height in another direction, it makes more sense to define the height to be a negative number
\[columns]{\co 5
\[zz]{
	\wangge {-3}{-3}33
\draw[->](-4,0)--(4,0) node[right] {x};
\draw[->](0,-4)--(0,4) node[above] {y};
%\draw[fill = pink,opacity=0.5] (0,0)--(1,0)--(1,1)--(0,1)--(0,0);
\draw[fill= cyan,opacity = 0.5] (0,0)--(3,0)--(3,-2)--(0,-2)--(0,0);
\draw[->,ultra thick] (0,0)--(3,0) node [right]{$\vv_1$};
\draw[->,ultra thick] (0,0)--(0,-2) node [above]{$\vv_2$};
	}

\co 5

Since the area is base times height, the area is $3\times (-2) = -6$.
	}


\a\aa
This signed area is called the \x{oriented area}. We call $\vv_1$ the base vector, and $\vv_2$ the height vector.
\vfill
If the height vector is on the counterclockwise side of the base vector, the area is positive. Otherwise, the area is negative. 
\a{Intended properties of determinant}
Before defining determinant, we should study what are the properties we are expecting for areas. Let's start with the easiest $2\times 2$ matrices.
\vfill
\textbf{Multilinearity}: We should have
$$
\det\m{\vv_1+\ww}{\vv_2}. = \det\m{\vv_1}{\vv_2}.+\det\m{\ww}{\vv_2}.
$$
and
$$
\det\m{\lambda\vv_1}{\vv_2}.=\lambda\det\m{\vv_1}{\vv_2}..
$$
\a{Explaination of multiplinearity}
The area of parallelogram expanded by $\vv_1+\ww$ and $\vv_2$ should have the same area as the sum of the one expanded by $\vv_1,\vv_2$ and $\ww,\vv_2$. The reason is explained in the following graph.

\dis{
\[z]{
	\ve12{\vec v_1}\vve1221{}\ve33{\vec w}\ve20{\vec v_2}
%\para[red,opacity=0.5]002012
%\para[yellow,opacity=0.5]122021
\para[green,opacity=0.3]002033
\draw (2,0)--(5,3);
\draw (2,0)--(3,2);
\draw (3,2)--(5,3);
}
}
{
\[z]{
\ve12{\vec v_1}\vve1221{}\ve33{\vec w}\ve20{\vec v_2}
\para[red,opacity=0.5]002012
\para[yellow,opacity=0.5]122021
%\para[green,opacity=0.3]002033
\draw (2,0)--(5,3);
}


}
\a\aa

\[z]{
\ve12{\vec v_1}\ve20{\vec v_2}\ve60{3\cdot\vec v_2}
\para[red,opacity=0.5]002012
\para[pink,opacity=0.3]006012
%\para[green,opacity=0.3]002033
}

\a\aa
The preceding properties yeilds another important property, the \textbf{\color{cyan}column swapping property}:
$$
\det\m{\vv_1}{\vv_2}. = -\det\m{\vv_2}{\vv_1}.
$$
This property can be deduced by the following steps
$$
0=\det\m{\vv_1+\vv_2}{\vv_1+\vv_2}.
$$
$$ 0=\det\m{\vv_1}{\vv_1}.  $$
$$ 0=\det\m{\vv_2}{\vv_2}.  $$
Then the expansion of $\det\m{\vv_1+\vv_2}{\vv_1+\vv_2}.$ gives
$$
\det\m{\vv_1}{\vv_1}. +\det\m{\vv_1}{\vv_2}.+\det\m{\vv_2}{\vv_1}.+ \det\m{\vv_2}{\vv_2}.
$$
%$$
%{
%\[split]{
%{\det\m{\vv_1}{\vv_2}.}&=
%{
%\underbrace{\det\m{-\vv_2}{-\vv_2}.}_{=0}
%-\underbrace{\det\m{\vv_1+\vv_2}{\vv_1+\vv_2}.}_{=0}
%+\underbrace{\det\m{\vv_1}{\vv_1}.}_{=0}
%+\det\m{\vv_1}{\vv_2}.} 
%\\\\
%&=
%{
%\det\m{-\vv_2}{-\vv_2}.
%-\det\m{\vv_1+\vv_2}{\vv_1+\vv_2}.
%+\det\m{\vv_1}{\vv_1+\vv_2}.} 
%\\\\
%&=
%{
%\det\m{-\vv_2}{-\vv_2}.
%+\det\m{-\vv_2}{\vv_1+\vv_2}.} 
%\\\\
%&=
%{
%\det\m{-\vv_2}{\vv_1}.}
%\\\\
%&=
%{
%-\det\m{\vv_2}{\vv_1}.}
%\\\\
%}
%}
%$$


\a{Normalization}
Don't forget that we are assuming the basic square box has area one. Therefore, we should have
$$
\det\m10,01.=1
$$
\[columns]{
	\co 5
\[zz]{
	\wangge {-3}{-3}33
\draw[->](-4,0)--(4,0) node[right] {x};
\draw[->](0,-4)--(0,4) node[above] {y};
\draw[fill = pink,opacity=0.5] (0,0)--(1,0)--(1,1)--(0,1)--(0,0);
\draw[->, ultra thick] (0,0)--(1,0);
\draw[->, ultra thick] (0,0)--(0,1);
	}
\co 5
In general, we will denote $\vec e_1=\m1,0.$ and $\vec e_2=\m0,1.$. We can denote $\det\m{\ee_1}{\ee_2}.=1.$
}
\aaa
%
%\aaa{Summary}
%These properties motivates us to give the following definition.
%\[defi]{A function 
%$$\map f{\mathrm{Mat}_{2\times 2}(F)}{F}$$
%is called \x{multiplinear} if 
%$$f(\lambda\ww+\mu\vv_1,\vv_2)=\lambda f(\ww,\vv_2)+\mu f(\vv_1,\vv_2),$$ 
%$$f(\vv_1,\lambda\ww+\mu\vv_2)=\lambda f(\vv_1,\ww)+\mu f(\vv_1,\vv_2)$$ for any $\lambda,\mu\in F$ and $\ww,\vv_1\in F^2$.
%Furthermore, it is called \x{alternating} if
%$$f(\vv_1,\vv_2)=-f(\vv_2,\vv_1)$$
%for any $\ww,\vv_1\in F^2$.
%}
%\a\aa
%In one word, an alternating multilinear function satisfies all required properties to describe parallelogram areas enclosed by two vectors.
%\a\aa
%
%\[lem]{If $\map f{\mathrm{Mat}_{2\times 2}(F)}{F}$ is multilinear, then $f(\vec 0,\vv)=f(\vv,\vec 0)=0$}
%\[proof]{Since $0\cdot\vec 0=\vec 0$, we have
%$$
%f(\vec 0,\vv)=f(0\cdot \vec 0,\vv)=0f(\vec 0,\vv) = 0.
%$$
%The proof for $f(\vv,\vec 0)=0$ is similar.}
%\a\aa
%
%\[lem]{If $\map f{\mathrm{Mat}_{2\times 2}(F)}{F}$ is alternating, then $f(\vv,\vv)=0$.}
%\[proof]{
%	We have $f(\vv,\vv) = -f(\vv,\vv)$, therefore $f(\vv,\vv)=0$.
%	}
%\a\aa
%\[cor]{\label{multi}If $\map f{\mathrm{Mat}_{2\times 2}(F)}{F}$ is multiplinear and alternating, then $f(\lambda\vv,\vv)=0$ for any $\lambda\in F$.}
%\[rem]{The geometric intuition for this proposition is clear: the parallelogram enclosed by colinear vectors must have area zero.}
%
%
%\a\aa
%\[defi]{The determiant function $\map \det{\mathrm{Mat}_{2\times 2}(F)}{F}$ is an \x{alternating} \x{multiplinear} function which is normalized in the sense that
%$$
%\det\m10,01. = 1.
%$$
%}
%
\aaa{Formula for determinant}

%Using the definition, it is not hard to deduce the formula of any multilinear alternating functions.
%\[lem]{If $\map f{\mathrm{Mat}_{2\times 2}(F)}{F}$ is multiplinear and alternating, then
%$$
%f\m ab,cd. = (ad - bc)f\m 10,01..
%$$
%}
%\a\aa
%\textbf{Proof}:Before entering to the proof, we claim that
%$$
%f \m00,cd. = 0
%$$
%by Corollary \ref{multi} since the second column is a scalar multiple of the first column.
%
%By multiplinearity, we can write
%$$
%f\m ab,cd. = f\m a0,cd.+f\m 0b,cd..
%$$
%Note that
%$$
%f\m a0,cd. = f\m a0,0d.+f\m 00,cd. = ad\cdot f\m10,01.+0
%$$
%and that
%$$
%f\m 0b,cd. = f\m 00,cd.+f\m 0b,c0. = -bc\cdot f\m10,01.+0.
%$$
%\a\aa
%Therefore 
%$$
%f\m ab,cd.=(ad-bc)\cdot f\m10,01.
%$$
%as desired.

\[prop]{We have
$$
\det\m ab,cd. = ad-bc.
$$}
\a\aa
This formula can be well-explained by the folloging graph. The red area of the parallelogram enclosed by the vector 
$$
\m a,c.,\m b,d.
$$
can be cut off from a rectangle with length $a+b$, height $c+d$ by the following way.

\[columns]{
\co5
\[z]{
\para[red,opacity=0.5]001221\draw[fill=yellow,opacity=0.3](0,0)--(2,1)--(2,0)--(0,0);
\draw[fill=yellow,opacity=0.3](1,2)--(1,3)--(3,3)--(1,2);
\draw[fill=green,opacity=0.3](0,0)--(1,2)--(0,2)--(0,0);
\draw[fill=green,opacity=0.3](2,1)--(3,1)--(3,3)--(2,1);
\draw[fill=cyan,opacity=0.3](2,0)--(2,1)--(3,1)--(3,0)--(2,0);
\draw[fill=cyan,opacity=0.3](1,2)--(1,3)--(0,3)--(0,2)--(1,2);
\draw[|-|](0,-0.2)-- node[below]{a} ++(2,0);
\draw[|-|](2,-0.2)-- node[below]{b} ++(1,0);
\draw[|-|](3.2,0)-- node[right]{c} ++(0,1);
\draw[|-|](3.2,1)-- node[right]{d} ++(0,2);
\draw[|-|](3,3.2)-- node[above]{a} ++(-2,0);
\draw[|-|](1,3.2)-- node[above]{b} ++(-1,0);
\draw[|-|](-0.2,3)-- node[left]{c} ++(0,-1);
\draw[|-|](-0.2,2)-- node[left]{d} ++(0,-2);
}
\co5
$$
(a+b)(c+d)-bc-bd-ac = ?
$$
}
\a\aa
\exe Using the formula of the determinant, calculate the following
\[enumerate]{
\item $\det\m 12,01.$
\item $\det\m 21,32.$
\item $\det\m 10,80.$
	}
\aaa

\aaa{Definition of determinant}

\[summ]{
\[itemize]{
\item Axiom 1:  
$$\det(\vec v_1,...,\vec v_i+\vec z_i,...,\vec v_n)=\det(\vec v_1,...,\vec v_i,...,\vec v_n)+\det(\vec v_1,...,\vec z_i,...,\vec v_n)$$

\item Axiom 2: For any $\lambda\in\mathbb R$, $\det(\vec v_1,\vec v_2,\cdots,\lambda\cdot\vec v_i,\cdots,\vec v_n)=\lambda\cdot \det(\vec v_1,\vec v_2,\cdots,\vec v_i,\cdots,\vec v_n)$ .
\item Axiom 3: $\det(\cdots,\vec v,\cdots,\vec v,\cdots)=0$ for any $\vec v\in V$.
\item Axiom 4: $\det I_n=\det(\vec e_1,\vec e_2,\cdots,\vec e_n)=1$.
}
}
Here  $\vec e_i$ are the $i$'th column in identity matrix $I_n$. 
%For example
%$$
%\vec e_1:=\m1,0,0.\quad
%\vec e_2:=\m0,1,0.\quad
%\vec e_3:=\m0,0,1.\quad\qquad
%I_3:=\m100,010,001..
%$$

\x{These axiomizes the notion of volumn in that space.}

\x{We require the matrix for determinant calculation to be a square matrix!}

\a{Common Mistakes}
Axiom 1: 
$$\det(\vec v_1,...,\vec v_i+\vec z_i,...,\vec v_n)=\det(\vec v_1,...,\vec v_i,...,\vec v_n)+\det(\vec v_1,...,\vec z_i,...,\vec v_n)$$
It states only for one column, other columns has to be fixed.

What is wrong with the following calculation???

\[exa]{
\x{\color{red}WRONG!}
$$
\det \m{1+2}{1+3},
{2+1}{3+2}.
=
\det
\m11,23.+\det\m23,12.
$$
}
\a\aa
The determinant is pretty like calculation of factors
In general, we have
$$
(x+y)(a+b)(c+d) = x(a+b)(c+d) + y (a+b)(c+d)
$$
but you can not do
$$
\x{\color{red}WRONG!} (\x x+y)(\x a+b)(c+d) = \x x\x a(c+d) + yb(c+d)
$$
Instead, there are intersection terms
$$
(x+y)(a+b)(c+d) = 
xa(c+d)+
xb(c+d)+
ya(c+d)+
yb(c+d)
$$
\a\aa
For determinant, similar situation happens, if you wanna expand multiple columns, you have to take care of intersection terms
$$
\det\m{v_1+w_1}{v_2+w_2}{v_3}.
$$$$=
\det\m{v_1}{w_1}{v_3}.+
\det\m{v_2}{w_1}{v_3}.+
\det\m{v_1}{w_2}{v_3}.+
\det\m{v_2}{w_2}{v_3}.
$$
\a\aa
Axiom 2: For any $\lambda\in\mathbb R$, $\det(\vec v_1,\vec v_2,\cdots,\lambda\cdot\vec v_i,\cdots,\vec v_n)=\lambda\cdot \det(\vec v_1,\vec v_2,\cdots,\vec v_i,\cdots,\vec v_n)$ .
\vfill
What is wrong for the following?
\[exa]{\x{\color{red}WRONG}
$$
\det (2A) = 2\det A?
$$
}
\a\aa
If fact, if $A=\m{v_1}{v_2}{v_3}.$, then $2A=\m{2v_1}{2v_2}{2v_3}.$
We have
$$
\det 2A = \det\m{2v_1}{2v_2}{2v_3}. = 2 \det\m{v_1}{2v_2}{2v_3}. 
$$$$= 4 \det\m{v_1}{v_2}{2v_3}. = 8\det\m{v_1}{v_2}{v_3}. = 2^3\det A.
$$
\aaa

\aaa{Algebraic cofactor}

\begin{equation}\label{general}
A:=\m{\vec v_1}{\vec v_2}\cdots{\vec v_n}.=\m
{a_{11}}{a_{12}}\cdots{a_{1n}},
{a_{21}}{a_{22}}\cdots{a_{2n}},
\vdots\vdots\ddots\vdots,
{a_{n1}}{a_{n2}}\cdots{a_{nn}}..
\end{equation}
Define \x{algebraic cofactors} by the \x{scalar}
$$
A_{ij}=\det\m{\vec v_1}{\vec v_2}\cdots{(\text{replace }\vec v_j\text{ by }\vec e_i)}\cdots{\vec v_n}..
$$ 
A quickway to remember $A_{ij}$ is to replace the element at ith row jth column by $1$ and put 0 to else where in its columns.
\a\aa
For example, 
\begin{equation}\label{special}
\text{ put }A=\m123,456,789.\implies A_{32}=\det\m103,406,7\19.
\end{equation}

 For this $A$ in \eqref{special}, please write down $A_{21}$, $A_{22}$ and $A_{23}$ as well (you don't need to calculate determinant for the exact number).

\a\aa

For 
$$
A=\m123,456,789.
$$
\vfill
$$
A_{12}=\det\m1\13,406,709. ␣ 
A_{22}=\det\m103,4\16,709. ␣ 
A_{32}=\det\m103,406,7\19.
$$

\a\aa

First important observation: Algebraic cofactors can be used for calculating the determinant with replaced columns.

$$
A[“replace 2nd col by another vector“]
=
\m1 
{\h x_1}3,4
{\h x_2}5,6
{\h x_3}9.
$$
Then
$$
\det \m1 
{\h x_1}3,4
{\h x_2}5,6
{\h x_3}9.
=
\x{\color{red}x_1}\underbrace{\det\m1\13,406,709.}_{A_{12}}
+
\x{\color{red}x_2}\underbrace{\det\m103,4\16,709.}_{A_{22}}
+
\x{\color{red}x_3}\underbrace{\det\m103,406,7\19.}_{A_{32}}
$$

This is called \x{Laplacian Expansion}.

\a\aa
Question: In the laplacian Expansion formula, which axiom did we used? and how did we use that?

$$
\det \m1 
{\h x_1}3,4
{\h x_2}5,6
{\h x_3}9.
=
\x{\color{red}x_1}\underbrace{\det\m1\13,406,709.}_{A_{12}}
+
\x{\color{red}x_2}\underbrace{\det\m103,4\16,709.}_{A_{22}}
+
\x{\color{red}x_3}\underbrace{\det\m103,406,7\19.}_{A_{32}}
$$

\[itemize]{
\item Axiom 1:  
$$\det(\vec v_1,...,\vec v_i+\vec z_i,...,\vec v_n)=\det(\vec v_1,...,\vec v_i,...,\vec v_n)+\det(\vec v_1,...,\vec z_i,...,\vec v_n)$$

\item Axiom 2: For any $\lambda\in\mathbb R$, $\det(\vec v_1,\vec v_2,\cdots,\lambda\cdot\vec v_i,\cdots,\vec v_n)=\lambda\cdot \det(\vec v_1,\vec v_2,\cdots,\vec v_i,\cdots,\vec v_n)$ .
\item Axiom 3: $$\det(\cdots,\vec v,\cdots,\vec v,\cdots)=0 ␣ “   for any “␣   \vec v\in V.$$
}

\a\aa
\exe Look at the formula, think about the question. Originally 
$$
A = \m1\23,4\56,7\89.
$$
$$
\det \m1 
{\h x_1}3,4
{\h x_2}5,7
{\h x_3}9.
=
\x{\color{red}x_1}\underbrace{\det\m1\13,406,709.}_{A_{12}}
+
\x{\color{red}x_2}\underbrace{\det\m103,4\16,709.}_{A_{22}}
+
\x{\color{red}x_3}\underbrace{\det\m103,406,7\19.}_{A_{32}}
$$
which choicse of $x_1$, $x_2$, $x_3$ could make the following equation true?
$$
\det A = \x{\color{red}x_1}A_{12} + \x{\color{red}x_2}A_{22} + \x{\color{red}x_3}A_{32}
$$

\a\aa
\exe Any repeated column will result $0$ for the determinant, therefore we have
$$
0 = \m1\13,4\46,7\79.
␣ 
0 = \m1\33,4\66,7\99.
$$
Recall the formula
$$
\det \m1 
{\h x_1}3,4
{\h x_2}5,7
{\h x_3}9.
=
\x{\color{red}x_1}\underbrace{\det\m1\13,406,709.}_{A_{12}}
+
\x{\color{red}x_2}\underbrace{\det\m103,4\16,709.}_{A_{22}}
+
\x{\color{red}x_3}\underbrace{\det\m103,406,7\19.}_{A_{32}}
$$
Please come up with more $x_1,x_2,x_3$ to keep the following equation true

$$
0 = \x{\color{red}x_1}A_{12} + \x{\color{red}x_2}A_{22} + \x{\color{red}x_3}A_{32}
$$

\a\aa
Clear, we may write 

$$
\det \m1 
{\h x_1}3,4
{\h x_2}5,7
{\h x_3}9.
=
\x{\color{red}x_1}A_{12} + \x{\color{red}x_2}A_{22} + \x{\color{red}x_3}A_{32}
=
\m{A_{12}}{A_{22}}{A_{32}}. \m
{\h x_1},
{\h x_2},
{\h x_3}.
$$
Would you please filling scalars in the following slot?
$$
\m\square\square\square. = \m{A_{12}}{A_{22}}{A_{32}}.\m1\23,4\56,7\89.
$$
\a\aa

$$
\m0{“det“(A)}0. = \m{A_{12}}{A_{22}}{A_{32}}.\m1\23,4\56,7\89.
$$
\a\aa

Now let us do first and third columns

$$
A_{11} = \m \123,056,089. ␣ 
A_{21} = \m 023,\156,089. ␣ 
A_{31} = \m 023,056,\189. ␣ 
$$

$$
\det \m 
{\h x_1}23,
{\h x_2}56,
{\h x_3}89.
=
\m{A_{11}}{A_{21}}{A_{31}}. \m
{\h x_1},
{\h x_2},
{\h x_3}.
$$


$$
\m\square\square\square. = \m{A_{11}}{A_{21}}{A_{31}}.\m\123,\456,\789.
$$

\a\aa

Let's do last column


$$
A_{13}=\m 12\1, 450, 780. ␣ 
A_{23}=\m 120, 45\1, 780. ␣ 
A_{33}=\m 120, 450, 78\1. ␣ 
$$

$$
\det \m 12
{\h x_1},45
{\h x_2},78
{\h x_3}.
=
\m{A_{13}}{A_{23}}{A_{33}}. \m
{\h x_1},
{\h x_2},
{\h x_3}.
$$

$$
\m\square\square\square. = \m{A_{13}}{A_{23}}{A_{33}}.\m12\3,45\6,78\9.
$$

\a\aa
Put all these together, what is your discovery?
$$
\m
{A_{11}}{A_{21}}{A_{31}},
{A_{12}}{A_{22}}{A_{32}},
{A_{13}}{A_{23}}{A_{33}}.
\m123,456,789. = 
\m
\square\square\square,
\square\square\square,
\square\square\square.
$$

\[defi]{Let $A$ be $ n × n$ matrix, we call the matrix
$$
A^* := 
\m
{A_{11}}{A_{21}}\cdots{A_{n1}},
{A_{12}}{A_{22}}\cdots{A_{n2}},
\vdots\vdots\ddots\vdots,
{A_{1n}}{A_{2n}}\cdots{A_{nn}}.
$$
The \x{adjugate} of matrix $A$. We have 
$$
AA^* = (\det A)\cdot I_n.
$$
}
\a\aa
Please be very careful on how position of the cofactors put into adjugate

$$
\m
{A_{11}}{\y A_{21}}{A_{31}},
{A_{12}}{A_{22}}{A_{32}},
{A_{13}}{A_{23}}{A_{33}}. ␣ 
A_{21}=
\m023,\156,089.
$$

Note:

$$
A^*
=
\m
{A_{11}}{\y A_{21}}{A_{31}},
{A_{12}}{A_{22}}{A_{32}},
{A_{13}}{A_{23}}{A_{33}}.
=
\m
{A_{11}}{ A_{12}}{A_{13}},
{\y A_{21}}{A_{22}}{A_{23}},
{A_{31}}{A_{32}}{A_{33}}.^T
$$

\a{Inverse matrix formula}
\[thm]{
If $\det A ≠ 0$, we have a formula for its inverse, given by
$$
A^{-1}=\frac{A^*}{\det A}.
$$
}

\a\aa
If $A$ is $2 × 2$ matrix, let's calculate the adjugate matrix of $A$

$$
A = \m ab,cd.
$$

$$
A_{11} = \m \1b,0d.␣ 
A_{21} = \m 0b,\1d.␣ 
$$

$$
A_{12} = \m a\1,c0.␣ 
A_{22} = \m a0,c\1.␣ 
$$

Therefore

$$
A^* = \m d{-b},{-c}a. ⟹   A^{-1}=\frac1{ad-bc}\m d{-b},{-c}a.
$$


\a{Cramer's rule}

Now that we have inverse matrix formula
$$
A^{-1} = \frac1{\det A}A^*.
$$

Suppose we wanna solve an equation
$$
\underbrace{A}_{n × n “ matrix “}
\m{x_1},{x_2},\vdots,{x_n}.
=
\m{b_1},{b_2},\vdots,{b_n}.
$$
where we are in lucky situations that $A$ invertible, then the solution is given by
$$
\m{x_1},{x_2},\vdots,{x_n}.
= A^{-1}\m{b_1},{b_2},\vdots,{b_n}.
$$
\a\aa
Therefore, we may write
$$
\m{x_1},{x_2},\vdots,{x_n}.
=
\frac{A^*}{“det“(A)}\m{b_1},{b_2},\vdots,{b_n}.
$$
So
$$
\m{x_1},{x_2},\vdots,{x_n}.
=
\frac1{“det“(A)}{\m
{A_{11}}{A_{21}}\cdots{A_{n1}},
{A_{12}}{A_{22}}\cdots{A_{n2}},
\vdots\vdots\ddots\vdots,
{A_{1n}}{A_{2n}}\cdots{A_{nn}}.
}\m{b_1},{b_2},\vdots,{b_n}.
$$
\a\aa
This gives a formula
$$
x_i = \frac1{“det“(A)}\m{A_{1i}}{A_{2i}}\cdots{A_{ni}}.\m{b_1},{b_2},\vdots,{b_n}.
$$
However, we remembered that the expression 
$$
\m{A_{1i}}{A_{2i}}\cdots{A_{ni}}.\m{b_1},{b_2},\vdots,{b_n}.
$$
is the determinant of replacing $\m{b_1},{b_2},\vdots,{b_n}.$ to $i$'th column of $A$.

\a{Cramer's rule}

\[thm]{Suppose 
$$
\underbrace{A}_{n × n “ matrix “}
\m{x_1},{x_2},\vdots,{x_n}.
=
\underbrace{\m{b_1},{b_2},\vdots,{b_n}.}_{\vec b}
$$
is a system of linear equations with $A$ invertible so that this equation has unique solution. Then $x_i$ is given by the formula
$$
x_i=\frac{\det A_i}{\det A}
$$
where $A_i$ is the matrix by replacing ith column of $A$ by constant $\vec b$
}

\a\aa
\exe Using Cramer's rule, write a formula for the component of the following solution in terms of determinant

$$
\m182,392,112. \m x,y,z. = \m1,2,1.
$$

$$
x=\frac{\det\m
\square\square\square,
\square\square\square,
\square\square\square.
}{
\det\m
\square\square\square,
\square\square\square,
\square\square\square.
}␣ 
y=\frac{\det\m
\square\square\square,
\square\square\square,
\square\square\square.
}{
\det\m
\square\square\square,
\square\square\square,
\square\square\square.
}␣ 
z=\frac{\det\m
\square\square\square,
\square\square\square,
\square\square\square.
}{
\det\m
\square\square\square,
\square\square\square,
\square\square\square.
}␣ 
$$




\aaa
