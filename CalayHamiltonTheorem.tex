
\aaa{Annihilating polynomial of matrices}
Let $A$ be an $n × n$ matrix, it is an element in the set
$$
ℝ^{n × n}:=\{M: ␣ M“ is an “n × n“ matrix “ \}
$$
which is $n × n$ dimensiona. Therefore, the following $n^2+1$ elements
$$
I=A^0, A=A^1, A^2, A^3, ..., A^{n^2}
$$
must be \x{linearly dependent}, and therefore gives some non-zero coefficients 
$$
c_0+c_1A+c_2A^2+...+c^{n^2}A^{n^2}=0.
$$
This implies that there exists a polynomial
$$
f(t)
=c_0+c_1t+c_2t^2+...+c^{n^2}t^{n^2}=0.
$$
with $f(A)=0$.
\a\aa
\[defi]{A polynomial $f(t)$ with $f(A)=0$ is called an \x{annilating polynomial} of an n × n matrix $A$.
}
\a\aa
\exe Try to find an annihilating polynomial of the matrxi
$$
A = \m 12,34.
$$

We calculate
$$
A^0 = \m 10,01. ␣ A^1 = \m 12,34. ␣ A^2 = \m 7{10},{15}{22}.
$$

We saw that $A^2-5A-2I=0$. Is there any other possible method to find the polynomial $x^2-5x-2$ other than solving equations?

\aaa


\aaa{Determinant and polynomial}

\[defi]{For any $n × n$ matrix $A$, the formula
$$
\det(tI-A)
$$
gives a polynomial of $t$. Call this \x{the characteristic polynomial} of $A$.
}

\[thm]{(Calay Hamilton Theorem)For any n × n matrix $A$, the charactiersitc polynomial $\det(tI-A)$ is an annilating polynomial of it.
}
{\color{red}\x{False proof}␣ $\det(tI-A)|_{t=A}=\det(AI-A)=\det0=0$}
\vfill
Instead, we should expand $\det(tI-A)$ first before plug in the matrix $A$.

\a\aa

Let $t$ be a variable.
\vfill
Why $“det“(tI-A)$ is a polynomial?

\a\aa

$$
\det\m{t-1}{3}4,
{1}{t-2}3,
11{t-1}.
$$
have at least 3 position with $t$ variables.

Expand it by a column, then for each determinant there have at most 2 position with $t$ variables. 
$$
(t-1)
“det“
{
\tiny\m
\134,
0{t-2}3,
01{t-1}.
}
+1
“det“
{\tiny\m
034,
\1{t-2}3,
01{t-1}.
}+1
“det“
{\tiny\m
034,
0{t-2}3,
\11{t-1}.
}
$$

Expanding a column containing variable $t$ reduces the number of variables inside determinant.

%Until we expanded out all column containing $t$, we will finally get a polynomial.
\a\aa
\exe Find the characteristic polynomial of 
$$
A = \m 12,34.
$$

$$
\det(tI-A)=“det“\m{t-1}{-2},{-3}{t-4}.=(t-1)(t-4)-(-2)(-3)=t^2-5t-2.$$
\a{Polynomial division}
Then why do we have

\[thm]{(Calay Hamilton Theorem)For any n × n matrix $A$, the charactiersitc polynomial $\det(tI-A)$ is an annilating polynomial of it.
}
{\color{red}\x{False proof}␣ $\det(tI-A)|_{t=A}=\det(AI-A)=\det0=0$ (You are only allowed to plug in numbers when writting this)}


\[rem]{
In scalar coeffient polynomial, you are allowed to plug in matrics. In matrix coefficient polynomial, you can only plug in scalars! \x{\color{red}Never plug in matrix to matrix coefficient polynomials!}
}
\a{About the False proof}
For example, 
$$
\det(tI_2)=\det \m t0,0t. = t^2
$$
\vfill
How do you plug in a matrix $t = B=\m12,34.$? 

Is that $\det(BI_2)$??

Is that
$$
\det \m B0,0B. = \det \m 1200,3400,0012,0034.?
$$
The above is closer, but it is not $B^2$, it is equal to $\det B^2$
\vfill
You can only first calculate out the polynomial before plug that in.
\a{Polynomial division}

Example. When simplifiying fractions, 
$$
t^3+2\over t-1
$$

$$
\begin{array}{rllll}
&&{\color{red} t^2}&+{\color{red}t}&+{\color{red}1}\\
    \cline{2-5}
    t-1 )& t^3 &  &  & +2 \\
    & t^3 & -t^2 &  & \\
    \cline{2-4}
    &  & t^2 &  & \\
    &   & t^2 & -t & \\
    \cline{3-5}
    &   &  & t & +2 \\
    &   &     & t & -1 \\
    \cline{4-5}
    &   &     &  & {\color{blue}3} \\
\end{array}
$$

$$
\frac{t^3+2}{t-1} = {\color{red}t^2 + t + 1} + { {\color{blue}3}\over t-1}
$$

\x{{\color{blue} 3} is the value of pluging $t=1$ to $t^3+2$}
\a\aa

$$
\frac{t^3+2}{t-1} = {\color{red}t^2 + t + 1} + { {\color{blue}3}\over t-1}
$$
⟺  
$$
t^3+2  ␣ = \underbrace{(t-1)({\color{red}t^2 + t + 1})}_{“This part vanishes when plugin“ t=1} +  ␣ {\color{blue}3}
$$
\a\aa
Instead of pluging matrix to determinant. We first calculate out determinant, and then apply polynomial division.

For example
$$
\det(tI-A)=\det\m{t-1}{-2},{-3}{t-4}.=t^2-5t-2
$$

$$
\begin{array}{rllll}
&&{\color{red} t}&+{\color{red}(A-5)}&\\
    \cline{2-5}
    t-A )& t^2 &-5t  &-2  & \\
    & t^2 & -At &  & \\
    \cline{2-4}
    &  & (A-5)t &-2  & \\
    &   & (A-5)t & A^2-5A & \\
    \cline{3-5}
      \cline{4-5}
    &   &     &   {\color{blue}A^2-5A+2} &\\
\end{array}
$$
The {\color{blue}remainder} is exactly we plug in $t=A$ to the polynomial.
$$
{t^2 - 5t -2 \over t-A }= {\color{red}t+(A-5)}+{{\color{blue}A^2-5A+2}\over t-A}
$$


\a\aa
We already have
$$
{t^2 - 5t -2 \over t-A }= {\color{red}t+(A-5)}+{{\color{blue}A^2-5A+2}\over t-A}
$$
\vfill
Is that possible to have another decomposition??
$$
{t^2 - 5t -2 \over t-A }
=
Bt+C+{D\over t-A}
$$
\a\aa
No, the polynomial division is uniquely determined, say, once
$$
{\color{red}t+(A-5)}+{{\color{blue}A^2-5A+2}\over t-A} = Et^2+Bt+C+{D\over t-A}
$$

Then $E=0$ , $B = I_2$, $C=(A-5)$ and $D=A^2-5A+2$.

$$
{t^2 - 5t -2 \over t-A }= {\color{red}t+(A-5)}+{{\color{blue}A^2-5A+2}\over t-A}
$$

\a\aa

\[thm]{
Let $C_{-1},C_0,...,C_n$ and $D_{-1},D_0,...,D_n$ be matrices that satisfying the following expression
$$
C_nt^n+C_{n-1}t^{n-1}+...C_0+C_{-1}(t-A)^{-1}
$$
$$
=
D_nt^n+D_{n-1}t^{n-1}+...D_0+D_{-1}(t-A)^{-1}
$$
Then $C_i=D_i$.
}

\textbf{Proof}: Dividing both side by $t^n$, we have
$$
C_n + \frac{C_{n-1}}t + ... +\frac{C_0}{t^n}+\frac{C_{-1}(I-\frac At)^{-1}}{t^{n+1}}
$$
$$
=
D_n + \frac{D_{n-1}}t + ... +\frac{D_0}{t^n}+\frac{D_{-1}(I-\frac At)^{-1}}{t^{n+1}}
$$
Comparing this equality at limit $t ⟶  ∞$, we have $C_n=D_n$.


%\a\aa
%In the above process, the quotient is uniquely determined, if 
%$$
%{t^2-5t-2\over t-A }=  Et^2+B t + C + {D \over t-A}
%$$
%then we must have
%$$
%E=0 ␣ B = I ␣ C = A-5I ␣ D = A^2-5A+2.
%$$
\a\aa
Going back to the proof of Calay Hamilton theroem, letting
$$
f(t):=\det(tI-A).
$$
on one hand, we may decompose the polynomial generally

$$
\frac{\det(tI-A)}{tI-A} =\frac{f(t)}{tI-A} = B_kt^k+...+B_0+\frac{f(A)}{tI-A}
$$
\a\aa

However, we have the adjugate formula.
$$
(tI-A)^*(tI-A) = \det(tI-A)I,
$$
which gives us another decomposition
$$
\frac{\det(tI-A)}{tI-A} = (tI-A)^*
$$
We claim that $(tI-A)^*$ MUST BE a polynomial!
$$
(tI-A)^* = C_mt^m+...+C_0 ␣ “ for some “C_i.
$$
Once we done this, then by comparing coefficients, 
$$
C_mt^m+...+C_0 ␣ ␣ 
$$
$$
=B_kt^k+...+B_0+\frac{f(A)}{tI-A}
$$

we obtain $f(A)=0$, the Calay--Hamilton theorem.

\a\aa

So why $(tI-A)$ is a polynomial? 

$$
“Illustrate idea with an example:“␣  A=\m
101,010,101.
$$
Think about how you find $(tI-A)^*$
$$
\underbrace{\m
{t-1}0{-1},
0{t-1}0,
{-1}0{t-1}.^*
}_{(tI-A)^*}
=
\m
{(t-1)^2}0{t-1},
0{t^2-2t}0,
{t-1}0{(t-1)^2}.
$$
\a\aa
Therefore, we may always write $(tI-A)^*$ as matrix polynomial of $t$.

$$
\m
{(t-1)^2}0{t-1},
0{t^2-2t}0,
{t-1}0{(t-1)^2}.
=
\m
{t^2-2t+1}0{t-1},
0{t^2-2t}0,
{t-1}0{t^2-2t+1}.
$$
$$
=\m100,010,001.t^2+\m{-2}01,0{-2}0,10{-2}.t +\m10{-1},000,{-1}01.
$$

\aaa



\aaa{Multiplinear expansion of determinant}
Having proved that the characteristic polynomial $f(t)=\det(tI-A)$ is annihilating polynomial $f(A)=0$, we want methods for calculating $\det(tI-A)$.
\a\aa
Determinant can be expanded once on one column
$$
\det \m{v+w}{u}.=\det \m vu. +\det\m wu.
$$
\a\aa
If wanna expand in two columns, we expand one by one
$$
\det\m{v+w}{u+k}.
$$
$$
=
\det \m v{u+k}.
+
\det \m w{u+k}.
$$
$$
=
\det\m vu.+\det\m vk.+\det \m wu.+\det\m wk.
$$
That is, to expand determinant from multiple columns, we add up all possible combinations of choosing a summand inside each factor.
\a\aa
This applies to the calculation of charcteristic polynomial.

Let $e_1,...,e_n$ be natural basis and let $A = \m{v_1}{...}{v_n}.$

$$
“det“(tI-A) = “det“\m{te_1-v_1}{te_2-v_2}{...}{te_n-v_n}.
$$
\a\aa
Example of expansion $\det(tI_3-A)$ for $3 × 3$ matrix.
$$
\det\m{te_1-v_1}{te_2-v_2}{te_3-v_3}.
$$$$
=
$$$$
\det\m{te_1}{te_2}{te_3}.
$$
$$
+
\det\m{-v_1}{te_2}{te_3}.
+
\det\m{te_1}{-v_2}{te_3}.
+
\det\m{te_1}{te_2}{-v_3}.
$$$$
\det\m{-v_1}{te_2}{e_3}.
+
\det\m{te_1}{-v_2}{e_3}.
+
\det\m{-v_1}{-v_2}{te_3}.
$$
$$
+
\det\m{-v_1}{-v_2}{-v_3}.
$$
\a\aa
We may factor out the sign and $t$

$$
\det\m{te_1-v_1}{te_2-v_2}{te_3-v_3}.
$$$$
=
$$$$
\det\m{te_1}{te_2}{te_3}.{\color{blue}t^3}
$$
$$
{\color{red}-}
\left(
\det\m{v_1}{e_2}{e_3}. 
+
\det\m{e_1}{v_2}{e_3}.
+
\det\m{e_1}{e_2}{v_3}.
\right)
{\color{blue}t^2}
$$$$
{\color{red}+}\left(
\det\m{v_1}{e_2}{e_3}.
+
\det\m{e_1}{v_2}{e_3}.
+
\det\m{v_1}{v_2}{e_3}.
\right){\color{blue}t}
$$
$$
{\color{red}-}\left(
\det\m{v_1}{v_2}{v_3}.
\right)
$$

\a\aa
If
$$
A=\m{v_1}{v_2}{v_3}{v_4}. = 
\m ****,****,****,****.
$$
as an example, then, 
$$
\det\m{e_1}{v_2}{e_3}{v_4}.
=
\det \m1*0*,0*0*,0*1*,0*0*.
=
\det \m1000,0*0*,0010,0*0*.
$$
Strategy of calculating such determinant
$$
\det \m1000,0{\h*}0{\h*},0010,0{\h*}0{\h*}.
=
\det\m {\h*}{\h*},{\h*}{\h*}.
$$
This submatrix has its diagonal lies in the same diagonal of original matrix.
\a\aa
\[defi]{A principle minor of size $k$ is the \x{determinant} of a $k × k$ submatrix, whose diagonal coincide with the diagonal of its father.}
Location of submatrices for principal minor of size $2$ of a $ 3 × 3$ matrix.
\def\q{{\h*}}
$$
\m\q\q*,\q\q*,***.
␣
\m***,*\q\q,*\q\q.
␣
\m\q*\q,***,\q*\q.
$$
\vfill

Location of submatrices for principal minor of size $1$ of a $ 3 × 3$ matrix.

$$
\m \q**,***,***.␣ 
\m ***,*\q*,***.␣ 
\m ***,***,**\q.
$$
\a\aa
\[thm]{The characteristic polynomial of n × n matrix $A$ has formula
$$
“det“(tI-A) = t^n - a_1t^{n-1}+a_2t^{n-2}- ... +(-1)^na_n
$$
where $a_i$ is the \x{sum} of principal minors of size $i$.
In particular $a_1=“tr“(A)$ and $a_n=“det“(A)$
}
\a\aa
\exe Calculate the charcteristic polynomial for the following matrices
$$
A = \m 111,111,111. ␣  B= \m 101,011,110.
$$
\aaa
\aaa{Lagurange interpolation polynomial}


\exe Suppose $f(x)=x-c$. Given that $f(2)=0$, determine the number $c$.

\a\aa
\exe Let f(x)=(x-a)(x-b). Determine the value of a,b such that 

\vfill
¡$f(3)=0 ␣ f(9) = 0$

\a\aa

\exe Consdier the following exercises 

 Suppose 
$$
f(x) = \frac{(x-2)(x-5)(x-7)(x-9)}{(11-2)(11-5)(11-7)(11-9)}.
$$
Please calculate the value of $f(x)$ to fill into the following value table
$$
\t{$x=$}{$f(x)=$},
2{},
5{},
7{},
9{},
{11}{}.
$$


\a\aa


Using the idea from previous exercise. Please write down a polynomial $f(x)$  such that 
$$
f(2)=f(3)=f(5)=f(8)=0,\qquad f(10)=1.
$$
\a\aa
 Write down polynomials $f_2(x),f_3(x),f_5(x),f_7(x)$ with the following value table
$$
\t{$x=$}{$f_1(x)=$}{$f_4(x)=$}{$f_5(x)=$},
1100,
4010,
5001.
$$
\a\aa
Try to determine a polynomial with its graph passing through the following points

\[tikzpicture]{[scale=0.38]\zbx{10}\func{-1}6{\x*\x/2-5*\x/2+3}\dian11\dian41\dian53}
\a\aa
We are trying to find $g(x)$ with $g(1)=1,g(4)=1,g(5)=3$. 

$$
\t{$x=$}{$f_1(x)=$}{$f_4(x)=$}{$f_5(x)=$}{$g(x)=$},
11001,
40101,
50013.
$$
and suppose that
$$
g(x)=a_1\cdot f_1(x)+a_4\cdot f_4(x)+a_5\cdot f_5(x).
$$
Determine the value of $a_1,a_4,a_5$.


$$
\t{$x=$}{$f_1(x)=$}{$f_4(x)=$}{$f_5(x)=$}{$a_1\cdot f_1(x)+a_4\cdot f_4(x)+a_5\cdot f_5(x)=$},
1100{a_1},
4010{a_4},
5001{a_5}.
$$
\aaa
\aaa{Lagurange Interpolation}
\[lem]{
Let $g(t)$ be an arbitray polynomial, let $x_1,x_2,...,x_n$ be distinct numbers. Let $f_{x_1},...,f_{x_n}$ be interpolation polynomials satisfying
$$
f_{x_i}(x_j) = 
\[cases]{
1 & x_i=x_j\\
0 & x_i ≠ x_j.
}
$$
Then the polynomial 
$$
h(t) := g(t) - g(x_1)f_{x_1}(t) - g(x_2)f_{x_2}(t) - ... - g(x_n)f_{x_n}(t)
$$
satisfies $h(x_1)=h(x_2)=...=h(x_n)=0$.
}
\a\aa
\[lem]{
If $h(x)$ is a polynomial satisfying
$h(x_1)=h(x_2)=...=h(x_n)=0$
for distinct inputs $x_i≠x_j$ for $i≠j$, then $h(x)$ is divisible by $(x-x_1)...(x-x_n)$ in the sense that
$$
h(t) = Q(t)(t-x_1)...(t-x_n)
$$
for some polynomial $Q$.
}
\a\aa
Explaination of the proof:

$$
\frac{h(t)}{t-x_1} = “Some polynomial“ + \frac{h(x_1)}{t-x_1}
$$
Therefore 
$$h(x_1)=0 ␣  ⟹   ␣ \frac{h(t)}{t-x_1} “ is a polynomial.“ $$
 Just write
$$
h_2(t) = \frac{h(t)}{t-x_1},
$$
then $h_2(x_2)=...=h_2(x_n)=0$. Use the method again and again, we know
$$\frac{h(t)}{(t-x_1)(t-x_2)\cdots(t-x_n)}$$
is a polynomial.
\a{Lagurange Interpolation Theorem}
\[thm]{Let $g(t)$ be arbitrary polynomials, given distinct points $x_1,...,x_n$ and a choice of interpolation polynomials $f_{x_1}(t),...,f_{x_n}(t)$ with $f_{x_j}(x_i)=0$ for $i≠j$ and $f_{x_i}(x_i)=1$ for any $1≤i≤n$, we can write
$$
g(t)=Q(t)(t-x_1)
\cdots
(t-x_n) + 
g(x_1)f_{x_1}(t)
+
%g(x_2)f_{x_2}(t)
%+
\cdots
+
g(x_n)f_{x_n}(t)
$$ }
\a\aa
\exe Write down an interpolation of $t^5$ at point $t=1$ and $t=2$.

\sol First we pick interpolation polynomials.
\t{$t=$}{$2-t$}{$t-1$}{$t^5$},
1101,
201{32}
.

Then we can write
$$
t^5 = Q(t)(t-1)(t-2) + (2-t) + 32(t-1)
$$
for some polyonimal $Q(t)$.
\aaa
