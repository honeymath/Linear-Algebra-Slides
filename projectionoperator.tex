
\aaa{Projection operators}

Projection operators are the most simple and intuitive, yet also the most important linear transformations. 
\[defi]{For a vector space $V$ over $F$, a projection operator $T : V ⟶  V$(or an idempotent) is an operator such that
$$
T^2 = T.
$$
}

\a{Geometric explaination of projection operator}


We will firstly introduce the idea of the projection operator by \x{sun light model} before explaining this definition.
\[exa]{\label{xiaomi}A vector standing straight at the origin (like a tree). The sunlight shines on it through some direction. The projection $T$ maps any vector $\vec v$ to its shadow $\vec w:=T\vec v$ on the floor.

\[tikzpicture]{[scale=0.5]
\draw[ultra thick](-7,0)--(7,0);
\draw[ultra thick,->,blue](0,0)--(0,3) node[right]{$\vec v$};
\draw[opacity=0.08,fill=cyan] (0,0)--(0,3)--(-3,0);
\draw[ultra thick,->,cyan](0,0.05)--(-3,0.05) node[below]{$\vec w:=T\vec v$};
\foreach\x in {-3,...,8}{
	\draw [red,->](\x,4)--(-4+\x,0.1);
}
\draw[fill=red] (0,0) circle[radius=0.1] node[below]{$O$};
}

The sunlight direction \x{needs not} to be perpendicular to the floor.
} 


\a\aa
\[rem]{In the definition, we have required $T^2=T$. Algebraically, this means that $T^2\vec v=T\vec v$ for any vector $\vec v$. Geometrically, it means that the shadow of a shadow is the shadow itself. Here is the explaination: Since $\vec w:=T\vec v$ represents the shadow of $\vec v$, the $\vec w$ lies on the floor. The shadow of a vector on the floor is itself. Therefore $T\vec w=\vec w$.}

\a{Why a projection operator?}

Intuitively, a projection needs to specify the floor and the sunlight. But we define them only by $T^2=T$.  Therefore, we need to answer the following questions:
\[enumerate]{
\item If $T^2=T$, how do we define the floor?
\item If $T^2=T$, how do we define the direction of sunlight?
\item After specifying the floor and the direction of sunlight, we have enough information to draw the projection geometrically. But why $T\vec v$ is the shadow of $\vec v$?
}

\a\aa
To answer this question, we need an example of an actural projection.
\[exa]{Let's assume $T$ is an actural projection. If a vector $\vec v$ is pointing to the sun, what does its shadow looks like? What is $\vec w=T\vec v$ in this case?

\[tikzpicture]{[scale=0.5]
\draw[ultra thick](-7,0)--(7,0);
\draw[ultra thick,->,blue](0,0)--(3,3) node[right]{$\vec v$};
%\draw[opacity=0.08,fill=cyan] (0,0)--(0,3)--(-3,0);
%\draw[ultra thick,->,cyan](0,0.05)--(-3,0.05) node[below]{$\vec w:=T\vec v$};
\foreach\x in {-3,...,8}{
	\draw [red,->](\x,4)--(-4+\x,0.1);
}
\draw[fill=red] (0,0) circle[radius=0.1] node[below]{$O$};
}

Should $\vec v$ be an element in the following set?
$$
\{\vec v\in V:T\vec v=\vec 0\}.
$$
Why this set describes the direction of the sunlight?
}

\a\aa
\[exa]{Let's assume $T$ is an actural projection. If a vector $\vec w$ is lying on the ground, what does its shadow looks like? What is $T\vec w$ in this case? Is $\vec w$ the same as $T\vec w$?

\[tikzpicture]{[scale=0.5]
\draw[ultra thick](-7,0)--(7,0);
\draw[ultra thick,->,blue](0,.1)--(3,.1) node[above]{$\vec w$};
%\draw[opacity=0.08,fill=cyan] (0,0)--(0,3)--(-3,0);
%\draw[ultra thick,->,cyan](0,0.05)--(-3,0.05) node[below]{$\vec w:=T\vec v$};
\foreach\x in {-3,...,8}{
	\draw [red,->](\x,4)--(-4+\x,0.1);
}
\draw[fill=red] (0,0) circle[radius=0.1] node[below]{$O$};
}

Should $\vec w$ be an element of the following set?
$$
\{\vec w:\vec w=T\vec v\text{ for some }\vec v\in V\}.
$$
}
\a\aa
\[exa]{
Furthermore, when $T$ is a projection, is the following statement equivalent?
$$
\vec w=T\vec v\text{ for some }\vec v\in V \iff T\vec w= \vec w.
$$
Which of the following set describes the floor? or both of them do?
$$
\{\vec w:\vec w=T\vec v\text{ for some }\vec v\in V\},\qquad \{\vec w\in V:\vec w=T\vec w\}.
$$
}

\a\aa
We define kernel and image for linear operators, this is just an analogue of null space and column space (if you think $T$ as a matrix)
\[defi]{For any linear operator $T : V ⟶  V$, the kernel and image of $T$ are defined by
$$
\ker(T):=\{\vec v:T\vec v=\vec 0\},
$$
$$\im(T):=\{\vec w:\vec w=T\vec v\text{ for some }\vec v\in V\}.
$$}
\a\aa
Come back to our abstract definition $T^2=T$. We may draw a line from the vector head along $\ker(T)$ until it hit $\im(T)$ to get its shadow, but why this is $T\vec v$? 

\[exa]{Let $T$ be an operator with $T^2=T$. By drawing $\ker(T)$ and $\im(T)$, the sunlight and the ground have been determined. Along the direction of the sunlight, one finds the shadow $\vec w$ of $\vec v$. Why $\vec w=T\vec v$?

\[zzz]{
\draw[ultra thick](-7,0)--(7,0) node[right]{$\mathrm{Im} T$};
\draw[ultra thick,red](-4,-4)--(4,4) node[right]{$\ker T$};
\draw[ultra thick,->,blue](0,0)--(0,3) node[right]{$\vec v$};
\draw[ultra thick,->,cyan](0,0.05)--(-3,0.05) node[below]{$\vec w$};
\foreach\x in {-3,...,8}{
	\draw [red,->](\x,4)--(-4+\x,0.1);
}
\draw[fill=red] (0,0) circle[radius=0.1] node[below]{$O$};
}
}
\a\aa
There are many operators with the same $\ker(T)$  and $\im(T)$ with $T$. 

\[zzz]{
\draw[ultra thick](-7,0)--(7,0) node[right]{$\mathrm{Im} T$};
\draw[ultra thick,red](-4,-4)--(4,4) node[right]{$\ker T$};
\draw[ultra thick,->,blue](0,0)--(0,3) node[right]{$\vec v$};
\draw[ultra thick,->,cyan](0,0.05)--(-3,0.05) node[below]{$\vec w$};
\foreach\x in {-3,...,8}{
	\draw [red,->](\x,4)--(-4+\x,0.1);
}
}

$T\vec v$ has to be a vector in $\im(T)$, but there are many choices. But only one choice makes
$$
T\vec v - \vec v ∈  \ker(T).
$$

Note that
$$
T^2 = T ⟺   T(T-I)=0 ⟺  T\vec v-\vec v ∈ \ker(T) “ for all “\vec v ∈ V.
$$
\a\aa

\[summ]{As long as an operator $T : V ⟶  V$ satisfies $T^2=T$, it is a projection such that
\[itemize]{
\item For any vector $\vec v\in V$, $T\vec v$ represents the shadow and $\vec v-T\vec v=(I-T)\vec v$ is a vector pointing to the direction of the sun.
\item $\{\vec v: T\vec v=\vec 0\}$ describes the direction of the sunlight, it consists of vectors $\vec v$ with $T\vec v=\vec 0$.
\item $\{\vec v: T\vec v=\vec v\}$ describes the floor, it consists of all possible shadows $\vec w=T\vec v$. In particular, for any $\vec w\in\im(T)$, we have $T\vec w=\vec w$.
}
}

\a\aa
Projection operator also has other equivalent definition.

\[prop]{Suppose $T : V ⟶  V$ is a linear operator, the following are equivalent
\[itemize]{
\item  $T^2=T$;
\item For any $\vec v ∈  “Im“(T)$, we have $T\vec v =\vec v$;
\item For any $\vec v ∈  V$, we have $T\vec v - \vec v ∈ \ker(T)$.
}
}

Proof: Exercise.





\aaa
