
\newcommand\map[3]{#1 : {#2}⟶  {#3}}
\newcommand\maps[5]{{#1}:{#2}⟶  {#3}, {#4} ↦ {#5}}
\aaa{Matrix of linear transformations}
To represent a linear transformation, we will use matrices.
\a\aa


In previous example, whenever we have a receipe table, it gives a linear transformation \x{from space of drink combinations} to \x{space of material combinations}.



\t{}\cola\tea,\bean21,\leaf11.



\[columns]{
\co4

\[zz]{
\org
%	\pgftransformcm{1}{2}{1}{1}{\pgfpoint{0}{0}}
\grid[red!60!white!40]{-1}{-1}11
\draw[color=orange] (1,1) circle[radius=0.3] node {\sbento};
\draw[color=red] (0,1) circle[radius=0.3] node {\scola};
\draw[color=red] (1,0) circle[radius=0.3] node {\stea};
	}

\co1
$$\longrightarrow$$
\co5

\[zzz]{
\org
\grid{-3}{-3}33
\draw[color=blue] (1,0) circle[radius=0.3] node {\sleaf} (0,1) circle[radius=0.3] node {\sbean};
\draw[color=red] (1,2) circle[radius=0.3] node {\scola} (1,1) circle[radius=0.3] node {\stea};
\draw[color=orange] (2,3) circle[radius=0.3] node {\sbento};
	\pgftransformcm{1}{2}{1}{1}{\pgfpoint{0}{0}}
\grid[red!60!white!40]{-1}{-1}11
%\draw (1,1) node {\sbento};
	}


}


\a\aa
If we call this map $T$. Then we use \cola, \tea as symbols for those drinks in the domain. And 
$$T\left(\cola\right)\qquad T\left(\tea\right)$$ as symbols for its position in the codomain. Since materials are all in the codomain, it makes more sense to write our table as


\t{}{$T\left(\cola\right)$}{$T\left(\tea\right)$},\bean21,\leaf11.

\a\aa
This table can be written as an expression

$$
\m{T\cola}{T\tea}. = \m\bean\leaf. \m21,11.
$$

Factor $T$ out, we can write
$$
T\m{\cola}{\tea}. = \m\bean\leaf. \m21,11.
$$

Note that here (\cola,\tea) is a basis of the domain, and (\bean,\leaf) is a basis of the codomain. We call the matrix
$$
\m21,11.
$$
The matrix representation of $T$ in the basis (\cola,\tea) and (\bean,\leaf). It determines the linear transformation completely.



\aaa





\aaa{Matrix Representation of Linear Transformation}
\[defi]{For a linear transformation $\map TVW$, let 
\[itemize]{
\item $\ce=\obasis\vv n$ be a basis of domain $V$
\item $\cw=\obasis\ww m$ be a basis of codomain $W$. 
}The \x{matrix representation} of $T$ with respect to $\ce$ and $\cw$, is the matrix $P$ such that
$$
T\obasis\vv n=\obasis\ww m P
$$
%We denote by $P=\nota T\ce\cw$
}
In other words, the matrix representation is the recipe table to make $T\obasis\vv n$ by materials $\obasis\ww m$.
\a\aa
The matrix representation $P$ of $\map TVW$ with basis $\ce$ and $\cw$, is the coordinate matrix of
$$
T\ce=\obasis{T\ee}n
$$
in the following basis of codomain $W$
$$
\cw = \obasis\ww m.
$$
The matrix $P$ fits into the following linear combination equation
$$
\overbrace{\obasis{T\ee}n}^{T\ce} = \overbrace{\obasis\ww m}^{\cw}P 
$$
Each column of $P$ is the coordinate of $T\ee_i$ in the \bas $\cw$.
$$
P=\m{\nota{T\ee_1}{}\cw}{\nota{T\ee_2}{}\cw}\cdots{\nota{T\ee_n}{}\cw}.
$$
\a\aa
\exe Let $V=\lPP xabc$, $W=\lPP tabc$

Consider a linear map 
$$\maps TVW{f(x)}{f(t+1)}$$

Find matrix representation of $T$ with bases 
$$\cw=\m1t{t^2}.\text{ in }V\qquad \ce=\m1{2x+1}{x^2+1}.\text{ in }W$$

\a\aa
\sol: Apply the \lt $T$ on each of the function on \bas and write the coordinate in \bas of the target.
We find 
\[equation*]{\[split]{
	T(1)=1 &={\underbrace{\m1t{t^2}.}_\cw\underbrace{\m1,0,0.}_{[T(1)]^\cw}}\\
T(2x+1)=2(t+1)+1 &={\underbrace{\m1t{t^2}.}_\cw\underbrace{\m3,2,0.}_{[T(2x+1)]^\cw}}\\
T(x^2+1)=(t+1)^2+1&={\underbrace{\m1t{t^2}.}_\cw\underbrace{\m2,2,1.}_{[T(x^2+1)]^\cw}}
}
}

\a\aa
We write this into a matrix form
$$
T\underbrace{\m1{2x+1}{x^2+1}.}_\ce=\underbrace{\m1t{t^2}.}_\cw\m132,022,001.
$$

We know the matrix representation of $T$ is $ \m132,022,001.$

\aaa








