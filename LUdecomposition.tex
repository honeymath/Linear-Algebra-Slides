
\aaa{LDU decomposition by row operation}
$$
\tiny{\m111,122,123. = \m100,010,001. \m100,010,001.^{-1} \m111,122,123.}
$$
$$
\xequal{
\substack{
r_2=r_2-r_1\\
r_3=r_3-r_1
}
}\m100,010,001. \m100,{-1}10,{-1}01.^{-1} \m111,011,012.
$$
$$ \xequal{
r_3=r_3-r_2
}\m100,010,001. \m100,{-1}10,0{-1}1.^{-1} \m111,011,001.  $$
$$ \xequal{
c_2=c_2+c_3
}\m100,010,011. \m100,{-1}10,0{0}1.^{-1} \m111,011,001.  $$
$$ \xequal{
c_1=c_1+c_2
}\m100,110,111. \m100,{0}10,0{0}1.^{-1} \m111,011,001.  $$
\aaa




\aaa{Forwarded Row Operation v.s. Cross Filling}
%%% Seems also good to say step by step. remainder of corss filling
One step cross-filling

$$
\m 222,233,234. = 
\m{\h2}{\y2}{\y2},{\y2}22,{\y2}22.
 + \m000,011,012.
$$

$$
= \m1,1,1.\m222.+\m00,10,01.\m011,012.
$$

$$
= \m100,{\x1}10,{\x1}01.\m{\h2}22,{\x0}11,{\x0}12.
$$
One step cross-filling = forward row operation

Cross-center = pivot

\a\aa

Look at the following cross-filling decomposition
$$
\m 11111,11333,11338. 
$$
$$= \m{\h1}{\y1}{\y1}{\y1}{\y1},{\y1}1111,{\y1}1111.
+
\m00{\y0}00,{\y0}{\y0}{\h2}{\y2}{\y2},00{\y2}22.
+
\m0000{\y0},0000{\y0},{\y0}{\y0}{\y0}{\y0}{\h5}.
$$
This decompose the matrix into the product
$$
\m100,110,111. \m
{\h1}{\y1}{\y1}{\y1}{\y1},
{0}{0}{\h2}{\y2}{\y2},
0000{\h5}.
$$
Cross-center = pivot
\a\aa
Don't forgot that you can also look column by column
$$
\m{\h1}00,{\y1}{\h2}0,{\y1}{\y2}{\h5}.
\m11111,00111,00001.
$$
The left factor is in fact obtained by \x{forward column operation.}
\a{Pro, Cons of two method.}

\t{Cross-Filling}{Row/Col Operation},
{Symmetric on rows/cols}{Only emphasis in one of row/columns},
{Useful for projection operator}{Easy for calculating inverse}.
\aaa




\aaa{Algebraic expression for cross-filling}
Can you write first summand into a product of matrices?
$$
A=\m 22222,22444,22449. 
$$
$$= \m{\h2}{\y2}{\y2}{\y2}{\y2},{\y2}2222,{\y2}2222.
+
\m00{\y0}00,{\y0}{\y0}{\h2}{\y2}{\y2},00{\y2}22.
+
\m0000{\y0},0000{\y0},{\y0}{\y0}{\y0}{\y0}{\h5}.
$$

\a\aa

$$
\m{\h2}{\y2}{\y2}{\y2}{\y2},{\y2}2222,{\y2}2222.
=
\m{\h2},{\y2},{\y2}.\frac1{\m{\h2}.}\m{\h2}{\y2}{\y2}{\y2}{\y2}.
$$
$$
=Ae_1(e_1'Ae_1)^{-1}e_1'A
$$
where
$$
e_1=\m1,0,0,0,0. ␣ e_1' = \m100.
$$
The expression \x{$Ae_1(e_1'Ae_1)^{-1}e_1'A$} is algebraic expression for cross-filling.

\a\aa

\exe $A$ is $m × n$ matrix, let $e_i$ be $i$th column of $I_n$, and $e_j'$ be $j$'th row of $I_m$. The corss-filling matrix obtained from the center at $j$'th row $i$'th column is given by
\vfill
\textbf{Answer}: $Ae_i(e_j'Ae_i)^{-1}e_j'A$.


\vfill

Algebraic expression is useful when writting proof. 

\a\aa

\exe A \x{projection matrix} P is a square matrix with $P^2=P$. Suppose $P$ is a projection matrix, and $P'$ is a matrix obtained from $P$ by \x{cross filling}. Is $P'$ again a projection matrix? 

\vfill

\textbf{Solution}: Suppose $P$ is $n × n$ matrix. Let $e_i$ be i'th column of $I_n$, then
$$
P' = Pe_i(e_j^TPe_i)^{-1}e_j^TP
$$
By calculation 
$$
P'P' = Pe_i(e_j^TPe_i)^{-1}e_j^TPPe_i(e_j^TPe_i)^{-1}e_j^TP
$$
$$
=Pe_i(e_j^TPe_i)^{-1}e_j^TPe_i(e_j^TPe_i)^{-1}e_j^TP
$$
$$
= Pe_i(e_j^TPe_i)^{-1}e_j^TP = P'.
$$

In the future class, the cross-filling method is \x{ extremely important for projection matrix}

\aaa




\aaa{Transpose of matrices}
Recall that when we try explaining rows of matrices,
$$
\t{}\bento\soup,\leaf{2}2,\lemon{\h1}{\h1},\bean{0}4,\cow{2}1.
$$
\alert{DO NOT MEAN}
$$
\lemon = 1\cdot\bento +1\cdot\soup \text{(Wrong)}
$$


Actual meaning:  $
[\lemon] = 1\cdot[\bento] +1\cdot[\soup] \text{(Correct)}
$
$$
\text{The number of }\lemon =1\cdot\text{The number of }\bento + 1\cdot\text{The number of }\soup.
$$
\a\aa
Therefore, a matrix of \x{making meals}, can therefore give a matrix of \x{making counters}.
$$
\t{}\bento\soup,\leaf{2}2,\lemon{1}{1},\bean{0}4,\cow{2}1.
$$

$$
\t{}{[\leaf]}{[\lemon]}{[\bean]}{[\cow]},
{[\bento]}2104,
{[\soup]}2141.
$$
This is the transpose of the matrix.
\a\aa
Since the transpose exchanges the position of the material list and compund list, the order of the matrix multiplication should change.
$$
\underbrace{
\t{}\milk\coffee\tea,\leaf{0}{0}{2},\lemon{0}{0}{1},\bean{0}{2}{0},\cow{1}{0}{0}.
}_A
\cdot
\underbrace{
\t{}\bento\soup,
\milk{2}{1},
\coffee{0}{2},
\tea{1}{1}.
}_B\quad=
\underbrace{
\t{}\bento\soup,\leaf{2}2,\lemon{\alert1}{\alert1},\bean{0}4,\cow{2}1.
}_C
$$

$AB = C$
\a\aa

$$
\underbrace{\t{}{[\milk]}{[\coffee]}{[\tea]},
{[\bento]}201,
{[\soup]}121.
}_{B^T}
·
\underbrace{\t{}{[\leaf]}{[\lemon]}{[\bean]}{[\cow]},
{[\milk]}0001,
{[\coffee]}0020,
{[\tea]}2100.
}_{A^T}
$$

$$
=\underbrace{\t{}{[\leaf]}{[\lemon]}{[\bean]}{[\cow]},
{[\bento]}2102,
{[\soup]}2141.}_{C^T}
$$

$B^TA^T=C^T$.
\a\aa
Therefore
\begin{thm}
$$
AB = C \iff B^TA^T = C^T
$$
\end{thm}
\aaa


\aaa{Inverse of matrices}
\exe Recall the way we introduce inverse is to flip the materials and compunds each other 
\vfill
\[columns]{\co 5

\t{}\milk\coffee,\bean02,\cow10.

\co 5 
\t{}\bean\cow,\milk01,\coffee{0.5}0.
}

Please use the same logic, try to understand why 
$$
AB = C \iff B^{-1}A^{-1}=C^{-1}.
$$

\a\aa
The expression can be deduced formally.

$$
AB=C \iff B=A^{-1}C \iff BC^{-1}=A^{-1} \iff C^{-1}=B^{-1}A^{-1}.
$$

Another story to understand this:
\vfill
You put socks on then shoes on ; But when take them off, you  put shoes off and then socks off.

\aaa






\aaa{LDU decomposition for symmetric matrix}
If $A$ is symmetric, we have $A=A^T$. 


Suppose
$$
A = LDU.
$$
Therefore $A^T = U^TD^TL^T$. 
If $A$ is symmetric, we have $A=A^T$, so 
$$
U^TD^TL^T = LDU
$$
Furthermore, if $A$ is invertible, its $LDU$ decomposition is unique, then
$$
U=L^T
$$
We can write LDU decomposition of invertible symmetric matrix $A$ as $LDL^T$.
\a\aa
\textbf{Warning}: Not all matrix is LDU decomposible (even invertible matrices). For example
$$
\m 01,13.
$$
is symemtric but not $LDU$-decomposable.
\a{Positive definite matrix}
However, things are nice for so-called \x{positive definite matrix}
\[defi]{
An $n × n$ matrix $A$ is called \x{positive definite} if 
$$
v^T Av > 0
$$
for any $v≠0$.
}

\x{YOU MUST REMEMBE THE FOLLOWING IMPORTANT THEOREM}
\[thm]{
A positive definite matrix $A$ is both \x{invertible} and \x{LDL^T}-decomposable, all diagonal entries of $D$ are positive numbers!
}
\a\aa
Recall: Suppose $A$ is $ n × n$ matrix.
$$
“LDU-decomposable “\iff “ cross-fillable with center on \x{diagonal}“
$$
$$
“invertible “ \iff “ decomposing into “ \x{n} “ matrices with cross-filling“ 
$$
In other words, the theorem is equivalent as saying.
$$
“positive definite matrix cross-fillable along diagonals one by one.“
$$
\a\aa
Let $A$ be a positive definite matrix, we need to start by proving its first element non-zero.

First diagonal element $e_1^TAe_1>0$ (recall $e_1$ first column of $I_n$). So is \x{cross-fillable}!

$$
\underbrace{\m111,122,123.}_A = \m{\h1}{\y1}{\y1},{\y1}11,{\y1}11.+\m000,0{\x1}{\x1},0{\x1}{\x2}.
$$ 
The process will finish if 
$$
A_1 = \m11,12.
$$
is again a positive definite matrix, then an induction can be use.
\a\aa
$$
\underbrace{\m111,122,123.}_A = \m{\h1}{\y1}{\y1},{\y1}11,{\y1}11.+\m000,0{\x1}{\x1},0{\x1}{\x2}.
$$
So why 
$$
A_1 = \m11,12.
$$
is positive definite? That is why
$$
\m xy.\m11,12.\m x,y. >0
$$
for $\m xy.≠\m00.$?
\a\aa
The idea is simple, firstly, realize that
$$
\m xy.\m11,12.\m x,y.= \m {\x*}xy.\m000,0{\x1}{\x1},0{\x1}{\x2}.\m{\x*},x,y.
$$
Then we may simply choose $*$ to make sure
$$
\m {\x*}xy.\m{\h1}{\y1}{\y1},{\y1}11,{\y1}11.\m{\x*},x,y.=0
$$
In our case, $*=-(x+y)$. This implies that
$$
\m xy.\underbrace{\m11,12.}_{A_1}\m x,y. = \m {\x {-(x+y)}}xy.A\m{\x {-(x+y)}},x,y. >0
$$
So $A_1$ is positive definite, and therefore the cross-filling process may continue.
\aaa

\aaa{Connection between two ways of representing subspaces}

%%% Cross Filling and LDU decomposition.  LDU decomposition is unique.
\a{Cross-Filling and LDU decomposition}
We explain the connection between cross-filling and matrix LDU decomposition. 
\vfill
\a\aa
Suppose a cross filling decomposition has been made by \x{choosing cross center on diagonals}
$$
\m111,132,{-1}14.=\m
{\y1} {\y1} {\y1},
{\y1}11,
{\y-1}{-1}{-1}.
+
\m0{\y0}0,{\y0}{\y2}{\y2},0{\y2}2.
+
\m
00{\y0},
00{\y0},
{\y0}{\y0}{\y3}.
$$

\a\aa
Observation, (think from perspective of column actions)
$$
\m
{\y1} {\y1} {\y1},
{\y1}11,
{\y-1}{-1}{-1}.
=
\m
{\y1} {*} {*},
{\y1}**,
{\y-1}**.
\m
{\y1} {\y1} {\y1},
000,
000.
$$
or (row actions)
$$
\m
{\y1} {\y1} {\y1},
{\y1}11,
{\y-1}{-1}{-1}.
=
\m
{\y1} {0} {0},
{\y1}00,
{\y-1}00.
\m
{\y1} {\y1} {\y1},
***,
***.
$$
\a\aa
Now you can see 
$$
\m
{\y1} {\y1} {\y1},
{\y1}11,
{\y-1}{-1}{-1}.
=
\m
{\y1} {*} {*},
{\y1}**,
{\y-1}**.
\m100,000,000.
\m
{\y1} {\y1} {\y1},
***,
***.
$$
\a\aa
\textbf{Question:} Why the following is true?
$$
\m0{\y0}0,{\y0}{\y{\x2}}{\y2},0{\y2}2. = 
\m*{\y0}*,*{\y2}*,*{\y2}*.
\m0{}{},{}{\frac1{\x2}}{},{}{}0.
\m***,{\y0}{\y2}{\y2},***.
$$
A \x{scaled} version:
$$
\m0{\y0}0,{\y0}{\y{\x2}}{\y2},0{\y2}2. = 
\m*{\y\frac0{\x2}}*,*{\y\frac{2}{\x{2}}}*,*{\y\frac{2}{\x{2}}}*.
\m0{}{},{}{\x2}{},{}{}0.
\m***,{\y\frac0{\x2}}{\y\frac{2}{\x{2}}}{\y\frac{2}{\x{2}}},***.
$$

\a\aa

$$
\m
{\y{\x1}} {\y1} {\y1},
{\y1}11,
{\y-1}{-1}{-1}.
+
\m0{\y0}0,{\y0}{\y{\x2}}{\y2},0{\y2}2.
+
\m
00{\y0},
00{\y0},
{\y0}{\y0}{\y{\x3}}.
$$
\a\aa
$$
\m
{\y{\x1}} {\y1} {\y1},
{\y1}11,
{\y-1}{-1}{-1}.
=
\m
{\frac{\x1}{\x1}}**,
{\frac{1}{\x1}}**,
{\frac{-1}{\x1}}**.
\m{\x1}{}{},{}0{},{}{}0.
\m{\frac{\x1}{\x1}}{\frac{1}{\x1}}{\frac{1}{\x1}},
***,
***.
$$
\vfill
$$
\m0{\y0}0,{\y0}{\y{\x2}}{\y2},0{\y2}2.=
\m
*0*,
*{\frac{\x2}{\x2}}*,
*{\frac{2}{\x2}}*.
\m{0}{}{},{}{\x2}{},{}{}0.
\m
***,
0{\frac{\x2}{\x2}}{\frac{2}{\x2}},
***.
$$
\vfill

$$
\m
00{\y0},
00{\y0},
{\y0}{\y0}{\y{\x3}}.=
\m
**0,
**0,
**{\frac{\x3}{\x3}}.
\m{0}{}{},{}{0}{},{}{}{\x3}.
\m
***,
***,
00{\frac{\x3}{\x3}}.
$$



\a\aa
$$
\m
{\y{\x1}} {\y1} {\y1},
{\y1}11,
{\y-1}{-1}{-1}.
=
\m
{1}**,
{1}**,
{-1}**.
\m{\x1}{}{},{}0{},{}{}0.
\m111,
***,
***.
$$
\vfill
$$
\m0{\y0}0,{\y0}{\y{\x2}}{\y2},0{\y2}2.=
\m
*0*,
*{1}*,
*{1}*.
\m{0}{}{},{}{\x2}{},{}{}0.
\m
***,
011,
***.
$$
\vfill

$$
\m
00{\y0},
00{\y0},
{\y0}{\y0}{\y{\x3}}.=
\m
**0,
**0,
**{1}.
\m{0}{}{},{}{0}{},{}{}{\x3}.
\m
***,
***,
00{1}.
$$


\a\aa
$$
\m
{\y{\x1}} {\y1} {\y1},
{\y1}11,
{\y-1}{-1}{-1}.
=
\m
{1}00,
{1}10,
{-1}11.
\m{\x1}{}{},{}0{},{}{}0.
\m111,
011,
001.
$$
\vfill
$$
\m0{\y0}0,{\y0}{\y{\x2}}{\y2},0{\y2}2.=
\m
{1}00,
{1}10,
{-1}11.
\m{0}{}{},{}{\x2}{},{}{}0.
\m111,
011,
001.
$$
\vfill

$$
\m
00{\y0},
00{\y0},
{\y0}{\y0}{\y{\x3}}.=
\m
{1}00,
{1}10,
{-1}11.
\m{0}{}{},{}{0}{},{}{}{\x3}.
\m111,
011,
001.
$$

\a\aa

$$
\m111,132,{-1}14. = \underbrace{\m
{1}{}{},
{1}1{},
{-1}11.}_L
\underbrace{\m{\x1}{}{},{}{\x2}{},{}{}{\x3}.}_D
\underbrace{\m111,
{}11,
{}{}1.}_U
$$
\a\aa

\begin{prop}
Let $A = P_1+P_2+\cdots+P_m$ be a cross-filling decomposition of $n\times n$ matrix. Suppose $A=LDU$ for some lower triangular $L$, upper $U$ an diagonal $D$ such that diagonals of $L$ and $U$ are all $1$. Assume
$$
D = \m{d_1}{}{}{}{},{}\ddots{}{}{},{}{}{d_i}{}{},{}{}{}\ddots{},{}{}{}{}{d_n}.\qquad
\underbrace{D_i = \m{0}{}{}{}{},{}\ddots{}{}{},{}{}{d_i}{}{},{}{}{}\ddots{},{}{}{}{}{0}.}_{\text{only keep ith entry}}
$$
Then $P_i = LD_iU$.
\end{prop}
\a\aa
This proposition has the following conclusions
\begin{enumerate}
\item The i'th diagonal entry $d_i$ of $D$ is the \x{cross-filling center} of $P_i$
\item Let $\vec c_i$ be i'th column of $L$, then the scaled column $\vec c_i\cdot d_i$ is the \x{column of the cross} of  $P_i$
\item Let $r_i$ be i'th row of $U$, then the scaled row $d_i\cdot r_i$ is the \x{row of the cross} of  $P_i$.
\end{enumerate}

In other words, the decomposition $A=LDU$ stores all the data of the cross-filling \x{along diagonals}. Therefore all matrices $L,D,U$ is \x{uniquely determined} by $A$.



%%%%%%
%Comments on triangular system of linear equations. 

\a{Triangular system of linear equations}
Our previous method of solving linear equations with cross-filling can be formally described by triangular systems.  

\a\aa
Consider previous linear system

$$
\underbrace{\m111,132,{-1}14.}_A\m x,y,z.=\m2,5,4.
$$
The cross filling
$$
\m111,132,{-1}14.=\m
{\y1} {\y1} {\y1},
{\y1}11,
{\y-1}{-1}{-1}.
+
\m0{\y0}0,{\y0}{\y2}{\y2},0{\y2}2.
+
\m
00{\y0},
00{\y0},
{\y0}{\y0}{\y3}.
$$

yields to a $LDU$ decomposition
$$
\m111,132,{-1}14. = \underbrace{\m
{1}{}{},
{1}1{},
{-1}11.}_L
\underbrace{\m{\x1}{}{},{}{\x2}{},{}{}{\x3}.}_D
\underbrace{\m111,
{}11,
{}{}1.}_U
$$

\a\aa
Each row of $DU$ is the \x{rows of each cross},
$$
\m111,132,{-1}14. = \underbrace{\m
{1}{}{},
{1}1{},
{-1}11.}_L
\underbrace{\m111,
{}22,
{}{}3.}_{DU}
$$
Each rows of the cross is exactly each intermediate equations that helpped us solving equations before.  Let
$$
\m u,v,w. := \underbrace{\m111,
{}22,
{}{}3.}_{DU}\m x,y,z.
$$
Then $u,v,w$ are those intermeidate equations.
\a\aa
The original equation just have the form
$$
\underbrace{\m
{1}{}{},
{1}1{},
{-1}11.}_L\m u,v,w.=\m a,b,c.,\qquad
\underbrace{\m111,
{}22,
{}{}3.}_{DU}\m x,y,z.=\m u,v,w.
$$

Solving triangular system is just by substitutions. For example $u=a$, $v = b-au=b-a^2$, and so on...
%% For matrices that can not be LDU, apply some row operation. 


\a{Not all matrices are LDU-decomposable}

A matrix is LDU-decomposable if and only if it can be cross-filled from diagonal. But this is not always the case. For example

$$
A = \m 03,12.
$$ 
It just fails from the first step, we got 0 on diagonal, preventing us from cross-filling from it.

\a\aa
Even if some matrix can be cross-filled from the first step, \x{the second step might fail} if we insists of picking diagonals

$$
\m123,124,135. = \m {\y1}{\y2}{\y3},{\y1}23,{\y1}23.+\underbrace{\m000,0{\x0}1,012.}_{\substack{????oh! NO!\\\text{a non-zero cross}\\\text{with zero center..}}}
$$
\a\aa
The problem comes from our restriction to ourselves. 
\vfill
WHY picking diagonals only? 
\vfill
A lot of trouble in this world come from the lack of freedom. 
\vfill
We ask for more freedom!
\a{P-LDU decomposition}
Now we are allow to do cross-filling following the order of columns, and lift the restriction on rows. This allows us to do more.

\a\aa
The following matrix is not LDU decomposable, but we may decompose it the following way
$$
\m0000,1234,1111,1358.=\m {\y0}000,{\y1}111,{\y1}{\y1}{\y1}{\y1},{\y1}111. + \m 0{\y0}00,{\y0}{\y1}{\y2}{\y3},0{\y0}00,0{\y2}46.+\m000{\y0},000{\y0},000{\y0},{\y0}{\y0}{\y0}{\y1}.
$$
For PLDU decomposition, every time we \x{must} choose the first non-zero column. But no restriction on rows.
\a\aa
The point is that the corss center can be distributed on diagonal after applying row operations on both side
$$
\underbrace{\m{}{}1{},{}1{}{},1{}{}{},{}{}{}1.}_P\underbrace{\m0000,1234,1111,1358. }_A
$$$$= \m {\y1}{\y1}{\y1}{\y1},{\y0}000,{\y1}111,{\y1}111. + \m 0{\y0}00,{\y0}{\y1}{\y2}{\y3},0{\y0}00,0{\y2}46.+\m000{\y0},000{\y0},000{\y0},{\y0}{\y0}{\y0}{\y1}.
$$
\a\aa
This gives the decomposition
$$
PA = \underbrace{\m1000,0100,1000,1201.}_L\underbrace{\m1000,0123,0000,0001.}_U.
$$
Therefore, we can always find some row operator $P$, such that $PA$ can be decomposed into LDU.
%% Invertibility, A triangular matrix is invertible if all its diagonal is non-zero
\a{Comments}
When we write it into LDU with one of the diagonal entry of D equal to 0, we can assign arbitray numbers in that corresponding col and ros of L and U
$$
\underbrace{\m1000,0100,1000,1201.}_L\underbrace{\m1000,0123,0000,0001.}_U.
$$
$$
=\underbrace{\m1000,0100,10*0,12*1.}_L\m1{}{}{},{}1{}{},{}{}0{},{}{}{}1.\underbrace{\m1000,0123,00**,0001.}_U.
$$
In other words, we may always assume $L$ and $U$ are triangular matrices with $1$ on diagonal.

\a{Column and row switchings together}
If we allow both column and row switchings, we have \x{complete freedom}

$$
\underbrace{\m{2}{1}{3},3{2}5,3{1}{-3}.}_A = \underbrace{\m{\y2}{\y1}{\y3},4{\y2}6,2{\y1}{3}.}_{P_1} + \underbrace{\m{\y0}00,{\y-1}0{6},{\y1}{\y0}{\y-6}.}_{P_2}+\underbrace{\m00{\y0},{\y0}{\y0}{\y-7},00{\y0}.}_{P_3}
$$
Then
$$
\m1{}{},{}{}1,{}1{}.A\m{}1{},1{}{},{}{}1.=
$$
$$
\m{\y1}{\y2}{\y3},{\y2}46,{\y1}2{3}. + \m0{\y0}0,{\y0}{\y1}{\y-6},0{\y-1}{6}.+\m00{\y0},00{\y0},{\y0}{\y0}{\y-7}.
$$
which can be decomposed into $LDU$.
\a{The invertibility of a matrix}
Observation:

\[prop]{A triangular matrix is invertible if and only if all its diagonal entries are non-zero.
}
\a\aa
\[prop]{An $n\times n$ matrix $A$ is invertible if and only if it can be decomposed into $n$ many rank-1 matrices by cross-filling.
}

A cross-filling corresponds to a decomposition of the form $P_1LDUP_2$, where all $P_1,L,U,P_2$ are invertible. The number of non-zero entries in $D$ corresponds to the number of rank 1 matrices in the decomposition.
%% Say something about rank. 

%% Say something about LDU by row operations.


%% 需要调整顺序

% 把 简化过程放在 UL 分解之后讲。cross-filling 就讲cross-filling,

%先只讲along diagonals for simplicity,然后再讲可以不along diagonals。

% 最后讲invertibility即可

% 然后讲对称矩阵如果有分解,那么分解一定是 R^TR之类的,这种时候当且仅当矩阵半正定(未来讲)

% 把分块矩阵乘法单独拿出来讲,以后还有用,在diagonalization有用。

\aaa
