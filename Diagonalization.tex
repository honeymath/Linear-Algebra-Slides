
\aaa{Diagonalization}

\[defi]{We call matrix $A$ similar to matrix $B$ if there is an invertible matrix $P$ such that $B = P^{-1}AP$
}
\a\aa
If $B=P^{-1}AP$, then we see

$$
B^n = \underbrace{P^{-1}AP P^{-1}AP\cdots P^{-1}AP}_{n-“many“} = P^{-1}A^nP
$$
Therefore, for any polynomial $f$, we have
$$
f(B)=f(P^{-1}AP)=P^{-1}f(A)P
$$
\a\aa

\[defi]{A matrix $A$ is called diagonalizable if it is similar to a diagonal matrix $Λ$
$$
P^{-1}AP = \m{λ_1}{}{}{},{}{λ_2}{}{},{}{}\ddots{},{}{}{}{λ_n}.
$$
}



\aaa
