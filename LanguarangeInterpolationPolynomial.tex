
\aaa{Computing polynomials of a matrix}

In previous slides, we leared that $“char“_A(t)=“det“(tI-A)$ is an annihilating polynomial, and we call the list of its roots the list of \x{eigenvalues}
\vfill

The imporatance of the characteristic polynomial is that $“char“_A(A)=0$, the Caylay Hamilton theorem. 
\vfill
In this part, we introduce an easier method to calculate $g(A)$ for general polynomial $g$ in this section. This enables us to calculate formulas like $A^n$ or $e^A$ for various purposes. This motivate us to define \x{eigenvectors} nand \x{eigenspaces}

\a{Review Lagurange Interpolation Polynomial}
\x{Interpolation} means to find a function with its graph passing through certain points,

\[columns]{\co6
\[tikzpicture]{[scale=0.30]\zbx{10}\func{-1}6{\x*\x/2-5*\x/2+3}\dian11\dian41\dian53}
\co4
 the left is a quadratic polynomial interpolation of 
\t x145,y113.


The interpolation is given by $f(x) = \frac{x^2-5x+6}2$
}
\a\aa
Note that the \x{interpolation is not unique}, for example, the following is also a polynomial interpolating of $(x,y) = (1,1), (4,1), (5,3)$
\[columns]{\co5
\[tikzpicture]{[scale=0.30]\zbx{10}\func{-1}6{(\x*\x*\x - 10*\x*\x + 29*\x - 20)/3 + \x*\x/2-5*\x/2+3}\dian11\dian41\dian53}
\co5
The polynomial used here is 
$$
{\tiny\frac{(x-1)(x-4)(x-5)}3 + \frac{(x-2)(x-3)}2.}
$$
}
\a\aa

\[columns]{
\co6

\[tikzpicture]{[scale=0.30]\zbx{10}
\func{-1}6{(\x*\x*\x - 10*\x*\x + 29*\x - 20)/3 + \x*\x/2-5*\x/2+3}
\func{-1}6{(\x*\x*\x - 10*\x*\x + 29*\x - 20)/4 + \x*\x/2-5*\x/2+3}
\func{-1}6{(\x*\x*\x - 10*\x*\x + 29*\x - 20)/6 + \x*\x/2-5*\x/2+3}
\func{-1}6{(\x*\x*\x - 10*\x*\x + 29*\x - 20)/(-6) + \x*\x/2-5*\x/2+3}
\func{-1}6{(\x*\x*\x - 10*\x*\x + 29*\x - 20)/(-10) + \x*\x/2-5*\x/2+3}
\dian11\dian41\dian53}
\co4
We have all arbitrary choices of interpolation polynomials for $(x,y) = (1,1), (4,1), (5,3)$. What is the difference between any two of the interpolation polynomial?
}
\a\aa
\[prop]{Suppose $x_1,...,x_k$ are distinct numbers. Let $g(x)$ and $h(x)$ interpolating the same data set in the sense that
$$
g(x_i)= h(x_i) = y_i 
$$
for 
$$
(x_i,y_i) ∈ \{(x_1,y_1),(x_2,y_2),\cdots (x_k,y_k) \}.
$$
Then $g(x)-h(x)$ is divisible by $(x-x_1)\cdots(x-x_k)$.
}
\a\aa
Therefore, 
$$
Q(x) = \frac{g(x)-h(x)}{(x-x_1)\cdots(x-x_k)}
$$
is a polynomial. We may write
$$
g(x) = Q(x)(x-x_1)\cdots(x-x_k) + h(x).
$$
for any two interpolation of the data at $x_1,...,x_k$.

\a\aa
For any polynomial $g(x)$, it interpolate itself at

\t{$x_1$}{$x_2$}{$\cdots$}{$x_n$},
{$g(x_1)$}{$g(x_2)$}{$\cdots$}{$g(x_n)$}.

Using Lagurange interpolation polynomial, we may also construct

$$
h(x) = g(x_1)\frac
{(x-x_2)\cdots(x-x_n)}
{(x_1-x_2)\cdots(x_1-x_n)}
+
\cdots
+
g(x_n)\frac
{(x-x_1)\cdots(x-x_{n-1})}
{(x_n-x_1)\cdots(x_n-x_{n-1})}
$$
Note that $h(x)$ also interpolates the same data, we have $g(x_1)=h(x_1)$, ..., $g(x_n)=h(x_n)$ . We have
$$
g(x)=Q(x)(x-x_1)\cdots(x-x_n) + h(x).
$$
\a\aa
\exe Write down an interpolation of $t^5$ at point $t=1$, $t=2$ and $t=3$.



\sol We have the following table

\t{}{$t=1$}{$t=2$}{$t=3$},
{$t^5$}{$\y\bf1$}{\color{green}$\h\bf32$}{\e\bf\color{white}243}.

The Lagurange interpolation at these 3 points is given by the following Lagurange interpolation.
$$
\t{\y\bf1}.
\underbrace{\frac{(t-2)(t-3)}{(1-2)(1-3)}}_{
\footnotesize
\t{t=1}{t=2}{t=3},100.}
+
\t{\h\color{green}\bf32}.
\underbrace{
\frac{(t-1)(t-3)}{(2-1)(2-3)}
}_{
\footnotesize
\t{t=1}{t=2}{t=3},010.
}
+
\t{\e\color{white}\bf243}.
\underbrace{
\frac{(t-1)(t-2)}{(3-1)(3-2)}
}_{
\footnotesize
\t{t=1}{t=2}{t=3},001.
}
$$


Then we can write
$$
\tiny
t^5 = Q(t)(t-1)(t-2)(t-3) 
$$$$+ \frac{(t-2)(t-3)}{(1-2)(1-3)}
+
32
\frac{(t-1)(t-3)}{(2-1)(2-3)}
+
243
\frac{(t-1)(t-2)}{(3-1)(3-2)}
$$
for some polyonimal $Q(t)$.
\a\aa
Let us write things more straightforward for comfortable.

$$g(t)=t^5$$ and
$$
h(t) = \frac{(t-2)(t-3)}{(1-2)(1-3)}
+
32
\frac{(t-1)(t-3)}{(2-1)(2-3)}
+
243
\frac{(t-1)(t-2)}{(3-1)(3-2)}
$$$$
=90t^2 - 239 t + 150.
$$
In fact
$$
t^5 = \underbrace{(t^2+6t+25)}_{Q(t)}(t-1)(t-2)(t-3) + \underbrace{90t^2-239t+150}_{h(t)}
$$
\a\aa
\exe Suppose $A$ is a matrix with 
$$(A-I)(A-2I)(A-3I)=0,$$
 write $A^5$ as a linear combination of $I,A,A^2$.

\vfill

Hint: use the Lagurange interpolation polynomial
$$
t^5 = \underbrace{(t^2+6t+25)}_{Q(t)}(t-1)(t-2)(t-3) + \underbrace{90t^2-239t+150}_{h(t)}
$$

\a\aa

Answer: $A^5 = 90A^2 - 239 A +150$

\a\aa


\[defi]{We say a polynomial $f(t)$ is of \x{simple roots} if one can write
$$
f(t) = (t-x_1)\cdots(t-x_k)
$$
for \x{distinct} $x_1,...,x_k$.
}
Here distinct means $x_i ≠ x_j$ for any $i≠ j$.
\[defi]{We say a matrix $A$ \x{satisfies a polynomial of simple roots} if $A$ satiesfies
$$
(A-x_1I)\cdots(A-x_mI)=0
$$
for some \x{distinct} $x_1,...,x_m$.
}
\a\aa

\[summ]{Suppose $t_1,...,t_m$ are \x{distinct} poimts, amd $A$ satisfies a polynomial with simple roots
$$
(A-λ_1I)(A-λ_2I)\cdots(A-λ_mI)=0.
$$
Let $g(t)$ and $h(t)$ be any polynomials, then
$$
\[cases]{
g(λ_1)=h(λ_1)\\
g(λ_2)=h(λ_2)\\
\cdots\\
g(λ_m)=h(λ_m)\\
}
⟹
g(t)-h(t)=Q(t)(t-λ_1)\cdots(t-λ_m)
$$
$$
t=A ⟹   g(A)-h(A)=0 ⟹  g(A)=h(A)
$$
So $g(A)$ only depends on the value $g(x_1),\cdots,g(x_n)$.
}



%On the other hand, we may also write 
%$$
%t^5 = \underbrace{(t^2+6t+25)}_{Q(t)}(t-1)(t-2)(t-3) + \underbrace{90t^2-239t+150}_{h(t)}
%$$
%as
%$$
%\frac{t^5}{(t-1)(t-2)(t-3)} = t^2+6t+25 + \frac{\color{red}{90t^2-239t+150}}{(t-1)(t-2)(t-3)}
%$$

%% Recall that given distinct points, x1,...,xn, the interpotation polynomial are n many polynomials f_i where each fi only equal to 1 at xi and 0 else in the set. Note this do not have any requirement for the value of the polynomial outside. 

%% For any g(x) and given interpolation polys at x1,...xn, the construction

%% g(x_1)f_1(x) + ... + g(x_n)f_n(x)

%% gives a polynomial that have the same value with g at x1,...,xn, so the differnce is a polynomial with x1,...xn as roots, so must be divisible by a factor, with the quotient written by Q. 

%%% Suppose A is a matrix that satisfies some poly with non-repeated root, then we can create a formula for A^n using this. Scarily.


%%% Emphasis the point that f(A) only determined by the value of f at x1,...xn, and not related to other values!!!!!! 

%%% Emphasis that the formula is not only true for polynomials, but also for rational functions!!!

%% Then paste a lot of exercises.

%% Deduce that all those are projections, ask student a question, how many different projection can one create using polynomials of A? 




%% Review 
\aaa
