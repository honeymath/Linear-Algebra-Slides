

\aaa{Sufficient linearly independent vectors give a basis}
\begin{prop}
If $V$ is a vector space of $“dim“(V)=n$, then any $n$ many \li vectors is a \bas.
\end{prop}
Let 
$$B=\m{\vec v_1}\cdots{\vec v_n}.$$
 be the $n × n$ matrix collecting those \li vectors. Then $B$ has a {\color{red}left inverse} $AB=I_n$. Since $B$ is a square matrix, $BA=I_n$
\vfill
But this means $B$ also a {\color{blue}right inverse}, then columns of $B$ \sws, which means it is a \bas. 
\a\aa
The following is application of above theorem
\begin{cor}
Suppose $W ⊆ V$ and $“dim“(W) = “dim“(V)$, then $W=V$.
\end{cor}

A \bas $\m{\vec e_1}\cdots{\vec e_n}.$ of $W$ is \li in $V$, but since there are $“dim“(V)$-many of them, it is a \bas of $V$. So
$$
V = “span“ \m{\vec e_1}\cdots{\vec e_n}. = W
$$

\aaa


\aaa{Extension of linearly indepdent basis}

Another application is extension of \li to a \bas

\a\aa
$B$ linearly independent, $B$ has left inverse $AB = I_m$.

$BA$ is projection matrix, then $I_n-BA$ is also proejction matrix.

Use cross filling, we may write 
$$
I_n-BA = DC ␣  CD = I_?
$$
how to determine the size of $CD$? 
$$
“tr“(CD) = “tr“(DC) = “tr“(I_n-BA)=n-“tr“(BA)=n-“tr“(AB)= n-m
$$
So $CD = I_{n-m}$.

\a\aa
Then 
$$
\m A,C.\m BD.  = \m{AB}{AD},{CB}{CD}.
$$


$$(BA)²=BA ⟺   BA(I_n-BA)=0 ⟺  BADC = 0 ⟺   AD = 0 $$
$$(BA)²=BA ⟺  (I_n-BA)BA=0 ⟺  DCBA = 0 ⟺   CB = 0 $$


$$
\m A,C.\m BD.  = \m{I_m}{0},{0}{I_{n-m}}. = I_n.
$$
\a\aa

\begin{prop}
Any linearly indepdent list of vectors can be extended to a \bas
\end{prop}

\a\aa
\begin{prop}
If $W ⊆ V$, then $“dim“(W) ≤ “dim“(V)$
\end{prop}
\aaa
