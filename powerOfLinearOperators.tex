
\newcommand\map[3]{#1:{#2}\longrightarrow  {#3}}
\newcommand\maps[5]{{#1}:{#2}\longrightarrow {#3},{#4} \mapsto {#5}}
\newcommand\ZZ{\mathbb{Z}}

\aaa{Applying Polynomial on Linear Operators}
Remember the definition of linear operators
\[defi]{
A linear operator $\map TVV$ is a \lt with domain identical to the codomain.
}
\a\aa

Since domain and codomain are identical, To obtain a matrix representation, we only need to take a basis in $V$. 

$$
T\obasis\ee n=\obasis\ee n M
$$

Onece we have another basis 
$$
\obasis\vv n=\obasis\ee n P
$$
We obtain
$$
T\obasis\vv n = \obasis\vv nP^{-1}MP
$$
\a\aa
\[defi]{We say two matrices $A,B$ similar to each other if there is an \inve matrix $P$ such that 
$$
B= P^{-1}AP
$$}

Similar matrices essentially could be a matrix representation for the same linear transformation. 

If a property is essentially defined for linear transformation, they shares for similar matrices.

\aaa

\aaa{Polynomials on Linear Operators}
Now we only consider one Linear Operator $\map TVV$. By
$$
T^n:=\underbrace{T\circ T\circ\cdots\circ T}_{n\text{ many }T}
$$
we define many new operators on $V$. 
We can also make linear combinations of them to define things like
$$
T^2+2T+\id_V
$$
You found it is exactly equals to
$$
(T+\id_V)^2
$$
For our convenience, we abbreviate $\id_V$ as $I$.

\a\aa
This motivates us we can apply any degree polynomial
$$
p(x)=a_mx^m+a_{m-1} x^{m-1}+\cdots+a_0
$$
to the operator $T$ by evaluating $P$ at $x=T$ by defining
$$
p(T):=a_mT^m+a_{m-1} T^{m-1}+\cdots+a_0
$$
\[cor]{Suppose $\mathcal E\subset V$ is a basis, we have
$$
[p(T)]^{\mathcal E}=p([T]^{\mathcal E})
$$
}
i.e. The matrix of applying polynomial is applying polynomial on the matrix of the linear operator.
\a\aa
\exe Suppose $T$ is a linear operator on $V$ such that 
$$
(T-I)^2=0.
$$
Show that $T$ is invertible.
\vfill
\textbf{Proof.}
We have $T^2-2T+I=0$, therefore $-T^2+2T=I$, so $T(2-T)=I$. This implies $T$ is invertible and 
$$
T^{-1}=2-T.
$$

\aaa

\aaa{Power series on Nilpotent Operators}
\[defi]{A linear operator $\map TVV$ is \x{nilpotent} if $T^n=0$ for some $n\in\ZZ_+$.}
At this time we should be careful for power series because it might cause convergence problem. But for a some operators $T$ that
$$
T^n=0\text{ for some }n\in\ZZ_+
$$
Since $T^n=0\implies T^{n+1}=0$. Apply power series to it would only result finitely many non-zero terms. Therefore it is the same as applyting a Polynomial to it.
\a\aa
\exe Let $V=\lPP xabc$, define $T(f(x))=f'(x)$ Consider the power series
$$
\mathrm{exp}(-x)=1-x+\frac{x^2}2-\frac{x^3}6+\cdots
$$
Evaluate $\mathrm{exp}(-T)(x^2)$
\vfill
\sol We have
$$
\[cases]{
I(x^2)=x^2.\\
T(x^2)=2x.\\
T^2(x^2)=2.\\
T^k(x^2)=0. \text{ for }k\geq 3.}
$$
Therefore $$\mathrm{exp}(-T)(x^2)=I(x^2)-T(x^2)+\frac{T^2(x^2)}2=x^2-2x+1=(x-1)^2.$$

\a\aa
Suppose $T$ is a linear operator on $V$ such that
$$
T^3=0.
$$
Show that $I-T$ is invertible.
\vfill
\textbf{Idea:} Let $p(x)=1-x$, we realize $I-T=p(T)$. We want to find it inverse so the idea is to apply $\frac1{p(x)}$ on it, which has the Talor Expansion
$$
\frac1{p(x)}=\frac1{1-x}=1+x+x^2+x^3+\cdots
$$
Since $T^3=0$, so we think
$$
\frac1{p(T)}=1+T+T^2+0+0+\cdots=1+T+T^2.
$$
So the idea is try $1+T+T^2$.

\a\aa
\textbf{Proof:} Since
$$(I-T)(I+T+T^2)=(I+T+T^2)-T(I+T+T^2)=I-T^4=I.$$
This implies $I-T$ is invertible and its inverse is given by $I+T+T^2$.
\vfill
\aaa

