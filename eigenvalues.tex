
\aaa{Trace and determinant in terms of eigenvalues}
\[defi]{An \x{eigenvalue} $λ$ of matrix $A$ is a value with 
$$
“det“(λI-A)=0.
$$
 Suppose one can factorize characteristic polynomial into
$$
“det“(tI-A)=(t-λ_1)(t-λ_2)\cdots(t-λ_n).
$$
Then we call the list $λ_1,λ_2,...,λ_n$ the list of its eigenvalues. (Note, the element can be repeated in the list.)
}
\a\aa
\exe Suppose matrix $A$ has 
$$
“det“(tI-A) = (t-1)^3(t-2)^2(t-3)^5
$$
what is a list of its eigenvalues?

\sol
$$
1,1,1,2,2,3,3,3,3,3.
$$
\a\aa
\[prop]{If $“det“(tI-A)=“det“(tI-B)$, then the sum of \x{principal minors} of size $k$ in $A$ and in $B$ are the same, because they are all given by the coefficient of $(-1)^{n-k}t^k$.}
Do you remember the definition of principal minor?? they are submatrices with diagonal on diagonal of its father...
\def\q{{\h*}}
$$
\m\q\q*,\q\q*,***.
␣
\m***,*\q\q,*\q\q.
␣
\m\q*\q,***,\q*\q.
$$

\a\aa
Therefore suppose $A$ is a matrix with
$$
“det“(tI-A) = (t-1)^2(t-2)^3
$$
Then we consider a diagonal matrix with the same chacracteristic polynomial with it, which can be written
$$
Λ = \m1{}{}{}{},{}1{}{}{},{}{}2{}{},{}{}{}2{},{}{}{}{}2.
$$
\a\aa
$$
\det(tI-Λ)
$$
$$=\m{t-1}{}{}{}{},{}{t-1}{}{}{},{}{}{t-2}{}{},{}{}{}{t-2}{},{}{}{}{}{t-2}.
=(t-1)(t-1)(t-2)(t-2)(t-2)
$$
\x{Note, $Λ$ has nothing to do with $A$ except sharing the same characteristic polynomial with $A$.}
Then we have 
$$“sum of principal minor of size 1“=“tr“(A)=“tr“(Λ)=“sum of eigenvalues of “A$$
$$“sum of principal minor of size n“=“det“(A)=“det“(Λ)=“product of eigenvalues “A$$
\a\aa
\[prop]{ n × n matrix has n eigenvalues. The trace of a matrix, equals to the sum of all its eigenvalues, the product of a matrix, equals to the product of all its eigenvalues.}

\a\aa
\exe Given one eigenvalue of the following matrix, are you able to find the other eigenvalue? and the determinant?
$$ \m11,{-2}4. ␣ “Eigenvalue : 2, \_\_“ , ␣ “det“:$$
$$ \m51,{1}5. ␣ “Eigenvalue : 4, \_\_“ ␣ “det“: $$
$$ \m50,{1}5. ␣ “Eigenvalue : 5, \_\_“ ␣ “det“: $$
%{The Vieta's formula}
%Let $f(x)$ be a polynomial
%$$
%f(t) = ∏_{i=0}^n(t-x_i)
%$$
%Then 
%$$
%f(t) = t^n - a_1t^{n-1}+a_2t^{n-2}-\cdots+(-1)^n a_n
%$$
%Here
%$a_i$ is the sum of all possible product of i many elements from the tuple $(x_1,...,x_i)$
%\a\aa
%For example, 
%$$
%(t-λ_1)(t-λ_2)(t-λ_3) = t^3 - (λ_1+λ_2+λ_3)t^2 + (λ_1λ_2+λ_2λ_3+λ_3λ_1)t - (λ_1λ_2λ_3).
%$$



%\a\aa
%\[prop]{ For any matrix $A$, it has $n$ many eigenvalues (multiplicity counted), and the sum of eigenvalues is the trace of $A$, the product of those eigenvalues is the determinant of $A$. 
%}
\aaa

%% trying to prove spectural theorem, how?

%% f(t)-f(A)/t-A  is a polynomial of t, det(f(t)-f(A)) is divisible by t-A. 

%% k-f(t) / 
