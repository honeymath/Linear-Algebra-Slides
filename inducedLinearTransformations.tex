
\newcommand\map[3]{#1 : {#2}⟶  {#3}}
\newcommand\maps[5]{{#1}:{#2}⟶  {#3}, {#4} ↦ {#5}}


\aaa{Induced Linear Transformation}

From now, we have learned three concepts

\[itemize]{
\item Matrices
\item Tuple of vectors
\item Linear Transformation
}

In order to link all those three concepts together, we would like to bring all object in one world: Linear Transformation.
\vfill

In the next part. We will give the way to translate each object as a linear transformation. 

\[itemize]{
\item $m\times n$ Matrices $\iff$ Linear transformations $F^n\longrightarrow F^m$
\item n Tuple of vectors $\iff$ Linear transformations $F^n\longrightarrow V$
}
\a\aa
There is also a way to translate subspaces into Linear Transformations. We would like to keep that part later because we would like you to be familiar with it first.
\vfill

After those translation. All linear algebra are essentially studying linear transformations. The goal for this part is to put the linear transformation into a central point of all this subject.

\a\aa

After we translate a tuple or a matrix into a linear transformation. We call it \x{induced linear transformation}.\footnote{This concept is borrowed from Prof. Jason Siefken's Book}

\vfill

Before talking about this, let's continue our story of making meals.

\a\aa
Afterwards, Shinchan start using pictures to represents everything. Until one day, an old Chef complained.

\vspace{0.5cm}

\centerline{Old Chef: {\it I want a clear number table! The recipe is not a doodle!}}

\[columns]{
\co5
\[zzz]{
\org
\grid{-4}{-4}44
\draw (1,0) node {\sleaf} (0,1) node {\sbean};
\draw (1,2) node {\scola} (1,1) node {\stea};
	\pgftransformcm{1}{2}{1}{1}{\pgfpoint{0}{0}}
\grid[red!60!white!40]{-2}{-2}22
\draw (1,1) node {\sbento} (1,2) node {\ssoup};
	}
\co5

Old Chef wants

\t{}\bento\soup,\cola11,\tea12.
}
\a\aa

But still Shinchan wants to keep pictures, so he just cut the table and labeled it to each corresponding meals.
\[columns]{\co6
\[zz]{
\org
\grid{-4}{-4}44
\draw (1,0) node {\sleaf} (0,1) node {\sbean};
\draw (1,2) node {\scola} (1,1) node {\stea};
	\pgftransformcm{1}{2}{1}{1}{\pgfpoint{0}{0}}
\grid[red!60!white!40]{-2}{-2}12
\draw (1,1) node {\sbento} node[right]{\tiny$\cc11$} (1,2) node {\ssoup} node[right]{\tiny$\cc12$};
	}
\co4
He did not put the row header \scola,\stea on it. And those label depends on what it the row header.

\vspace{1cm}

To prevent the old chief complaining, he will label all future meals. 
}

\a\aa

Call the combination space of meals as $V$. Let $\ce=\rr\scola\stea$ be the tuple of materials in the row header. The way Shinchan label the meals, is a map


$$
\maps {L_\ce}{F^2}V{\cc xy}{x\scola+y\stea}
$$
\vfill
By the meaning of matrix multiplication, this map can also be written as


$$
\maps {L_\ce}{F^2}V{\cc xy}{\rr{\scola}{\stea}\cc xy}
$$

In other words, this is just the map of left multiplication by $\rr\scola\stea$


\a{Viewing tuple of vectors as a linear transformation}

\[defi]{Let $\ce=\rrrr{\vv_1}{\vv_2}\cdots{\vv_n}$ be n-tuples of vectors in $V$, the \x{induced linear transformation} of $\ce$ is a linear transformation $L_\ce$ with the domain $F^n$, codomain $V$, by the rules of assiging coefficient list to corresponding linear combinations of $\ce$. Explicitely,
$$
\maps {L_\ce}{F^n}V{\cccc{a_1}{a_2}\vdots{a_n}}{a_1\vv_1+a_2\vv_2+\cdots+a_n\vv_n}.
$$
}
\a\aa
Note that the linear combination can be written as a matrix multiplication.

$$
a_1\vv_1+a_2\vv_2+\cdots+a_n\vv_n=\obasis\vv n\coro an
$$
\a\aa
Let $\ce=\rrrr{\vv_1}{\vv_2}\cdots{\vv_n}$, the induced linear transformation $L_\ce$ is the same as \x{left multiplying } the list of vectors $\rrrr{\vv_1}{\vv_2}\cdots{\vv_n}$ as a matrix. 

$$
\maps {L_\ce}{F^n}V{\cccc{a_1}{a_2}\vdots{a_n}}{{\obasis\vv n\coro an}}.
$$

The letter L refers to the {\it \x{L}eft multiplication}
\a\aa
The induced transformation is defined for \x{any finite tuple of vectors.} The induced transformation may map two coefficient list to the same element. For example,  
\[columns]{\co6
\[zz]{
\org
\grid{-4}{-4}44
\draw (1,0) node {\sleaf} (0,1) node {\sbean};
\draw (1,2) node {\scola} (1,1) node {\stea};
	\pgftransformcm{1}{2}{1}{1}{\pgfpoint{0}{0}}
%\grid[red!60!white!40]{-2}{-2}12
	}
\co4
We consider the induced transformation $\map{L_\ce}{F^3}V$ with $\ce = \rrr\sbean\sleaf\stea$, then 
$$
\stea=0\sbean+0\sleaf+1\stea
$$
$$\stea=1\sbean+1\sleaf+0\stea
$$

Both coefficients list maps to the same vector
$$
L_\ce\ccc001=L_\ce\ccc110 = \tea
$$
}
\a\aa

Some element in the codomain may not be associated with a coefficient list, for example
\[columns]{\co6
\[zz]{
\org
\grid{-4}{-4}44
\draw (1,0) node {\sleaf} (0,1) node {\sbean};
\draw (1,2) node {\scola} node[right]{$(1)$} (1,1) node {\stea};
\draw (2,4) node {\small $(2)$};
\draw (-1,-2) node {\small $(-1)$};
\draw (-2,-4) node {\small $(-2)$};
\draw[thick,blue] (-2.5,-5)--(2.5,5);
	\pgftransformcm{1}{2}{1}{1}{\pgfpoint{0}{0}}
%\grid[red!60!white!40]{-2}{-2}12
	}
\co4
We consider the induced transformation $$\map{L_\ce}{F}V$$ with $\ce = (\scola)$. 
\vspace{0.5cm}

The image of this map is as shown. Clear, \sleaf can not be writen as any kinds of linear combination of \scola, so $\sleaf$ do not have an coefficient list corresponds to it.
}


\a\aa

\[summ]{A tuple of vectors $\ce = \obasis\vv m$ in $V$ can be viewed as a linear transformation.
$$
\map{L_\ce}{F^m}{V}
$$

Any linear transformation $\map T{F^m}V$ must be an induced linear transformation by an m-tuples in $V$. This tuple is given by
$$
\m{T\cccc10\vdots0}{T\cccc01\vdots0}\cdots{T\cccc00\vdots1}.
$$

.
	}

\aaa




\aaa{Viewing matrices as linear transformations}




In the previous part we have understand how m tuples corresponds to linear transformations from domain $F^m$. Previously we give the link between tuple of vectors and linear transformations.
\vfill
$$
\xymatrix{
	\text{Tuple of vectors}\ar[rr]{}{}&& \text{Linear Transformations}\ar[ll]{}{}\\
	&\text{Matrices}&\\
	}
$$
\vfill
Now we will draw our road map to matrices.

\a{Viewing matrices as tuples}

If we look at a $n\times m$ matrix column by column. Each column is representing an element in $F^n$.
$$
{\m 12,23,34.} \iff \m{\ccc123}{\ccc234}.
$$

Therefore, $n\times m$ matrices are equivalent to m-tuples in $F^n$. This gives a interpretation of matrices in tuples.

$$
\xymatrix{
	\text{Tuple of vectors}\ar[rd]{}{}\ar[rr]{}{}&& \text{Linear Transformations}\ar[ll]{}{}\\
	&\text{Matrices}\ar[ul]{}{}&\\
}
$$
	





\a\aa

Since matrices can be viewed as tuples, $n\times m$ matrix is equivalent to choosen $m$ vectors in $F^n$. This tuple induces a map of
$$
\map {L_\ce}{F^m}{F^n}
$$
This map is obtained by left multiplying a matrix with columns by those selected vectors $\ce$ in $F^n$. Since this is an actual matrix( every entry of this matrix are numbers $\in F$ ). This is called \x{the induced linear transformation by a matrix}.

This gives a interpretation of matrices in linear transformations.

$$
\xymatrix{
	\text{Tuple of vectors}\ar[rd]{}{}\ar[rr]{}{}&& \text{Linear Transformations}\ar[ll]{}{}\ar[ld]{}{}\\
	&\text{Matrices}\ar[ul]{}{}\ar[ur]{}{}&\\
}
$$

This viewpoint tell us we can view a matrix as a linear transformation from $F^m$ to $F^n$.

\a\aa



Now we replace the abstract vector space $V$ by $F^n$. This implies we can natrually understand m-tuples in $F^n$ as a linear transformation.

\vfill

\a\aa
\exe Select two vectors in $F^3$ as a 2-tuple as follows
$$
\ce = \left(\ccc120,\ccc101\right)
$$
Write down the induced linear tranformation.
\vfill
\sol
$$
\maps{L_\ce}{F^2}{F^3}{\cc ab}{a\ccc120+b\ccc101}
$$
\vfill
One can also write this as
$$
\cc ab\mapsto \m11,20,01.\cc ab.
$$

\a\aa
Conversely, \x{any linear tranformation} $\map{}{F^m}{F^n}$ \x{is induced by a matrix}. By looking at the image of natural basis. 
\vfill
\exe Consider a linear transformation $\map T{F^2}{F^3}$ with
$$
T\cc 11 = \ccc132,\qquad T\cc12=\ccc 135
$$
Find a tuple of vectors $\ce$, with $T=L_\ce$.

\a\aa
\sol We only need to know $T\cc10$ and $T\cc01$. Those are easy
$$
T\cc10=T\cc22-T\cc12=\ccc264-\ccc135=\ccc13{-1}
$$
$$
T\cc 01 = T\cc12-T\cc11 = \ccc003
$$
This implies $T$ is a linear tranformation induced by 
$$
\m 10,30,{-1}3.
$$

\a\aa
Now let's summarise the induced linear transformation.
\a\aa
For any $n\times m$ matrix $P$, its \idu\lt is defined by
$$
\maps{L_P}{F^m}{F^n}{{\coro am}}{P{\coro am}}
$$

Remember here $P$ is a \x{Matrix}, $L_P$ is a \x{Linear Transformation}, $L_P$ and $P$ has the exactly the same information, but different meaning. $L_P$ means how we use $P$ to creat a map.(by left multiplication)


{\color{gray}\exe{What's the differene between {\it you} as a person and {\it you eat} as an action?}}

\a\aa

For any $m$-tuple of vectors $\cv=\obasis\vv m$ in $V$, its \idu\lt is defined by
$$
\maps{L_\cv}{F^m}{V}{{\coro am}}{{\obasis\vv m\coro am}}
$$

Remember here $\obasis\vv m$ is a \x{m-tuple of vectors}, $L_\cv$ is a \x{Linear Transformation}, $L_\cv$ and $\cv$ has the exactly the same information, but different meaning. $L_\cv$ means how we use $\cv$ to creat a map. (by left multiplication)

\aaa


\aaa{Composition of induced transformation}

From very begining we mentioned that we could write
$$
\left\{\[array]{{ll}
3\ee_1+\ee_2&=\ww_1\\
\ee_1+\ee_2&=\ww_2
}\right.
$$

as
$$
\obaba\ww=\obaba\ee \m31,11.
$$
\vfill

The reason of writing this way can be stated in the viewpoint of induced linear transformation.

\a\aa

Label each part with a symbol.

$$
\underbrace{\obaba\ww}_{\cw}=\underbrace{\obaba\ee}_{\ce}\underbrace{ \m31,11.}_A
$$

We have associated induced linear transformation $L_\cw,L_\ce,L_A$.
\vfill
The order of our symbol is logical in the sense that we have
$$
L_\cw = L_\ce\circ L_A
$$
\a\aa

More generally, in this part, you would find all our symbol make sense because it is representing its meaning of induced transformations
\vfill
We will prove the following equations in this part
\t{Regular Symbol}{Translate to Induced Transformation},
{$
\underbrace{\obaba\ww}_{\cw}=\underbrace{\obaba\ee}_{\ce}\underbrace{ \m31,11.}_A
$}{$
L_\cw = L_\ce\circ L_A
$
},
{$
T\underbrace{\obaba\ww}_{\cw}=\underbrace{\obaba\ee}_{\ce}
$}{$
T\circ L_\cw = L_\ce
$
},
{$
C=AB
$}{$
L_C = L_A\circ L_B
$
}.
\a\aa
\[lem]{For any tuple $\cw=\obasis\ww n$ and $\ce=\obasis\ee m$in $V$, suppose
$$
\underbrace{\obasis\ww n}_\cw=\underbrace{\obasis\ee m}_{\ce}A
$$
Then 
$$
L_{\cw}= L_\ce\circ L_A
$$
The following picture shows domain and codomain for each map.
$$
\xymatrix{
	F^n\ar[r]^{L_A}\ar[rr]_{L_\cw}&F^m\ar[r]^{L_\ce}&V
	}
$$
}
\a\aa
\textbf{Proof:} We only need to prove $L_{\cw}(\vv)= L_\ce\circ L_A(\vv)$ for any $\vv\in F^n$. To see this, we see

$$
L_\ce\circ L_A(\vv) = L_\ce(L_A(\vv))=L_\ce(A\vv)=\underbrace{\obasis\ee m}_\ce A\vv
$$
Note that this equals to
$$
\underbrace{\obasis\ww n}_\cw\vv = L_\cw(\vv).
$$
Therefore we proved this Lemma.






\a\aa
\[lem]{For any tuple $\ce=\obasis\vv n$ in $V$, any \lt $\map TVW$. Let $\cw = T\ce$ be the corresponding output in $W$. In other words,
$$
\underbrace{\obasis\ww n}_{\cw} = T\underbrace{\obasis\vv n}_{\ce}
$$
Then 
$$
L_{\cw}=T\circ L_\ce
$$
The following picture shows domain and codomain for each map.
$$
\xymatrix{
	F^n\ar[r]^{{L_\ce}}\ar[rr]_{L_\cw}&V\ar[r]^{T}&W
	}
$$


}
\a\aa
\[proof]{By definition, we have
$$
\underbrace{\obasis\ww n}_{\cw} = \obasis{T\vv} n
$$
To prove $L_{\cw}=T\circ L_\ce$, we only need to show $L_{\cw}(x)=T\circ L_\ce(x)$ for any
$$x=\coo an\in F^n,$$ Clear
$$
L_{\cw}(x)=a_1T\vv_1+\cdots+a_nT\vv_n=T(a_1\vv_1+\cdots+a_n\vv_n)=T(L_\ce(x))=T\circ L_\ce(x).
$$

Therefore $L_\cw = T\circ L_\ce$.

}
\a\aa


\[lem]{
Let $A$ be an $n\times m$ matrix, $B$ an $m\times p$ matrix, let $C$ be the product of them
$$
C=AB
$$
Then we have
$$
L_C=L_A\circ L_B
$$
The following picture shows domain and codomain for each map.
$$
\xymatrix{
	F^p\ar[r]^{L_B}\ar[rr]_{L_C}&F^m\ar[r]^{L_A}&F^n
	}
$$

}
\a\aa

\[proof]{We only need to show for any $x\in F^p$, we have
$$
L_C(x)=L_A\circ L_B(x).
$$
To do so, we see
$$
L_C(x)=Cx=ABx=L_A(Bx)=L_A(L_B(x))=L_A\circ L_B(x).
$$

}

\a{Invertible matrix and its induced transformation}

In the world of matrices, we have identity matrix and invertible matrices.

\vfill

In the world of \lt, we have identity \lt and invertible \lt.

\vfill

In this part, we demonstrate the concept of being identity and invertible in matrix world is a kind of translation of the same concept from \lt world. 
\a\aa

\[lem]{Let $I_n$ be an $n\times n$ identity matrix, then
$$
L_{I_n}=\id_{F^n}
$$
The following picture shows domain and codomain for each map.
$$
\xymatrix{
	F^n\ar[r]^{L_{I_n}}&F^n
	}
$$


}
\[proof]{
	For any $x\in F^n$, $L_{I_n}(x)=I_nx=x$. So $L_{I_n}=\id_{F^n}$.
	}
\a\aa

\[lem]{Let $A$ be an $n\times n$ invertible matrix and let $A^{-1}$ be its inverse, then $L_A$ is an invertible linear transformation and its inverse is $L_{A^{-1}}$.

The following picture shows domain and codomain for each map.
$$
\xymatrix{
	F^n\ar[rr]^{L_{A}}&&F^n\ar[ll]^{L_{A^{-1}}}
	}
$$


	}
\[proof]{We have $L_A\circ L_{A^{-1}}=L_{AA^{-1}}=L_{I_n}=\id_{F^n}$, and $L_{A^{-1}}\circ L_A=L_{A^{-1}A}=L_{I_n}=\id_{F^n}$. This implies the lemma.}


\aaa









\aaa{Basis and its induced transformation}
In the world of tuples in $V$, we have discussed how to verify certain tuples are \li, \sws, or being a \bas. 
\vfill

In this part, we will show being a \bas translates to the property of being \inve in \lt world.

\vfill

The inverse of the \idu\lt of a \bas is called \x{the coordinate map}.


\a\aa

\[lem]{A tuple $\ce=\obasis\ee n$ in $V$ is a \bas if and only if its \idu\lt $L_\ce$ is \inve.


\vspace{1cm}
The following picture shows domain and codomain for each map.
$$
\xymatrix{
	F^n\ar[rr]^{L_\ce}&&V
	}
$$

}

We will prove this lemma by directly giving its inverse. 



\a\aa

\[prop]{For a basis $\ce=\obasis\ee n$, let $L_\ce$ be its \idu\lt, then
$$
[\vv]^\ce = L_\ce^{-1}(\vv)
$$

In other words, the \lt $$\xymatrix{V\ar[r]^{L_\ce^{-1}}&F^n}$$ maps every vector to its coordinate. We call the map $L_\ce^{-1}$ the \x{coordinate map} of the \bas $\ce$. %(footnote)\footnote{We also write $L_\ce^{-1}$ as $L^\ce$ for index cancelation purposes(see later).}
	}
\a\aa
\textbf{Proof}Consider the following map
$$
\maps{R}{V}{F^n}{\vv}{[\vv]^\ce}
$$
We will prove this map is the inverse of $L_\ce$. For any $\vv\in V$
$$
L_\ce\circ R(\vv) = L_\ce(R(\vv))=L_\ce([\vv]^\ce) = \underbrace{\obasis\ee n}_\ce [\vv]^\ce = \vv.
$$

\a\aa
\textbf{Continue from previous}

And for any coefficient list $$x=\coo xn\in F^n,$$ we have
$$
R\circ L_\ce(x) = R(x_1\ee_1+\cdots+x_n\ee_n)=\coo xn = x
$$

This implies the coordinate map $R$ is the inverse of the linear transformation $L_\ce$.

\a\aa
\[prop]{
Let $\ce=\obasis\ee n$ be a basis of $V$. For any vectors $\vv,\ww\in V$, we have
$$
[\lambda\vv+\mu\ww]^\ce = \lambda[\vv]^\ce+\mu[\ww]^\ce
$$

}
\[proof]{Since the inverse of a \lt must be a \lt. We know $L_\ce^{-1}$ is a \lt. Therefore
$$
[\lambda\vv+\mu\ww]^\ce = L_\ce^{-1}(\lambda\vv+\mu\ww) =  \lambda L_\ce^{-1}(\vv)+\mu L_\ce^{-1}(\ww) = \lambda[\vv]^\ce+\mu[\ww]^\ce
$$
	}








\aaa
