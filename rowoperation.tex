
\subtitle{Column/Row operations}\maketitle
\aaa{Learning Objectives}
\begin{tikzpicture}[scale=0.8,transform shape, block/.style={draw, rectangle, minimum width=2cm, minimum height=1cm}]
\node[block,fill=green] (block1) {Matrix Multiplication(1.4)};
    \node[block, right=1cm of block1.east, anchor=west] (block2) {Row/Column Operations(1.3)};
%    \node[block, right=0cm of block2.east, anchor=west] (block3) {Cross Filling method(1.5)};
\draw[->] (block1.east) -- ([xshift=-0.2cm]block2.west);
\end{tikzpicture}

\begin{itemize}
\item What \x{three} kind of row/column operations are there?
\item For the product $AB=C$, 
	\begin{itemize}
		\item Which two matrices is allowed to play \x{simutaneous row operation} without changing the equality?
		\item Which two matrices is allowed to play \x{simutaneous column operation} without changing the equality?
	\end{itemize}
\item Will simutaneous (row/column) operation changes $A^{-1}B$ or $AB^{-1}$?
\item Let $E$ be a matrix obtained by applying some row operations from $I$, what does $EA$ mean?
\item Let $E$ be a matrix obtained by applying some column operations from $I$, what does $AE$ mean?
\item Why matrix equation $Ax=b$ is a system of linear equations?
\item How row operation help with solving system of linear equations? \begin{itemize}\item What kind of matrix are we reducing to?\end{itemize}
\item How to use row/column opetations to find \x{inverse}?
\end{itemize}

\x{Explain row/column operation to elementary school students!}
\a{A special request}
A custormer requests for a special \textbf{new drink } need the following ingradients
\vfill
\[columns]{
\co5 \textbf{Old Drinks ingradients:}\t{}\milk\soup\coffee\tea,\leaf0202,\lemon0101,\bean0420,\cow1100.
\co3 \textbf{New Drink requirement: }\t{}\cola,\leaf4,\lemon2,\bean2,\cow4.
\co2\textbf{Problem:}\t{}\cola,\milk?,\soup?,\coffee?,\tea?.
}
But the chef only have \milk,\soup,\coffee,\tea at the hand, can he produce \cola by those materials?
\a\aa
The chef thought this problem is the same as a matrix product equation, indeed, replace those questionmarks by $x,y,z$, he need the following equation to be true
\[columns]{
\co5 \textbf{Old Drinks ingradients:}\t{}\milk\soup\coffee\tea,\leaf0202,\lemon0101,\bean0420,\cow1100.$ \quad\times $
\co3\textbf{Problem:}\t{}\cola,\milk x,\soup y,\coffee z,\tea w. $\quad = $
\co2 \textbf{New Drink requirement: }\t{}\cola,\leaf4,\lemon2,\coffee2,\cow4.
}
\a\aa
Mathematically, this equation is writting as 
$$
\m 0202,0101,0420,1100.\m x,y,z,w.=\m4,2,2,4.
$$
\a\aa
By understanding by columns, solving it is the same as asking for

\t4,2,2,4. =\t{0},{0},{0},{1}. \textbf{x} +\t{2},{1},{4},{1}.\textbf{y}+\t{0},{0},{2},{0}. \textbf{z} +\t{2},{1},{0},{0}. \textbf{w}

This can be write as the following and we call it the \textbf{Linear equation}
$$
\begin{cases}
0x+2y+0z+2w=4\\
0x+1y+0z+1w=2\\
0x+4y+2z+0w=2\\
1x+1y+0z+0w=4\\
\end{cases}
$$
\a{Changing materials}
Observation: The question is only asking for the meal's demand for semi-product meals. It does not asking anything related to the raw material.

\t{}\cola,\milk x,\soup y,\coffee z,\tea w.

No \lemon,\leaf,\bean,\cow appeared in this question. Therefore we can change materials to \x{simplify} the problem.

\a{Row reduction}
The clever chef changes the \textbf{material} so the ingradients table is easier

\[columns]{
\co{45}\t{}\smilk\ssoup\scoffee\stea\scola,\leaf02024,\lemon01012,\bean04202,\cow11004.
\co{55}\t{}\smilk\ssoup\scoffee\stea\scola,\cow{\alert1}1004,{\bean\bean}02{\alert1}01,{\lemon+\leaf\leaf}010{\alert1}2,\leaf00000.
}
$$
\begin{cases}
0x+2y+0z+2w=4\\
0x+1y+0z+1w=2\\
0x+4y+2z+0w=2\\
1x+1y+0z+0w=4\\
\end{cases}
\Longrightarrow
\begin{cases}
1x+1y+0z+0w=4\\
0x+2y+1z+0w=1\\
0x+1y+0z+1w=2\\
0x+0y+0z+0w=0\\
\end{cases}
$$
\a{Row Multiplying}
Doubling the material $\bean\mapsto\bean\bean$   will \alert{\textbf{multiply 3rd row by $\frac12$}}. \vfill
\[columns]{
\co{45}\t{}\smilk\ssoup\scoffee\stea\scola,\leaf02024,\lemon01012,\bean04202,\cow11004.
\co{55}\t{}\smilk\ssoup\scoffee\stea\scola,\leaf02024,\lemon01012,{\bean\bean}{\alert{0}}{\alert{2}}{\alert{1}}{\alert{0}}{\alert{1}},\cow11004.
}
$$
\begin{cases}
0x+2y+0z+2w=4\\
0x+1y+0z+1w=2\\
\x0x+\x4y+\x2z+\x0w=\x2\\
1x+1y+0z+0w=4\\
\end{cases}
\Longrightarrow
\begin{cases}
0x+2y+0z+2w=4\\
0x+1y+0z+1w=2\\
\x0x+\x2y+\x1z+\x0w=\x1\\
1x+1y+0z+0w=4\\
\end{cases}
$$
On equation, the third equation has been divided by $2$. This is called \alert{\textbf{row multiplying}}
\a{Row Switching}
\textbf{Switching the order} would not change problem, \vfill
\[columns]{
	\co{5}\t{}\smilk\ssoup\scoffee\stea\scola,\leaf02024,\lemon01012,{\bean\bean}02101,\cow11004.
\co{5}\t{}\smilk\ssoup\scoffee\stea\scola,\cow11004,{\bean\bean}02101,\lemon01012,\leaf02024.
}\vfill
$$
\begin{cases}
{\color{red}0x+2y+0z+2w=4}\\
{\color{purple}0x+1y+0z+1w=2}\\
{\color{black}0x+2y+1z+0w=1}\\
{\color{blue}1x+1y+0z+0w=4}\\
\end{cases}
\Longrightarrow
\begin{cases}
{\color{blue}1x+1y+0z+0w=4}\\
0x+2y+1z+0w=1\\
{\color{purple}0x+1y+0z+1w=2}\\
{\color{red}0x+2y+0z+2w=4}\\
\end{cases}
$$
 This is called \alert{\textbf{row switching}}

\a{Row Adding}
Replacing \lemon by \lemon\leaf\leaf. Each time I use the package \lemon\leaf\leaf I save \leaf\leaf. Then the row for \leaf is reduced by 2 times the row for \lemon.\vfill

\[columns]{
\co 5 \t{}\smilk\ssoup\scoffee\stea\scola,\cow11004,{\bean\bean}02101,\lemon01012,\leaf02024.
\co 5 \t{}\smilk\ssoup\scoffee\stea\scola,\cow11004,{\bean\bean}02101,{\lemon\leaf\leaf}01012,\leaf00000.
}
\vfill
$$
\begin{cases}
1x+1y+0z+0w=4\\
0x+2y+1z+0w=1\\
0x+1y+0z+1w=2\\
0x+2y+0z+2w=4\\
\end{cases}
\Longrightarrow
\begin{cases}
1x+1y+0z+0w=4\\
0x+2y+1z+0w=1\\
0x+1y+0z+1w=2\\
\x0x+\x0y+\x0z+\x0w=\x0\\
\end{cases}
$$
On equation, 4'th equation has been subtracted by 3rd equation. This is called \alert{\textbf{row adding}}
\a{When should row reduction stop?}
In one words, row operation is \textbf{updating the raw ingradient list}. 

We should stop if the matrix is \x{simple enough} for us to solve equations. So what is simple? There are many discussions.

\textbf{Theory 1}: We should stop if 
\begin{itemize}
\item Each non-zero row has an entry $1$, such that this $1$ is the only non-zero entry on its columns.
\end{itemize}
\vfill
\t{}\milk\soup\coffee\tea\cola,\cow{\alert{\textbf1}}1004,{\bean\bean}02{\alert{\textbf1}}01,{\lemon+\leaf\leaf}010{\alert{\textbf1}}2,\leaf00000.
\a\aa
\textbf{Explaination to Theory 1}: If the matrix has been reduced that way, we obtain an equation
$$
\begin{cases}
1\x x+1y+0z+0w=4\\
0x+2y+1\x z+0w=1\\
0x+1y+0z+1\x w=2\\
\end{cases}
$$
such that each equation has a variable, such that this variable does not appear in other equations.
When this happens, we may move these variable to one side of equation. Assigning other variables arbitrary values would end up with a solutoin.
$$
\begin{cases}
x=4-y\\
 z=1-2y\\
 w=2-y\\
\end{cases}
$$
Put $y=0$, then $(x,y,z,w)=(4,0,1,2)$ is a solution.
\a\aa
\textbf{Explaination to Theory 1 without equation}:
A column with a single $1$ and $0$ elsewhere can help us to replace a material with some compunds. For example, 
\vfill
\[columns]{
\co{25}\t{}\coffee,\cow0,{\bean\bean}{\textbf1},{\lemon+\leaf\leaf}0,\leaf0.
\co{05}$\implies$
\co7 \coffee = \textbf0\cow+\textbf{\alert1}\bean\bean+\textbf0(\lemon\leaf\leaf)+\textbf0\leaf
~\\
~\\~\\
which means
\bean\bean \textbf{=} \coffee
}

\a\aa
With this observation, we can replace \textbf{certain materials} by \textbf{certain meals}
\[columns]{
\co{55}
\t{}\smilk\ssoup\scoffee\stea\scola,{\h\cow}{\alert1}1004,{\y\bean\bean}02{\alert1}01,{\e\lemon+\leaf\leaf}010{\alert1}2,\leaf00000.
\co{45}
\t{}\smilk\ssoup\scoffee\stea\scola,{\h\milk}{\alert1}1004,{\y\coffee}02{\alert1}01,{\e\tea}010{\alert1}2,\leaf00000.
}
\a\aa
Note that the leaves \leaf is no longer needed for those packaged materials, we can delete it.
\t{}\milk\soup\coffee\tea\cola,{\h\milk}{\alert1}1004,{\y\coffee}02{\alert1}01,{\e\tea}010{\alert1}2.

\a\aa
which tell us directly the list we want, lets compare the original question\vfill
\[columns]{
\co3 \textbf{Output}
\t{}\cola,{\h\milk}4,{\y\coffee}1,{\e\tea}2.
\co 4\textbf{Original Question}
\t{}\cola,{\milk}x,\soup y,{\coffee}z,{\tea}w.
}\vfill
So $x=4,y=0,z=1,w=2$.(\soup have not been used.)
This process is equivalent as setting $y=0$ to obtain a solution.
$$
\begin{cases}
x&=4\\
y &=0\\
z&=1\\
w&=2\\
\end{cases}
$$
 
\a\aa
\textbf{Theory 2}: We should stop if
\begin{itemize}
\item After deleting zero rows, there are columns that can be rearranged into a triangular matrix with non-zero diagonal.
\end{itemize}
Example:

$$
\m {\y1}{\y2}{\y3}466,
   {\y2}0{\y3}211,
   {\y1}00210,
   000000.
\qquad\longrightarrow
\qquad
\m 231,032,001.
$$
\a\aa
We beifly explain why , firstly, circle the entry that corresponding to the diagonal of the triangular matrix
$$
\m {1}{\y2}{3}466,
   {2}0{\y3}211,
   {\y1}00210,
   000000.
$$
We read equations from bottom to top, one by one
$$
x + 0y +0z +2w+1u = 0 \implies x = -2w-1u.
$$
The circled variable would appear as a new variabl never appeared before
$$
2x + 0y +3z +2w+1u = 1 \implies z = \frac{-2x-2w-u}3.
$$
Then the next new variable is $y$.
$$
x+2y+3z+4u+6w=6\implies y=\frac{6-6w-4u-3z-x}2.
$$ 
Therefore, the equation can be solved. We call this process \x{backwards substituting}.



%
\a{Identity matrix}
Filling the following blanks.
\vfill
\[columns]{
\co4
\t{}\tea\milk\coffee,?100,?010,?001.
\co4
\t{}???,\tea100,\milk010,\coffee001.
%\co4
%\t{}\tea\milk\coffee,\tea???,\milk???,\coffee???.
}
\a\aa
Suppose \milk \coffee \tea , any two can not blend to the third drink (called \li). Filling the following blanks.
\vfill
\t{}\tea\milk\coffee,\tea???,\milk???,\coffee???.
\a\aa
\[columns]{\co4\t{}\tea\milk\coffee,\tea100,\milk010,\coffee001.
\co6 This matrix is the \x{ingradient list of making ingradient}, i.e. do nothing.}
\[defi]{
The \x{identity matrix} is a $n\times n$ square matrix with 1 on the diagonal and 0 elsewhere.
}
\[prop]{For any $n\times m$ matrix $P$, $I_nP=PI_m=P$ .
}
\a{Inverse Matrix}
The chef is wondering if another guest coming with a special request, so he would like a list to produce the ingradient out of meals. 
\vfill
\[columns]{\co 5\textbf{He has a list}

\t{}\milk\coffee,\bean02,\cow10.

\co 5 \textbf{How could he make another list?}
\t{}\bean\cow,\milk??,\coffee??.
}
\vfill
He realize this list should have a property, combinging them should be.
\[columns]{
\co3
\t{}\milk\coffee,\bean02,\cow10.
\co3 \t{}\bean\cow,\milk??,\coffee??.
\co 1=\co3\t{}\bean\cow,\bean{1}{0},\cow{0}{1}.
}
\a\aa
\[defi]{For a $n\times n$ matrix $A$, an inverse is a matrix $B$, such that 
$$
AB=BA=I_n.
$$
If such a $B$ exists, $A$ is called \inve and denote the inverse as $A^{-1}$.
}
 

\a{Simutaneous Row Operation}
For the product $C=AB$, changing raw materials only affect rows of $A$ and $C$ by certain simutaneous row operations. the matrix $B$ would not change since it does not depend on raw materials. Therefore,
\raisebox{2cm}{
\[prop]{The equality $C=AB$ will still be true if we perform arbitrary simutaneous row operation on $A$ and $C$
}

}

\a\aa
\[equation*]{\overbrace{\t{}\milk\coffee\tea,\leaf{0}{0}{2},\lemon{0}{0}{1},\bean{0}{2}{0},\cow{1}{0}{0}.}^{\text{left factor}}\overbrace{\t{}\bento\soup,
\milk{2}{1},
\coffee{0}{2},
\tea{1}{1}.}^{\text{right factor}}\quad=
\overbrace{\t{}\bento\soup,\leaf{2}2,\lemon{\alert1}{\alert1},\bean{0}4,\cow{2}1.}^{\text{product}}}
\a\aa
If $A$ is invertible, we can write $B=A^{-1}C$, this actually tells us that
\[cor]{If $A$ is invertible, the product $A^{-1}C$ does not change if we perform simutaneous row operation on $A$ and $C$. 
}
\a\aa
Let me show you an application of this in calculation
$$
\m 110,001,010.^{-1}\m123,121,029.
$$

$$
\xequal{r_1\mapsto r_1-r_3}\m 100,001,010.^{-1}\m10{-6},121,029.
$$

$$
\xequal{r_2\leftrightarrow r_3}\m 100,010,001.^{-1}\m10{-6},029,121.
$$

$$
=\m10{-6},029,121.
$$

\a{Simutaneous Column Operation}
Similarly, Column operations corresponding to updating list when changing final meals by its order, amount, or packing them together. For the product $C=AB$, when final meals changes, the matrices affected is $C$ and $B$. $A$ is the list of making intermediates, it does not change.
\[prop]{The equality $C=AB$ will still be true if we perform arbitrary simutaneous column operation on $B$ and $C$
}
If $B$ is invertible, we can write $A=CB^{-1}$, this actually tells us that
\[cor]{If $B$ is invertible, the product $CB^{-1}$ does not change if we perform simutaneous column operation on $B$ and $C$. 
}
\a{Elementary Matrices}
We will introduce elementary matrices.

\a\aa
Let $A$ be an $m\times n$ matrix, and let $I_m$ be $m\times m$ identity matrix. We have a equation
$$
A = I_mA
$$
After some simutaneous row operation, $A$ changes to $A'$ and $I_m$ changes to $E$, by what we discussed before, the following equation is true
$$
A' = EA
$$
\textbf{We may view this fact by another perspective}: The product $EA$ is the same as applying row operations recorded in $E$ to $A$.
\a\aa
\textbf{Experiment:} 
$$
\m 10,01. \m 24,13. = \m 24,13. 
\implies
\m 11,01. \m 24,13. = \m 37,13.
$$
\textbf{Explaination from another perspective:}
\t{left factor}{row operation recorded}{product: apply row operation},
{${\m11,01.}$}{add 2nd row to 1st row.}{${\m 11,01. \m {\x2}{\x4},13. = \m {\x3}{\x7},13.}$}.




\a\aa

\begin{prop}
Suppose $E$ is a matrix obtained by applying some row operations from $I_n$, then $EA$ is exactly the matrix by applying the same row operations on $A$.
\end{prop}

\begin{defi}
An elementary matrix $E$ is a matrix after one-step row operation from identity matrix.
\end{defi}

\a\aa

\t{Name}{Elementary matrices example}{Inverse},
{Switching}{{$\m100,001,010.$}}{{$\m100,001,010.$}},
{Multiplying}{{$\m100,020,001.$}}{{$\m100,0{\frac12}0,001.$}},
{Adding}{{$\m100,011,001.$}}{{$\m100,01{-1},001.$}}.

\a\aa
\textbf{Exercise}: What is the following product? calculate in mind.
$$
\m11,01.\m10,02.\m 12,34.
$$

\textbf{Solution}
$$
\m11,01.\m10,02.\m 12,34.=\m11,01.\m 12,68.=\m7{10},68.
$$

\a\aa

Look at the following matrix(maybe not elementary). By thinking how it changed from identity matrix, what kind of row operation does it record?


\t{left-factor}{row operation recorded.},
{${\m01,10.}$}{Switch two rows},
{${\m01.}$}{Delete the first row},
{${\m11,11.}$}{},
{${\m20,01.}$}{},
{${\m01,01.}$}{}.


\a\aa
The same logic is also for columns. Recall that $AB=C$ stays as an equation when applying column oprtations on $B$ and $C$.
\textbf{Experiment:} 
$$
 \m 24,13.\m 10,01. = \m 24,13. 
\implies
 \m 24,13.\m 11,01. = \m 26,14.
$$
\textbf{Explaination from another perspective:}
\t{right factor}{row operation recorded}{product: apply row operation},
{${\m11,01.}$}{add 1st column to 2nd column.}{${ \m {\x2}{4},{\x1}3.\m 11,01. = \m {2}{\x6},1{\x4}.}$}.


\a\aa

\begin{prop}
Suppose $E$ is a matrix obtained by applying some column operations from $I_n$, then $AE$ is exactly the matrix by applying the same column operations on $A$.
\end{prop}

\a\aa
\t{right-factor}{column operation recorded.},
{${\m01,10.}$}{Switch two columns},
{${\m0,1.}$}{},
{${\m11,11.}$}{},
{${\m01,01.}$}{}.


\a\aa

What is the meaning of the following product? From the perspective of column action?

$$
\underbrace{A}_{3\times 4 \text{Matrix}} \m 0,0,1,0.
$$

What is the meaning of the following product? From the perspective of column action?

$$
\underbrace{A}_{3\times 4 \text{Matrix}} \m0000,0000,0010,0000.
$$

\a\aa
What is the meaning of the following matrix
$$
\m 10,01,10,01. \underbrace{A}_{2\times 2\text{Matrix}}
$$
How about the following
$$
\m 10,10,10,10. \underbrace{A}_{2\times 2\text{Matrix}}
$$
Is this product depends on the second row of $A$?
\a\aa

Let $A$ be a $m\times n$ matrix, and let $l$ be the 2nd row of $A$, can you write $l$ in terms of a matrix multiplication?
\vfill
Let $\vec v$ be the 3rd column of $A$, can you write $\vec v$ into a matrix multiplication?

\aaa
